%-------------------------------------------------------------------------------
\section{Background and Signal Modeling}
\label{sec:backgroundAndSignalModelling}
%-------------------------------------------------------------------------------
After event selection, the largest remaining background comes from the production of a \w boson in association with
jets ($W$+jets) with multi-jet QCD production contributing to a lesser extent. Small contributions arise from single top quark, $Z$+jets and diboson ($WW,WZ,ZZ$) production. The absolute yield of all considered backgrounds rises in the 1 exclusive \bt tag region while the contributions from single top and $W$+jets backgrounds also increase relatively with respect to the other backgrounds.

Multi-jet events contribute to the selected sample via the mis-identification of a jet or a photon as an electron or the presence of a non-prompt lepton, e.g.~from a semileptonic \bt- or $c$-hadron decay, and the corresponding yield is estimated using a data-driven matrix method ~\cite{ttbar_3pb}.
%The \w+jets background is estimated using simulation. %The heavy flavor component of the \w+jets background is determined from data and the overall yield normalized using a data-driven charge asymmetry method.

%Details on the estimation of the multijet background are given in Sect.~\ref{sec:DataDrivenBackground}.
%Yields from the remaining backgrounds (as well as the \ttbar signal) are determined from Monte Carlo simulations normalised to their appropriate theoretical cross sections.

%In the di-lepton channel backgrounds other than $t\bar{t}$+jets production
%arise mainly from $Z$+jets production and from
%the \w+jets and \ttbar\ production with a single lepton in the final state.
%The latter contain non-prompt leptons that pass lepton isolation
%requirement or misidentified leptons arising from jets. In this analysis
%their yield is estimated using simulation as described in
%Sect.~\ref{sec:DataDrivenBackground}.
%$Z$+jets background is estimated from simulation corrected for the discrepancy
%of the $Z$ \pt\ spectrum between data and simulation. Heavy flavor component of
%$Z$+jets background is also determined from data (see Appendix~\ref{sec:Vjets}).

Small background contributions coming from single top quark, $Z$+jets and diboson production are estimated from simulation and normalized to their theoretical cross sections.

Samples of $W/Z$+jets events at 8 TeV are generated using the {\sc Alpgen v2.14}~\cite{alpgen} leading-order (LO) generator and the {\sc CTEQ6L1} PDF set~\cite{cteq6}. Parton shower and fragmentation are modeled with {\sc Pythia} 6.425~\cite{Sjostrand:2006za} for $W$+jets and $Z$+jets production. To avoid double-counting of partonic configurations generated by both the matrix-element calculation and the parton-shower evolution, a parton-jet matching scheme (``MLM matching")~\cite{mlm} is employed. Since final states with hadronic $Z$ decays of $WZ$ are missing in {\sc Alpgen}+{\sc Herwig} production, {\sc Sherpa} samples generated with massive $b$ and $c$ quarks (i.e. $b$ and $c$ quark masses are set to their PDG values in the matrix element calculation as opposed to being set massless) are used for these decay modes.

The \w+jets samples are generated with up to five additional partons, separately for $W$+light jets, $Wb\bar{b}$+jets, $Wc\bar{c}$+jets, and $Wc$+jets. The overlap between $Wq\bar{q}$ ($q=b,c$) events generated from the matrix element calculation and those generated from parton-shower evolution in the $W$+light jet samples is removed by using the angular separation between the extra heavy quarks: if $\Delta R(q,\bar{q})>0.4$, the matrix-element prediction is used, otherwise the parton-shower prediction is used. The $W$+jets background is normalized using data-driven calibration factors which take into account the charge of the lepton and heavy flavor components of the associated jets. The method and factors were derived in the 8 TeV ATLAS \ttbar charge asymmetry analysis ~\cite{Juste:1647184}.

The $Z$+jets background is generated using the {\sc Alpgen v2.14}~\cite{alpgen} leading-order (LO) generator with up to five additional partons separated by different parton flavors and is normalized to the inclusive NNLO theoretical cross section~\cite{vjetsxs}.

The $WW/WZ/ZZ$+jets samples are generated using {\sc Sherpa} with up to three additional partons in the hard process and are normalized to their NLO theoretical cross sections~\cite{dibosonxs}. Heavy flavor quarks (\bt and $c$) are treated as massive in the matrix element calculation and generation of all diboson samples.

The $t\bar{t}$ sample is generated using {\sc Powheg} NLO generator~\cite{powheg,powbox1,powbox2} with the {\sc CT10} PDF set assuming a top quark mass of $172.5\gev$ and setting the $h_{damp}$ factor (the parameter responsible for high \pt radiation damping in the {\sc POWHEG-BOX} generator) equal to the top quark mass. {\sc Powheg} is interfaced to {\sc Pythia} 6.425~\cite{Sjostrand:2006za} with the {\sc CTEQ61L} set of parton distribution functions and Perugia2011C underlying event tune. The sample is normalized to the theoretical calculation performed at next-to-next-to leading order (NNLO) in QCD including resummation of next-to-next-to-leading logarithmic (NNLL) soft gluon terms with top++2.0~\cite{ref:xs1,ref:xs2,ref:xs3,ref:xs4,ref:xs5,Czakon:2011xx}. At 8 TeV, the \ttbar cross-section is calculated to be $253^{+15}_{-16}$~pb. 

Effects due to PDF variations, choice of $\alpha_S$, and the input top quark mass are taken as systematic uncertainties. The PDF and $\alpha_S$ uncertainties are calculated using the PDF4LHC prescription~\cite{ref:pdf4lhc} with the MSTW2008 68\% CL NNLO~\cite{mstw1,mstw2}, CT10 NNLO~\cite{Lai:2010vv,ct102} and NNPDF2.3 5f FFN~\cite{nnpdf} PDF sets added in quadrature to the scale uncertainty. Unlike multi-leg generators, {\sc Powheg} is expected to describe jet multiplicity properly only for \ttbar accompanied by up to two jets. Nevertheless when interfaced with {\sc Pythia}, {\sc Powheg} provides good description of jet multiplicity in data up to much higher jet multiplicities and reproduces the associated heavy flavor fraction in \ttbar production despite the fact that heavy flavor component originates only from the parton shower~\cite{Aad:2015gra}. %The  {\sc Powheg}+{\sc Pythia} \ttbar sample is re-weighted as a function of top quark \pt to remove discrepancies in the high top quark and \ttbar \pt regions as observed in data~\cite{topdiff_7TEV}. The details of the correction procedure and associated systematic uncertainties are described in Appendix~\ref{app:topmodel}.

%A detailed study comparing the $t\bar{t}$ sample generated using {\sc Powheg} NLO generator~\cite{powheg,powbox1,powbox2}
%with the {\sc CT10} PDF set assuming top quark mass of $172.5\gev$ and $h_{damp}$ factor set to infinity with the default signal sample mentioned above is presented in Appendix \ref{app:hdamp_comparison}.

Samples of single top quark backgrounds corresponding to the $s$-channel, $t$-channel and $Wt$ production mechanisms are generated with {\sc Powheg}~\cite{powheg,powbox1,powbox2} using the {\sc CT10} PDF set~\cite{ct10}.  In the case of the $Wt$-channel, the nominal sample uses a diagram removal approach to handle the interference arising at NLO. All samples are interfaced to {\sc Pythia} 6.425~\cite{Sjostrand:2006za} with the {\sc CTEQ61L} set of parton distribution functions and Perugia2011C underlying event tune. Overlaps between the \ttbar\ and $Wt$ final states are removed~\cite{mcatnlo_2}. The single top quark samples are normalized to the approximate NNLO theoretical cross sections~\cite{stopxs,stopxs_2} using the {\sc MSTW2008} NNLO PDF set.

All event generators using {\sc Herwig} are also interfaced to {\sc Jimmy} v4.31~\cite{jimmy} to simulate the underlying event. All simulated samples utilize {\sc Photos 2.15}~\cite{PhotosPaper} to simulate photon radiation and {\sc Tauola 1.20}~\cite{TauolaPaper} to simulate $\tau$ decays. All simulated samples (including the \ttbar signal) include multiple $pp$ interactions and are processed through a simulation~\cite{Aad:2010ah} of the detector geometry and response using {\sc Geant4}~\cite{Agostinelli:2002hh}. All simulated samples are processed through the same reconstruction software as the data. Simulated events are
corrected so that the object identification efficiencies, energy scales and energy resolutions match those determined in data control samples.

Table~\ref{tab:NBparam} provides a summary of basic parameters of the MC samples used in the analysis. %Full list of MC samples is available in Appendix \ref{app:mcSamples}.

\begin{table}
\small
\centering     % 1 2 3 4 5                                                                                                                                                                                     
\begin{tabular}{|c|c|c|c|c|}
\hline
Sample & Generator & PDF & Shower & Normalization \\
\hline
%\tth    & HELAC-Oneloop  & CT10  & Pythia $8.1$ & NLO\\
\ttbar + jets &  Powheg & CT10 & Pythia $6.425$ & NNLO+NNLL~\cite{ttbarNorm1,Frixione:2007vw,ttbarNorm3}\\
\w + jets & Alpgen & CTEQ$6$L$1$ & Pythia $6.426$ & 8 TeV charge asymmetry CAs~\cite{Juste:1647184}\\
$Z$ + jets & Alpgen & CTEQ$6$L$1$ & Pythia $6.426$ & NLO~\cite{ZjetsNorm} \\
Single top (s-channel, Wt) & Powheg & CT10 & Pythia $6.426$ & aNNLO~\cite{Kidonakis:2011wy, Kidonakis:2010ux,singleTopNorm3}\\
Single top (t-channel) & Powheg & CT10 & Pythia $6.427$ & aNNLO~\cite{Kidonakis:2011wy, Kidonakis:2010ux,singleTopNorm3}\\
Diboson & Sherpa & CT10 & Sherpa  &  NLO~\cite{dibosonNorm} \\
\hline
\end{tabular}
\caption{A summary of basic generator parameters used to simulate various
processes}
\label{tab:NBparam}
\end{table}
