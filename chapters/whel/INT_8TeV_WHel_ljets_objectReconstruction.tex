%-------------------------------------------------------------------------------
\section{Object Reconstruction}
\label{sec:objectReconstruction}
%-------------------------------------------------------------------------------
The main physics objects considered in this analysis are electrons, muons,
jets, \bt jets and missing transverse momentum. A brief summary of the
main reconstruction and identification criteria applied for each
of these physics objects is given below.

Electron candidates~\cite{ElectronPerformance} are reconstructed from
energy deposits (clusters) in the electromagnetic calorimeter that are
associated to reconstructed tracks in the inner detector.  They are
required to have $\pt > 25 \gev$ and $|\eta_{\rm cluster}| < 2.47$ (where $|\eta_{\rm cluster}|$ is the
pseudorapidity of the calorimeter cluster associated with the electron
candidate).  Candidates in the calorimetry transition region $1.37 < |\eta_{\rm cluster}| < 1.52$ are excluded.  To reduce the background
from non-prompt electrons, i.e.~from decays of hadrons (including
heavy flavor) produced in jets, electron candidates are also required
to be isolated. An $\eta$-dependent, 90\% efficient isolation cut is made
on the sum of transverse energy deposited in a radius of $\Delta R = 0.2$
around the calorimeter cells associated to the electron. The sum of energy
deposited around the cells associated with the electron cluster is corrected
for leakage from the electron cluster itself.  A further 90\%
efficient isolation cut is made on the track transverse momentum
($\pt$) sum around the electron in a cone of radius $\Delta R = 0.3$ ($p_{T}^{cone30}$). All candidate electrons
are required to pass the tight++ ID requirement. Lastly, the longitudinal
impact parameter of the electron track with respect
to the selected event primary vertex, $z_{0}$, is required to be less
than 2 mm.% (see Section~\ref{sec:EventSelection}).

%In the dilepton channel, electron selection was optimized for this search
%to increase the acceptance since that fake rate in this channel is quite small.
%The details of the optimisation study can be found in Appendix~\ref{app:electronID}.                                                                                                                          
%Optimised selection requires electrons that pass an improved likelihood-based electron identification 
%(medium working point) 
%where electrons are selected from the ``tight++'' definition.
%The isolation in the dilepton channel is looser that in the single lepton one and is based on the tracker information only. It requires that the ratio of the sum of track transverse momenta in a cone \ptcth\ around the
%electron track to the \pt\ of the electron to be less than 0.12 (\ptcpt $\leq$ 0.12). Furthermore, the second leading lepton, either muon or electron is only required to have $\pt > 15 \gev$.

Muon candidates are reconstructed from track segments in the various
layers of the muon spectrometer, and matched with tracks found in the
inner detector. The final candidates (taken from the $muid$ collection) refitted using the complete
track information from both detector systems, and required to satisfy
$\pt > 25\gev$ and $|\eta|<2.5$. They are referred to as ``combined'' muons.
Additionally, muons are required to
be separated by $\Delta R > 0.4 $ from any selected jet (see below).

Furthermore, muons are required to satisfy a $\pt$-dependent
track-based isolation
requirement that has good performance under high pileup
conditions or in boosted configurations where the muon is close
to a jet:  the scalar sum of the track $\pt$ in a cone of
variable radius $\Delta R < 10\gev/\pt^\mu$ around the muon
(excluding the muon track itself) must be less than 5\% of the muon $\pt$.
Muons  are required to have a hit pattern in the inner detector
consistent with a well-reconstructed track. Analogously to the electrons,
the muon
track longitudinal impact parameter with respect to the primary
vertex, $z_{0}$, is required to be less than 2 mm.

Jets are reconstructed with the anti-$k_t$
algorithm~\cite{ref:Cacciari2008,ref:Cacciari2006,ref:fastjet} with a
radius parameter $R=0.4$ from calibrated topological
clusters~\cite{ATLASTechnicalPaper} built from energy deposits in the
calorimeters.  Prior to jet finding, a local cluster calibration
scheme~\cite{LCW1,LCW2} is applied to correct the topological cluster
energies for the effects of non-compensation, dead material and
out-of-cluster leakage. The corrections are obtained from simulations
of charged and neutral particles.  After energy
calibration~\cite{ATLASJetEnergyMeasurement} jets are required to have
$\pt > 25\gev$ and $|\eta| < 2.5$.

To avoid selecting jets from secondary $pp$ interactions, a selection
on the so-called ``jet vertex fraction" (JVF) variable above 0.5
is applied to jets with $\pt <50\gev$ and $|\eta|<2.4$. This requirement ensures that at least 50\% of the sum of the $\pt$ of tracks
with $\pt>1\gev$ associated with a jet comes from tracks compatible
with originating from the primary vertex.

During jet reconstruction,
no distinction is made between identified electrons and jet energy
deposits.  For jets within $\Delta R < 0.2$ of selected electrons, the single closest jet is discarded to avoid double-counting of electrons as jets.  After this removal procedure, electrons within $\Delta R < 0.4$ of all remaining jets are removed.

Jets are identified as originating from the hadronization of a \bt
quark (\bt tagged) via an algorithm ~\cite{Aad:2015ydr}
using multivariate techniques to combine information from the impact
parameters of displaced tracks as well as topological properties of
secondary and tertiary decay vertices reconstructed within the jet.
The working point used for this search corresponds to 70\% efficiency to tag
a \bt quark jet, with a light jet rejection factor of $\sim$130 and
a charm jet rejection factor of 5, as determined for \bt tagged jets with
\pt > 20 \gev and $|\eta|<2.5$ in simulated \ttbar events. The simulated \bt tagging efficiency is corrected to that measured in data using calibrations produced for the 6 bin \ttbar PDF calibration (combined\_pdf\_dijet\_7) ~\cite{ATLAS-CONF-2014-004}. This calibration uses a combinatorical likelihood to measure \bt tagging efficiency in a data sample of dileptonic \ttbar events and helps reducing \bt tagging uncertainties by considering correlations between the measured jets.
%To minimize statistical overlap with the dilepton channel the version of PDF calibration that uses only data with exactly 2 jets is used.

The missing transverse momentum (\met) is used to estimate the
transverse momentum of an assumed neutrino originating from the decay
of \w bosons in the \ttbar final state.  \met is itself reconstructed
by first matching each calorimeter energy cluster with either a
reconstructed lepton or jet.  Failing this, the cluster is left
unassociated.  The remaining unassociated clusters are then calibrated
for energy losses in un-instrumented regions and for different
responses of the calorimeters to electromagnetic and hadronic shower
components.  This calibration scheme is similar to that described in
Ref.~\cite{ref:ATLAS-CONF-2011-080}. \met is calculated from a
vector sum of the calibrated cluster momenta, together with a term
associated with muon momenta.

A complete list of packages used for object reconstruction and
calibration can be found in Appendix ~\ref{app:packages}.
