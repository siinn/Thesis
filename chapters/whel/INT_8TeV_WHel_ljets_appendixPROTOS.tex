%-------------------------------------------------------------------------------
\clearpage
\section{PROTOS Studies}
\label{app:protosStudies}
%-------------------------------------------------------------------------------
This appendix contains a summary of the study done by Andrea Knue in 2013 comparing reweighted Powheg+Pythia with pure states generated by PROTOS. The slides presented at the Top Properties meeting can be found at https://indico.cern.ch/event/308142/contributions/710186/attachments/588190/809526/Talk\_ProtosRew\_TopProp\_20032014.pdf . The following plots come directly from the presentation.

The nominal \ttbar sample used in this analysis is generated from NLO Powheg+Pythia. While the NLO kinematics do a decent job describing the data, Powheg has the drawback of being unable to generate pure helicity states. The LO generator PROTOS can be configured to directly generate pure helicity states, but PROTOS was not used as the nominal sample for this analysis because the leading order kinematics don't correctly reproduce the distributions observed in data. In order to have decently reproduced kinematics, the approach taken is to reweight the Standard Model mixture produced in Powheg+Pythia into pure helicity states. As a closure test of the reweighting procedure, a comparison was done at truth and reco level of the reweighted templates (produced from NLO Powheg+Pythia) with pure templates generated directly using the LO generator PROTOS. The first comparison is done at truth-level before selection, and the results can be seen in Figures \ref{fig:powhegVprotos_el_truth} and \ref{fig:powhegVprotos_mu_truth}.

\begin{figure}[htbp]
\begin{center}
		\includegraphics[height=60mm]{figures/appendixPROTOS/f0_protosVpowheg_el_truth}
		\includegraphics[height=60mm]{figures/appendixPROTOS/fL_protosVpowheg_el_truth}
		\includegraphics[height=60mm]{figures/appendixPROTOS/fR_protosVpowheg_el_truth}
	\caption{A comparison of the truth level, electron channel $\cos\theta^*$ distributions generated directly by PROTOS and produced via reweighting the Powheg+Pythia sample. Good agreement is seen between PROTOS and reweighted Powheg+Pythia.}
	\label{fig:powhegVprotos_el_truth}
\end{center}	
\end{figure}

\begin{figure}[htbp]
\begin{center}
		\includegraphics[height=60mm]{figures/appendixPROTOS/f0_protosVpowheg_mu_truth}
		\includegraphics[height=60mm]{figures/appendixPROTOS/fL_protosVpowheg_mu_truth}
		\includegraphics[height=60mm]{figures/appendixPROTOS/fR_protosVpowheg_mu_truth}
	\caption{A comparison of the truth level, muon channel $\cos\theta^*$ distributions generated directly by PROTOS and produced via reweighting the Powheg+Pythia sample. Good agreement is seen between PROTOS and reweighted Powheg+Pythia.}
	\label{fig:powhegVprotos_mu_truth}
\end{center}	
\end{figure}


Next, the templates are compared after event selection and reconstruction since the acceptance and isolation effects can have a large effect on the reconstruction efficiencies of different helicity states. This is because, for example, left-handed (right-handed) \w bosons emit charged leptons in the opposite (same) direction as their own momentum resulting in softer (harder) observed \pt spectrums. The reco-level templates are compared in Figures \ref{fig:powhegVprotos_el_reco} and \ref{fig:powhegVprotos_mu_reco}.

\begin{figure}[htbp]
\begin{center}
		\includegraphics[height=60mm]{figures/appendixPROTOS/f0_protosVpowheg_el_reco}
		\includegraphics[height=60mm]{figures/appendixPROTOS/fL_protosVpowheg_el_reco}
		\includegraphics[height=60mm]{figures/appendixPROTOS/fR_protosVpowheg_el_reco}
	\caption{A comparison of the reco level, electron channel $\cos\theta^*$ distributions generated directly by PROTOS and produced via reweighting the Powheg+Pythia sample. Good agreement is seen between PROTOS and reweighted Powheg+Pythia except at the high end of \fr.}
	\label{fig:powhegVprotos_el_reco}
\end{center}	
\end{figure}

\begin{figure}[htbp]
\begin{center}
		\includegraphics[height=60mm]{figures/appendixPROTOS/f0_protosVpowheg_mu_reco}
		\includegraphics[height=60mm]{figures/appendixPROTOS/fL_protosVpowheg_mu_reco}
		\includegraphics[height=60mm]{figures/appendixPROTOS/fR_protosVpowheg_mu_reco}
	\caption{A comparison of the reco level, muon channel $\cos\theta^*$ distributions generated directly by PROTOS and produced via reweighting the Powheg+Pythia sample. Good agreement is seen between PROTOS and reweighted Powheg+Pythia except at the high end of \fr.}
	\label{fig:powhegVprotos_mu_reco}
\end{center}	
\end{figure}

A clear deviation is observed in \fr near $\cos\theta^*\sim1$. Looking at the object kinematics (Figures \ref{fig:powhegVprotos_el_kin}, \ref{fig:powhegVprotos_lep_kin}), a large deviation occurs for lepton \pt > 150 GeV. Kinematic mismodelling would likely manifest itself in the high \pt tails, and mismodelling at high \pt has a proportionally larger effect in the last bin of $\cos\theta^*$. Reconstruction level comparisons after requiring that the lepton \pt be less than 150 GeV can be found in Figures \ref{fig:powhegVprotos_el_reco_pt150} and \ref{fig:powhegVprotos_el_reco_pt150}.

\begin{figure}[htbp]
\begin{center}
		\includegraphics[height=60mm]{figures/appendixPROTOS/leptonEt_el_reco}
		\includegraphics[height=60mm]{figures/appendixPROTOS/nJets_el_reco}
		\includegraphics[height=60mm]{figures/appendixPROTOS/nbTags_el_reco}
	\caption{A comparison of some reco level, electron channel kinematic distributions generated by PROTOS and Powheg+Pythia sample. A deviation is observed in the high end tail of the lepton \pt.}
	\label{fig:powhegVprotos_el_kin}
\end{center}	
\end{figure}

\begin{figure}[htbp]
\begin{center}
                \includegraphics[height=60mm]{figures/appendixPROTOS/leptonEt_lep_reco}
		\includegraphics[height=60mm]{figures/appendixPROTOS/nJets_lep_reco}
		\includegraphics[height=60mm]{figures/appendixPROTOS/nbTags_lep_reco}
	        \caption{A comparison of some reco level, electron+muon channel kinematic distributions generated by PROTOS and Powheg+Pythia sample. A deviation is observed in the high end tail of the lepton \pt.}
        	\label{fig:powhegVprotos_lep_kin}
\end{center}	
\end{figure}


\begin{figure}[htbp]
\begin{center}
		\includegraphics[height=60mm]{figures/appendixPROTOS/f0_protosVpowheg_el_reco_pt150}
		\includegraphics[height=60mm]{figures/appendixPROTOS/fL_protosVpowheg_el_reco_pt150}
		\includegraphics[height=60mm]{figures/appendixPROTOS/fR_protosVpowheg_el_reco_pt150}
	\caption{A comparison of the reco level, electron channel $\cos\theta^*$ distributions (after requring lepton \pt < 150 GeV) generated directly by PROTOS and produced via reweighting the Powheg+Pythia sample. Good agreement is seen between PROTOS and reweighted Powheg+Pythia.}
	\label{fig:powhegVprotos_el_reco_pt150}
\end{center}	
\end{figure}

\begin{figure}[htbp]
\begin{center}
		\includegraphics[height=60mm]{figures/appendixPROTOS/f0_protosVpowheg_mu_reco_pt150}
		\includegraphics[height=60mm]{figures/appendixPROTOS/fL_protosVpowheg_mu_reco_pt150}
		\includegraphics[height=60mm]{figures/appendixPROTOS/fR_protosVpowheg_mu_reco_pt150}
	\caption{A comparison of the reco level, muon channel $\cos\theta^*$ distributions (after requring lepton \pt < 150 GeV) generated directly by PROTOS and produced via reweighting the Powheg+Pythia sample. Good agreement is seen between PROTOS and reweighted Powheg+Pythia.}
	\label{fig:powhegVprotos_mu_reco_pt150}
\end{center}	
\end{figure}

After the cut on lepton \pt (i.e. excluding the poorly modeled LO high \pt region), the PROTOS and Powheg samples agree within uncertainty. From this, we conclude that reweighting the NLO Powheg sample gives reasonable results compared to the purely generated helicity states available in LO PROTOS.
