%-------------------------------------------------------------------------------
\section{Introduction}
\label{sec:intro}
%-------------------------------------------------------------------------------
Discovered in 1995 at the Tevatron~\cite{CDF, D0}, the top quark is the heaviest particle in the Standard Model of particle physics. In proton-proton collisions at the LHC~\cite{LHC}, top quarks are produced in pairs through the strong interaction and individually through electroweak processes. With a mass of 173.34 $\pm$ 0.76 GeV~\cite{arXiv:1403.4427} 
%172.3$\pm$1.1 GeV~\cite{arXiv:1403.4427} 
(close to the electroweak symmetry breaking scale~\cite{Beneke:2000hk}\footnote{The electroweak symmetry breaking scale is of order $\sim$100 GeV.}), measurements of the top quark properties can provide important tests of the Standard Model. Due to its extremely short lifetime (${\sim 10^{-25}}$ seconds), the top quark decays before hadronization, passing its spin information to the decay products. The top quark decays almost exclusively into a \w boson and a \bt quark, and thus studying the \Wtb vertex is of importance in the search for new interactions.

The \w boson can be produced with left-handed, longitudinal, or right-handed helicity. The SM predictions for the \w boson helicity fractions at next-to-next-to-leading-order (NNLO) in the strong coupling constant, including the finite \bt quark mass and electroweak effects, are \fo = 0.687 $\pm$ 0.005, \fl = 0.311 $\pm$ 0.005, and \fr = 0.0017 $\pm$ 0.0001~\cite{nnlo_theory} for a bottom quark mass \mb = 4.8 GeV and a top quark mass \mt = 172.8 $\pm$ 1.3 GeV. These fractions can be extracted from measurements of the angular distribution of the decay products of the top quark. The helicity angle $\theta^{*}$ is defined~\cite{Fischer:1998gsa} as the angle between the momentum direction of the charged lepton (down-type quark) from the decay of the leptonic (hadronic) \w boson and the reversed momentum direction of the \bt quark from the decay of the top quark, both boosted into the \w boson rest frame. The distribution of the cosine of the helicity angle has a dependence on the helicity fractions given by

\begin{equation}
\frac{1}{N}\frac{dN} {d\cos{\theta^{*}}} = \frac{3}{8} (1-\cos{\theta^{*}})^2 \fl +  \frac{3}{4} \sin{^{2}\theta^{*}} \fo +  \frac{3}{8} (1+\cos{\theta^{*}})^2 \fr,
\label{eq:Cos}
\end{equation}
The helicity fractions, $F_{\text{i}}$, can then be extracted via a fit of the reconstructed \ttbar candidate events measured in data.

A kinematic likelihood fit is used to determine the best association of \bt jets, light jets, and lepton candidates to the top quark and anti-quark decay hypotheses, considering the momentum imbalance as the existence of a neutrino originating from the leptonically decaying \w boson. Both ATLAS~\cite{Aad:2012ky} and CMS~\cite{Chatrchyan:2013jna} produced 7 TeV measurements using leptonically decaying \w's in semileptonic \ttbar decays. CMS also produced a measurement at 8 TeV~\cite{Khachatryan:2016fky} using the leptonically decaying $W$ bosons produced in semileptonic \ttbar events. For this analysis, the helicity fractions were measured using both the leptonically and hadronically decaying \w's from the \ttbar decay in both the electron and muon channels. While previous analyses from D0~\cite{Abazov:2007ve} and CMS~\cite{Chatrchyan:2013jna} have performed indirect helicity measurements using the hadronic \w decay in semileptonic \ttbar events, this analysis is the first to perform this measurement directly by identifying the identities (up type or down type) of the daughter jets from the hadronic \w decay.

After event selection and reconstruction, the helicity fractions were measured using a template fit method. Signal and background templates were generated using Monte Carlo (MC) simulation except for contributions from mis-identified leptons (QCD processes) which were generated via a data-driven method. Pure helicity templates for the signal sample were obtained from using a Monte Carlo re-weighting procedure described in Section~\ref{sec:templateFitting}. The final fit to data was performed using a binned likelihood fit.

