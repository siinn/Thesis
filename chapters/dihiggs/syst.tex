\section{Systematic uncertainties}
\label{sec:dihiggs_syst}
%-------------------------------------------------------------------------------

This section describes the sources of systematic uncertainty
considered in the analysis. These uncertainties are divided into three
categories: experimental uncertainties, uncertainties on the modeling
of background processes estimated from simulation, and theoretical
uncertainties on the signal processes. In the statistical analysis
each systematic uncertainty is treated as a nuisance parameter the
names of which are defined below. These systematic variations are
estimated on the final expected yield in the signal regions.

\subsection{Experimental uncertainties}

Each reconstructed object has several sources of uncertainties, each
of which are evaluated separately. Wherever possible, we follow the
latest available recommendations from the combined performance (CP)
groups. The leading instrumental uncertainties are
the uncertainty on the $b$-tagging efficiency and the jet energy scale
(JES). The summary of experimental Uncertainties is presented in
Table~\ref{tab:syst_summary_sources}.
 
 \begin{table}[h]
\centering
\small
\begin{center}
\begin{tabular}{|l|l|l|l|}
\hline
Source        & Description                          & Analysis Name                         \\ 
\hline
%Muons         & Trigger                                 &    Muons\_Trig                              \\
Muons         & \pt resolution MS                 &   MUONS\_MS                           \\ 
Muons         & \pt resolution ID                   &   MUONS\_ID                             \\ 
Muons         & \pt scale                               &   MUONS\_SCALE                     \\ 
Muons         & Isolation efficiency SF         &   MUON\_ISO\_SYS               \\ 
Muons         & Isolation efficiency SF         &   MUON\_ISO\_STAT              \\ 
Muons         & Reconstruction efficiency SF         &  MUON\_EFF\_SYS             \\ 
Muons         & Reconstruction efficiency SF         &  MUON\_EFF\_STAT           \\ 
Muons         & Trigger efficiency SF            &  MUON\_EFF\_TrigStatUncertainty \\ 
Muons         & Trigger efficiency SF            &  MUON\_EFF\_TrigSystUncertainty  \\ 
%Muons         & TTVA efficiency SF         &   MUON\_TTVA\_SYS              & \\ 
%Muons         & TTVA efficiency SF         &   MUON\_TTVA\_STAT             & \\ 
\hline
Electrons         & \pt resolution                           &   EG\_RESOLUTION\_ALL       \\ 
Electrons         & \pt scale                                  &   EG\_SCALE\_ALL                   \\ 
Electrons         & Isolation efficiency SF             &   EL\_EFF\_Iso\_TOTAL\_1NPCOR\_PLUS\_UNCOR  \\ 
Electrons         & Reconstruction efficiency SF  &   EL\_EFF\_Reco\_TOTAL\_1NPCOR\_PLUS\_UNCOR  \\ 
Electrons         & Trigger efficiency SF               &   EL\_EFF\_Trigger\_TOTAL\_1NPCOR\_PLUS\_UNCOR  \\ 
Electrons         & Identification efficiency SF      &   EL\_EFF\_ID\_TOTAL\_1NPCOR\_PLUS\_UNCOR  \\ 
\hline
MET           & Soft term                       &   MET\_SoftTrk\_ResoPerp            \\ 
MET           & Soft term                       &   MET\_SoftTrk\_ResoPara            \\ 
MET           & Soft term                       &   MET\_SoftTrk\_ScaleUp               \\ 
\hline
Small-R Jets  & JES strongly reduced            &   JET\_SR1\_JET\_GroupedNP\_1                   \\ 
Small-R Jets  & JES strongly reduced            &   JET\_SR1\_JET\_GroupedNP\_2                   \\ 
Small-R Jets  & JES strongly reduced            &   JET\_SR1\_JET\_GroupedNP\_3                   \\ 
Small-R Jets  & JES strongly reduced            &   JET\_SR1\_JET\_EtaIntercalibration\_NonClosure  \\ 
Small-R Jets  & Energy resolution                  &   JET\_JER\_SINGLE\_NP                               \\ 
Small-R Jets  & JVT efficiency SF                  &   JvtEfficiency                                                   \\ 
\hline
$b$-tagging     & Flavor tagging scale factors    &    FT\_EFF\_Eigen\_Light\_0                               \\
$b$-tagging     & Flavor tagging scale factors    &    FT\_EFF\_Eigen\_Light\_1                               \\
$b$-tagging     & Flavor tagging scale factors    &    FT\_EFF\_Eigen\_Light\_2                               \\
$b$-tagging     & Flavor tagging scale factors    &    FT\_EFF\_Eigen\_Light\_3                               \\
$b$-tagging     & Flavor tagging scale factors    &    FT\_EFF\_Eigen\_Light\_4                               \\
$b$-tagging     & Flavor tagging scale factors    &    FT\_EFF\_Eigen\_Light\_5                               \\
$b$-tagging     & Flavor tagging scale factors    &    FT\_EFF\_Eigen\_Light\_6                               \\
$b$-tagging     & Flavor tagging scale factors    &    FT\_EFF\_Eigen\_Light\_7                               \\
$b$-tagging     & Flavor tagging scale factors    &    FT\_EFF\_Eigen\_Light\_8                               \\
$b$-tagging     & Flavor tagging scale factors    &    FT\_EFF\_Eigen\_Light\_9                               \\
$b$-tagging     & Flavor tagging scale factors    &    FT\_EFF\_Eigen\_Light\_10                             \\
$b$-tagging     & Flavor tagging scale factors    &    FT\_EFF\_Eigen\_Light\_11                             \\
$b$-tagging     & Flavor tagging scale factors    &    FT\_EFF\_Eigen\_Light\_12                             \\
$b$-tagging     & Flavor tagging scale factors    &    FT\_EFF\_Eigen\_Light\_13                             \\
$b$-tagging     & Flavor tagging scale factors    &    FT\_EFF\_Eigen\_B\_0                               \\
$b$-tagging     & Flavor tagging scale factors    &    FT\_EFF\_Eigen\_B\_1                               \\
$b$-tagging     & Flavor tagging scale factors    &    FT\_EFF\_Eigen\_B\_2                               \\
$b$-tagging     & Flavor tagging scale factors    &    FT\_EFF\_Eigen\_B\_3                               \\
$b$-tagging     & Flavor tagging scale factors    &    FT\_EFF\_Eigen\_B\_4                               \\
$b$-tagging     & Flavor tagging scale factors    &    FT\_EFF\_Eigen\_C\_0                              \\
$b$-tagging     & Flavor tagging scale factors    &    FT\_EFF\_Eigen\_C\_1                               \\
$b$-tagging     & Flavor tagging scale factors    &    FT\_EFF\_Eigen\_C\_2                               \\
$b$-tagging     & Flavor tagging scale factors    &    FT\_EFF\_Eigen\_C\_3                               \\
$b$-tagging     & Flavor tagging scale factors    &    FT\_EFF\_Eigen\_C\_4                               \\
$b$-tagging     & Flavor tagging scale factors    &    FT\_EFF\_Eigen\_extrapolation                           \\
$b$-tagging     & Flavor tagging scale factors    &   FT\_EFF\_Eigen\_extrapolation\_from\_charm     \\
\hline
\hline
\end{tabular}
\caption{ Qualitative summary of the object systematic uncertainties included in this analysis. }
\label{tab:syst_summary_sources}
\end{center}
\end{table}

 
 
 \iffalse
\begin{table}[h]
\centering
\small
\begin{center}
\begin{tabular}{|l|l|l|}
\hline
Source        & Description                     & Analysis Name
  \\ \hline
Muons          & Trigger      & Muons\_Trig \\
Muons         & \pt resolution MS               &   MUONS\_MS                          \\ 
Muons         & \pt resolution ID               &   MUONS\_ID                           \\ 
Muons         & \pt scale                       &   MUONS\_SCALE                        \\ 
Muons         & Isolation efficiency SF         &   MUON\_ISO\_SYS                      \\ 
Muons         & Isolation efficiency SF         &   MUON\_ISO\_STAT                     \\ 
Muons         & Identification efficiency SF    &   MUON\_EFF\_SYS                      \\ 
Muons         & Identification efficiency SF    &   MUON\_EFF\_STAT                     \\ \hline
MET           & Soft term                       &   MET\_SoftTrk\_ResoPerp              \\ 
MET           & Soft term                       &   MET\_SoftTrk\_ResoPara              \\ 
MET           & Soft term                       &   MET\_SoftTrk\_ScaleUp               \\ \hline
Small-R Jets  & JES strongly reduced            &   JET\_GroupedNP\_1                   \\ 
Small-R Jets  & JES strongly reduced            &   JET\_GroupedNP\_2                   \\ 
Small-R Jets  & JES strongly reduced            &   JET\_GroupedNP\_3                   \\ 
Small-R Jets  & Energy resolution               &   JET\_JER\_SINGLE\_NP                \\ \hline
$b$-tagging     & Flavor tagging scale factors    &    FT\_EFF\_Eigen\_Light0                               \\
$b$-tagging     & Flavor tagging scale factors    &    FT\_EFF\_Eigen\_Light1                               \\
$b$-tagging     & Flavor tagging scale factors    &    FT\_EFF\_Eigen\_Light2                               \\
$b$-tagging     & Flavor tagging scale factors    &    FT\_EFF\_Eigen\_Light3                               \\
$b$-tagging     & Flavor tagging scale factors    &    FT\_EFF\_Eigen\_Light4                               \\
$b$-tagging     & Flavor tagging scale factors    &    FT\_EFF\_Eigen\_B0                               \\
$b$-tagging     & Flavor tagging scale factors    &    FT\_EFF\_Eigen\_B1                               \\
$b$-tagging     & Flavor tagging scale factors    &    FT\_EFF\_Eigen\_B2                               \\
$b$-tagging     & Flavor tagging scale factors    &    FT\_EFF\_Eigen\_C0                               \\
$b$-tagging     & Flavor tagging scale factors    &    FT\_EFF\_Eigen\_C1                               \\
$b$-tagging     & Flavor tagging scale factors    &    FT\_EFF\_Eigen\_C2                               \\
$b$-tagging     & Flavor tagging scale factors    &    FT\_EFF\_Eigen\_C3                               \\
$b$-tagging     & Flavor tagging scale factors    &    FT\_EFF\_Eigen\_extrapolation                               \\
$b$-tagging     & Flavor tagging scale factors    &   FT\_EFF\_Eigen\_extrapolation\_from\_charm                               \\
\hline
Modeling      & $t\bar{t}$ ME              & Matching &       TopMCaNLOtt    \\ 
Modeling      & $t\bar{t}$ Scale             &      ScaleVariation  & Top\_Scale      \\ 
Modeling      & $t\bar{t}$ PDF             &    PDFVariation &    Top\_PDF     \\ 
Modeling      & $t\bar{t}$ PS               & Parton Shower &       Top\_PS    \\ 
\hline
\end{tabular}
\caption{ Qualitative summary of the systematic uncertainties included in this analysis. }
\label{tab:syst_summary_sources}
\end{center}
\end{table}
\fi

\subsubsection{Luminosity}
The uncertainty on the integrated luminosity for the 2015 dataset is
approx. 2\% and for the 2016 ICHEP dataset it is 3.7\%.\footnote{\url{https://twiki.cern.ch/twiki/bin/view/Atlas/LuminosityForPhysics}}. The luminosity uncertainty is
applied to those backgrounds estimated from simulation and the signal
samples.

\subsubsection{Trigger}
Systematic uncertainties on the efficiency of electron and muon triggers are
evaluated as recommended by the corresponding combined performance groups as documented here.\footnote{el: \url{https://twiki.cern.ch/twiki/bin/viewauth/AtlasProtected/ElectronEfficiencyRun2} \\ mu:{https://twiki.cern.ch/twiki/bin/view/AtlasProtected/MCPAnalysisGuidelinesMC15}} 

\subsubsection{Muons}
The following systematic uncertainties are applied to muons in estimations based on the simulation:

\begin{itemize}
\item Identification efficiency: The efficiencies are measured with the tag and probe method using the $Z$ mass peak.
\item Energy and Momentum scales: These are also measured with $Z$ mass line shape, and provided by the CP groups. 
\end{itemize}

\subsubsection{Electrons}
Uncertainties  on  the  energy  scale  (EES)  and  resolution  (EER)  of  the  selected  electron  are and taken as recommended by the CP group.  


\subsubsection{Jet uncertainties}
The jet energy uncertainties are derived based on in situ measurements performed during Run1 and from MC simulation extrapolations from Run1 to Run2 conditions ~~\cite{ATL-PHYS-PUB-2015-015}. The jet energy resolution uncertainty is evaluated by smearing jet energies according to the systematic uncertainties of the resolution measurement~~\cite{Aad:2014bia}. The uncertainty in the $b$-tagging efficiency is evaluated by propagating the systematic uncertainty in the measured tagging efficiency for $b$-jets~~\cite{ATLAS-CONF-2014-004}. 


\subsubsection{Missing transverse energy}
The systematic uncertainties related to the missing transverse energy
are obtained by the propagation of the systematic uncertainty on the
objects that build the MET, in particular the muon, electron and jets
energy resolution and scale. 
The resolution and scale of the MET soft-term is broken down into its components: METScale, METResoPara, METResoPerp, and full uncertainties from each component is taken into account in the final fit. 
 
\subsection{Background modeling uncertainties}
Several systematics have been evaluated to take into account the
uncertainties on the modeling of backgrounds. 

\subsubsection{Uncertainties from the modeling of $t\bar{t}$}
\label{sec:topsys}
The dominant background $t\bar{t}$, is normalised in CR1 for the low
mass selection and in CR2 for the high mass selection. MC is used to extrapolate the shapes from the control regions to the signal region, so theoretical uncertainties are related to such extrapolation. PDF and scale uncertainties are evaluated by applying event selection at truth level. The resulting uncertainties are approximately 4\% and included in the final fit.  

Additional uncertainties in $t\bar{t}$ modeling stems from the difference in the matrix element (ME) implementation across generators, hadronisation and fragmentation modeling (called parton shower, PS), and the amount of initial and final state radiation (ISR/FSR). The ME uncertainty is computed by comparing the events generated by \textsc{aMC@NLO} with the events generated by \textsc{Powheg-Box} v2, both interfaced to \textsc{Herwig++}  for parton shower. The difference computed close to the signal region with enough statistics is used. The PS uncertainty is computed by comparing the the nominal \textsc{Powheg+Pythia6} sample with the PS variation \textsc{Powheg+Herwig++} sample in a region close to the SR but with enough statistics. For ISR/FSR, the dedicated \textsc{radHi} and \textsc{radLo} samples with modified \textsc{hDamp} parameter are compared. The sample with the higher impact on the fit is kept as the uncertainty due to ISR/FSR. Table~\ref{tab:ttbarModeling} shows the numbers provided to the fit for the various \ttbar modeling systematics for the low and high mass selections.

\begin{table}
\centering
\begin{tabular}{l|cc}
\hline
Source               &  Low Mass Selection Uncertainty   & High Mass Selection Uncertainty   \\\hline\hline 
Parton Shower        &               46.0                &              26.0                 \\\hline
Matrix Element       &               9.4                 &              4.5                 \\\hline
ISR/FSR              &               9.5                 &             6.5                 \\\hline
PDF                  &               2.4                 &               6.9                 \\\hline
Scale                &               5.2                 &               3.7                 \\\hline\hline

\end{tabular}
\caption{Extrapolation uncertainties from the CR to the SR,  provided to the fit for the \ttbar modeling systematics.}
\label{tab:ttbarModeling}
\end{table}



\iffalse
An uncertainty on the shape of the $m_{hh}$  for $t\bar{t}$ is derived
comparing  the default Powheg+Herwig++ sample with the distribution obtained
using aMC@NLO+Heriwg++ as alternative generator. Additional systematic
uncertainties are evaluated by comparing the nominal sample showered
with Pythia to one showered with Herwig++. Sample with renormalisation
and factorisation scales doubled
and halved are also available and used as systematic uncertainty. 
The Samples used for the systematic uncertainties are the following:

\begin{itemize}
\item mc15\_13TeV.410000.PowhegPythiaEvtGen\_P2012\_ttbar\_hdamp172p5\_nonallhad - Nominal
\item mc15\_13TeV.410003.aMcAtNloHerwigppEvtGen\_ttbar\_nonallhad - TTBarMCNLO
\item mc15\_13TeV.410004.PowhegHerwigppEvtGen\_UEEE5\_ttbar\_hdamp172p5\_nonallhad - TTBarHerwig
\end{itemize}
\fi

\subsubsection{Single top uncertainty}
Theoretical cross section uncertainties of 5.3\% is assigned to the associated $Wt$ production,  3.9\% to the s-channel and 4.2\% to t-channel single top production.  The single top modeling systematic uncertainties have been calculated employing the difference between the nominal and the (available) systematic variation samples in CR and SR. The recommendation is taken from the Top Twiki.{\footnote {\url{https://twiki.cern.ch/twiki/bin/viewauth/AtlasProtected/TopSystematics2015}}} 
The uncertainties vary across Wt, t, and s channels , and for low mass versus high mass selection. At the smallest it is 3.9\% and at the largest 67\%. To avoid too many nuisance parameters, we have added them in quadrature and now use the following: For Low Mass, 16\% in CR and 34\% in SR. For high mass, 66\% in CR and 69\% in SR.


\subsubsection{$W/Z$+jets modeling uncertainty}
\label{sec:WjetsModeling}
Inclusive theoretical uncertainty of 6\% is added in quadrature with 24\% uncertainty for each additional jet. Therefore, in the final SR with 4 or 5 jets, a total uncertainty of 48\% is applied. This procedure has been used in published Top Group analyses~~\cite{Alwall2008}. 

\subsubsection{Diboson uncertainty}
The same procedure as for $W/Z$+jets is followed. 


\subsubsection{QCD uncertainty}
An overall 30\% uncertainty is assigned to the QCD multijet background. 


\subsection{Model uncertainties on the signal}

The systematic uncertainties on the modeling of the high mass
$H\rightarrow hh$ signal sample have been evaluated checking the
acceptance change when spanning on the CT10 PDF error set, they amount
to 0.4\% while the uncertainies due to missing higher order has been
computed varying normalisation and factorisation scales by a factor
two and they amount to 0.6\%.

\clearpage