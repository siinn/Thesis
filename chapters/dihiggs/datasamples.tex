\section{Data and Monte Carlo samples}
\label{sec:dihiggs_datasamples}
%-------------------------------------------------------------------------------

\subsection{Data}
The analysis presented here uses the full $pp$ collision dataset
collected in 2015 and the 2016 dataset collected until July 2016
at the center-of-mass energy of 13\,\TeV\, passing data quality
checks requiring good conditions of all sub-detectors.
The data that are currently used correspond to an integrated
luminosity of 13.2~fb$^{-1}$ (3.2~fb$^{-1}$ from 2015 plus
10.0~fb$^{-1}$ from 2016) \footnote{The following GRL is used:  data15\_13TeV.periodAllYear\_DetStatus-v73-pro19-08\_DQDefects-00-01-02\_PHYS\_StandardGRL\_All\_Good\_25ns.xml       or data16\_13TeV.periodAllYear\_DetStatus-v81-pro20-10\_DQDefects-00-02-02\_PHYS\_StandardGRL\_All\_Good\_25ns.xml   ~\cite{GRL}}.

\subsection{MC samples}
MC simulated events are used to estimate SM
backgrounds and the signal acceptances. Table~\ref{tabular:mc_samples} summarizes the MC samples
used for background estimation.
\begin{table}[!htb]
\caption{SM MC samples used for background estimation.}
\label{tabular:mc_samples}
\begin{center}
\begin{tabular}{|l|c|c|}
  \hline
 Process & Generator       & $\sigma\times\text{BR}$ [pb]  \\ 
\hline

$t\bar{t} \to WWbb \to l \nu bb + X$ & \textsc{Powheg+Pythia6} & 451.65 \\
$Wt~incl.$ & \textsc{Powheg+Pythia6} & 71.7 \\
single $t$,  s-channel, $\to l \nu + X$  & \textsc{Powheg+Pythia6} & 3.31 \\ 
single $t$,  t-channel, $\to l \nu + X$  & \textsc{Powheg+Pythia6} & 69.5 \\ 
$W$+jets, $W \to l \nu$ & \textsc{Sherpa} & 20080 \\
$Z$+jets, $Z \to l l$ & \textsc{Sherpa} & 2067 \\
Dibosons~incl. & \textsc{Sherpa} & 47.3 \\
$ggh~incl.$ & \textsc{Powheg+Pythia8} & 43.9 \\
$tth$, $\to l \nu + X$  & \textsc{aMC@NLO + Herwig++} & 0.223 \\
\hline
\end{tabular}
\end{center}
\end{table}

 
The $t\bar{t}$ and single top-quark samples are generated
with \textsc{Powheg-Box} v2~\cite{Frixione:2007vw} using \textsc{CT10} parton distribution functions (PDF)
interfaced to \textsc{Pythia} 6.428~\cite{Sjostrand:2006za} for parton shower,
using the \textsc{Perugia2012}~\cite{Skands:2010ak} tune with
CTEQ6L1~\cite{Pumplin:2002vw} PDF for the underlying event descriptions.
\textsc{EvtGen} v1.2.0~\cite{Lange:2001uf} is used for properties of the bottom
and charm hadron decays. The mass of the top quark is set to $m_{t} =
172.5\,\GeV$. At least one top quark in the $t\bar{t}$ event is required to
decay to a final state with a lepton. The cross section of $t\bar{t}$ is 
known to NNLO in QCD
including re-summation of next-to-next-to-leading logarithmic (NNLL) soft gluon
terms, and the reference value used in ATLAS is calculated using \textsc{Top++}
2.0~\cite{Czakon:2011xx}. The parameter \textsc{Hdamp}, used to regulate the
high-\pt\ radiation in \textsc{Powheg}, is set to $m_{t}$ for good data/MC
agreement in the high \pt\ region~\cite{ATL-PHYS-PUB-2014-005}. Each process of
single top-quark ($t$-channel, $s$-channel and $Wt$-channel) is generated separately. The cross
section of single-top is calculated with the prescriptions in
Ref.~~\cite{Kidonakis:2011wy, Kidonakis:2010ux}. 
\textsc{Sherpa} v2.2~\cite{Gleisberg:2008ta} with the
\textsc{NNPDF}~\cite{Lai:2010vv} PDF set is used as the baseline
generator for the ($W \to \ell\nu$)/($Z\to \ell\ell$)+jets background.
The diboson processes ($WW$,
$WZ$ and $ZZ$) are generated with \textsc{Sherpa} with the \textsc{CT10} PDF
set.  The ggH inclusive sample is generated with \textsc{Powheg} using
the \textsc{CT10} PDF set interfaced to \textsc{Pythia8} for parton
shower, while ttH is a semi-leptonic sample generated with
\textsc{MADGRAPH5\_aMCAtNLO} interfaced to \textsc{Herwig++}. The cross
sections are normalised to the available N3LO cross sections.

Signal samples are
generated with \textsc{MADGRAPH5\_aMCAtNLO}~\cite{Alwall:2014hca} interfaced to
\textsc{Herwig++}. The procedure defined in ~\cite{CP3Paper} is applied. 
It consists of generating events with an effective
lagrangian in the infinite top-quark mass approximation, and  reweighting the
generated events, vertex by vertex,  with form factors that take into
account the finite mass of the top quark.  This procedure partially
accounts for the finite top-quark mass effects avoiding the need to
calculate two-loop amplitudes that are still not fully 
available~\cite{Degrassi_Ramona}.



Table~\ref{tabular:mc_samples_hh} shows the list of $hh$ signals. 
%Two sets of samples are used. The 
%non-resonant signal sample use 
%SM production for the 
They use a heavy Higgs scalar model as the signal hypothesis. The 
masses of the heavy Higgs range from 260 GeV to 3000 GeV while
the Higgs width is set to $~10 MeV$, therefore the model is valid in
the Narrow Width Approximation (NWA). The signals are normalised to the cross section upper limits from the Run1 ATLAS combined result~\cite{Aad:2015xja}. 


\begin{table}[!htb]
\caption{Higgs signal samples used in the analysis. }
\label{tabular:mc_samples_hh}
\begin{center}
\begin{tabular}{|c|l|c|c|c|c|r|}
	\hline
 Process                                    & Generator    \\ \hline
%$hh$ SM & Madgraph5\_aMCatNLO + Herwig++including FF \\
$H \to hh$ ($m_H =260 - 3000$) GeV & Madgraph5\_aMCatNLO +
                                     Herwig++including FF \\
\hline
\end{tabular}
\end{center}
\end{table}


Additional $pp$ collisions generated with \textsc{Pythia} 8.186 are
overlaid to model the effects of the pileup for all simulated
events. All simulated events are processed with the same
reconstruction algorithm used for data. Samples are processed
through the full ATLAS detector simulation~\cite{Aad:2010ah} based 
on \textsc{GEANT4}~\cite{Agostinelli:2002hh}.


\subsection{Event topology}
This analysis is based on the final state 
$hh \to WWb\bar{b} \to l \nu qq b\bar{b}$. This final state was 
chosen as a good compromise between an acceptable branching 
fraction and the reduction of multi-jet background. The full 
hadronic channel suffers from large multi-jet contamination 
while the di-leptonic channel suffers from the low branching
fraction $WW \to l \nu l \nu$. Since  $t \bar{t} \to WWb\bar{b}$ 
is the dominant background, selecting the di-leptonic channel does
not reduce the top background, it merely results in a reduction 
of the signal yield.

The final state is therefore one lepton, 2 light jets and 2 b-jets and
missing momentum from the escaping neutrino.

\clearpage




