\chapter{Conclusion}
%\begin{quote}
%  Well, ya gotta stop somewhere.
%\end{quote}
This thesis presented two analyses, each of which tests the predictions of the Standard Model and constrain the range of possible new theories of physics. The helicities of $W$ bosons produced in semileptonic \ttbar decays were measured with the highest precision to date using the decays from the leptonic $W$, and the world's first direct measurement was produced using the hadronically decaying $W$. Limits on the upper limit of resonant and non-resonant production of Higgs boson pairs was set for the first time using the semileptonic \bbWW final state where one of the $W$'s decays hadronically. At the end of the day, the predictions of the Standard Model were confirmed, and no evidence for the presence of new physics was observed. But that doesn't mean the search is over. There are still places to search and measurements to make. 

Smashing particles together to understand their inner workings has been likened to smashing two watches together and picking through the resulting mess of glass and gears in order to understand how they kept time. While the comparison may be apt, particle colliders are one of the best tools we have for exploring the fundamental nature of our Universe. The LHC will be colliding protons and ions for years to come and at higher intensities and energies than ever before. More energetic collisions open new phase spaces for exploration; new analysis methods can improve and further the sensitivity and reach of our experiments. The application of techniques to understand and unpack the substructure of highly boosted hadronic objects and the introduction of machine learning algorithms in particle identification and event classification have already had significant impacts in many analyses. These tools and others will continue to mature in the coming years, pushing the frontier of high energy physics forward. In short, the future is bright and full of potential.

