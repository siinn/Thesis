\chapter{Introduction}
%\vspace{-.5in}
\begin{quote}
\noindent It's always best to start at the beginning.\\ \quotebox{\hfill - The Wizard of Oz}
\end{quote}

High energy physics is the study of the smallest and most fundamental building blocks of nature on the scale of attometers (10$^{-15}$ m) and smaller. In order to study these incredibly tiny objects, physicists participate in scientific collaborations made up of thousands of scientists and engineers to collide particles like protons and atomic ions at speeds of over 99.9999\% the speed of light. The enormous amounts of time, energy, and money dedicated to facilities like the Large Hadron Collider (LHC) are a testament to the significance of the questions high energy physics aims to answer, namely what are we made of, how did we get here, and why does it all work? There's a certain poetry in the fact that studying interactions on such small scales requires colliders tens of kilometers long, experiments the size of four story buildings, and the combined work of people scattered across the globe.

The Standard Model (SM) of particle physics is a model that describes nearly all we know about the observed fundamental particles and their electromagnetic, weak nuclear, and strong nuclear interactions. Missing from this picture is gravity, and the correct description of gravity at a quantum level is an open issue in theoretical physics. Chapter~\ref{sec:standardModel} provides a description of the various forces described by the Standard Model as well as more in-depth discussions of the Higgs mechanism and the top quark.

Experiments like ATLAS were built to test the predictions of the SM and search for new phenomena. Weighing more than the Eiffel Tower and measuring thirty by fifty meters, ATLAS is the largest detector at the LHC. Chapter~\ref{sec:atlasAndLHC} discusses the various sub-components of the detector and kind of tells you how they work. From measurements gathered by the hundreds of millions of channels in the detector, the various particles produced in collisions are identified and used to work backwards to understand the fundamental interactions taking place at the heart of the detector. Chapter~\ref{sec:ObjectDefinitions} discusses how the particles used to `do physics' are identified and reconstructed.

The analyses presented in this thesis are examples of the kinds of questions we can ask and the types of answers we can provide using the information produced at the LHC. Chapter~\ref{sec:whelicity} deals with precisely measuring the angular distributions and helicity fractions of particles produced in the decays of top quarks, enabling a better understanding of the mathematical structure governing such decays. Chapter~\ref{sec:dihiggs} presents a search for the production of exotic new particles decaying into pairs of Higgs bosons, providing a way to discover or exclude new theories governing the fundamental interactions of particles. 

\section{Original Work}
The enormity and scale of the experiments at the LHC precludes any single physicist or engineer from being able to claim individual responsibility for the complete whole of any physics result or analysis; everyone stands on the shoulders of giants. As such, much of the work presented in this thesis takes advantage of the collaboration of thousands of physicists over the course of several years. Several elements are however the unique and original contributions to the field by the author, namely:
\begin{itemize}
\item the kinematic fitting used to reconstruct the \ttbar system in the W-helicity measurement especially with respect to the studies and results of the sensitivity of the kinematic fit to reconstruct and correctly identify the flavor of the daughters of the hadronic $W$ decay. The helicity measurements produced from the hadronic $W$ decay are the first direct helicity measurements using hadronic $W$ decays in \ttbar events,
\item the template fitting used in the W-helicity measurement to extract the helicity fractions from the final distributions, 
\item the development, validation, and use of an ABCD method to estimate the multi-jet background in the dihiggs search,
\item the statistical fit used in the dihiggs measurement to produce upper limits on non-resonant and exotic resonant dihiggs production.
\end{itemize}