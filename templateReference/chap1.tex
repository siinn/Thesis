\chapter{My First Chapter}
\begin{quote}
\lipsum[1]
\end{quote}

\lipsum[1-3]


\section{Glossaries}
The latex glossaries package can be useful for keeping track of your acronyms, and making a nice hyperlinked 
list at the beginning of your document.  Note: a glossary/list of acronyms is NOT required by the GS.
To use the glossaries package, you must load it in your preamble:
\begin{verbatim}
\usepackage[toc,acronym, section=chapter]{glossaries} %load package
\makeglossaries %required to actually make a glossary
%A list of common acronyms
%Only those used will be displayed, so you can just add to this list
\newacronym{AFM}{AFM}{atomic force microscopy}
\newacronym{BCS}{BCS}{Bardeen-Cooper-Schrieffer}
\newacronym{EPR}{EPR}{electron paramagnetic resonance}
\newacronym{NMR}{NMR}{nuclear magnetic resonance}
\newacronym{QCD}{QCD}{quantum chromodynamics}
\newacronym{QED}{QED}{quantum electrodynamics}
\newacronym{WKB}{WKB}{Wentzel-Kramers-Brillouin}
\newacronym{FRET}{FRET}{fluorescence resonance energy transfer}
\newacronym{ssDNA}{ssDNA}{single-stranded DNA}
\newacronym{dsDNA}{dsDNA}{double-stranded DNA}
\newacronym{ssNA}{ssNA}{single-stranded nucleic acid}
 %load list of acronyms contained in acronyms.tex
\end{verbatim}
Then, to define an acronym, you will need to include it in your preamble:
\begin{verbatim}
\newacronym{AFM}{AFM}{atomic force microscopy}
\end{verbatim}
In this template, the acronyms are all placed in a separate file called
\verb#acronmys.tex# which is then loaded with \verb#%A list of common acronyms
%Only those used will be displayed, so you can just add to this list
\newacronym{AFM}{AFM}{atomic force microscopy}
\newacronym{BCS}{BCS}{Bardeen-Cooper-Schrieffer}
\newacronym{EPR}{EPR}{electron paramagnetic resonance}
\newacronym{NMR}{NMR}{nuclear magnetic resonance}
\newacronym{QCD}{QCD}{quantum chromodynamics}
\newacronym{QED}{QED}{quantum electrodynamics}
\newacronym{WKB}{WKB}{Wentzel-Kramers-Brillouin}
\newacronym{FRET}{FRET}{fluorescence resonance energy transfer}
\newacronym{ssDNA}{ssDNA}{single-stranded DNA}
\newacronym{dsDNA}{dsDNA}{double-stranded DNA}
\newacronym{ssNA}{ssNA}{single-stranded nucleic acid}
#, to
make the main source file easier to read.  You can also just put all
your acronyms directly in the header of your main source file.

Once you have defined an acronym, you can use it with the \verb#\gls{<label>}# command.
The first time you use the acronym, the full definition is printed: \gls{AFM}.
On subsequent uses, just the abbreviation is printed: \gls{AFM} - Latex keeps track
for you, so you don't have to do this manually.  There are fancier forms
of \verb#\gls# that allow you to capitalize (\verb#\Gls{<label>}#) or use the 
plural (\verb#\glspl{<label>}#) forms of your abbreviations.  See 
for example \url{http://en.wikibooks.org/wiki/LaTeX/Glossary} for details
on the glossaries package.

If you want to list all your acronyms at the beginning of your document, you
will need to include the \verb#\makeglossaries# command in your preable, as
shown above, and then 
\begin{verbatim}
\printglossary[type=\acronymtype]
\end{verbatim}
where you want the list of abbreviations to actually appear.  Then you have
to go to the terminal (in the directory containing your document), and run the command
\begin{verbatim}
makeglossaries (yourdocumentname)
\end{verbatim}
to actually create the glossary.  If you just want Latex to keep track of your acronyms for you 
and print no list, you can just skip this step and the \verb#printglossary# command.

One final comment, this template uses \verb#makeindex# to create the glossaries
for compatability (some versions of Ubuntu for example don't support \verb#xindy#).
However, there is a better program for making glossaries that you can use by calling
\begin{verbatim}
\usepackage[xindy,toc,acronym, section=chapter]{glossaries}
\end{verbatim}
instead.  In particular, \verb#xindy# allows you to use symbols in your abbreviations,
such as Greek letters or accented characters, which are not supported by \verb#makeindex#.


\section{Example Figure}
 \begin{figure}
 \centering

 %The actual figure
 \includegraphics[width=0.45\textwidth]{sine}

 \caption[My first figure]%the text in [] will be displayed in the list of figures.  
  %If this is omitted, the full figure caption (follows in {}) will be displayed in the list of figures.
  {\label{fig:sine} My first figure, showing a plot of the sine function.}
 \vspace{0.5 in}
 \end{figure}
Fig.\ref{fig:sine} is produced by the following
\begin{verbatim}
 \begin{figure}
 \includegraphics[width=0.45\textwidth]{sine}
 \caption[My first figure] 
  {\label{fig:sine} My first figure, showing a plot of the sine function.}
 \end{figure}
\end{verbatim}
The text \verb#[My first figure]# will be displayed in the list of figures.
If this is omitted, then the full figure caption (follows in \verb#{}#) 
will be displayed.  If you compile using \verb#latex#, then \verb#sine.eps# will be included.
If you compile using \verb#pdflatex#, then \verb#sine.pdf# will be used.  \verb#pdflatex#
also supports including \verb#.jpg# files if you prefer, but there can be artifacts when 
\verb#.jpg# files are rescaled.  Note that you do \textbf{not} need to include a file extension 
for the figure - latex adds the appropriate extension automatically.  If you do
not want to place your figure in the same directory as the latex root document 
(in this case \verb#template.txt#), then you could instead use something like:
\begin{verbatim}
 \includegraphics[width=0.45\textwidth]{directory/figure}
\end{verbatim}

\section{This is a Section!}

\lipsum[2-4]
\subsection{This is a Subsection?}

\lipsum[5-10]

\subsubsection{This is a Subsubsection.}

\lipsum

\section{Another Section}

\lipsum[2-4]

\begin{table}
\begin{center}
\begin{tabular}{lll}
\hline
\\[3pt]
Alphabet Character & Vowel & Number \\[3pt]
\hline
\\[3pt]
A & Yes & 1 \\
B & No & 2 \\
C & No & 3 \\[3pt]
\hline
\end{tabular}
\label{tbl1}
\caption{My first Table}
\end{center}
\end{table}

\lipsum[4-6]
\begin{table}
\begin{center}
\begin{tabular}{lll}
\hline
\\[3pt]
Alphabet Character & Vowel & Number \\[3pt]
\hline
\\[3pt]
A & Yes & 1 \\
B & No & 2 \\
C & No & 3 \\[3pt]
\hline
\end{tabular}
\label{tbl2}
\caption{My second Table}
\end{center}
\end{table}

\lipsum[1-12]
\begin{equation}
\partial_\mu F^{\mu\nu} = J^\nu
\end{equation}
Lorem ipsum your face. Lorem ipsum your face. 

\lipsum[1]A footnote\footnote{Another foot.}.\lipsum[2-5]
\begin{equation}
E = \gamma mc^2
\end{equation}
Lorem ipsum your face. Lorem ipsum your face. 

\lipsum[2-5]
