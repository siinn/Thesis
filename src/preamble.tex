% SB: These packages were chosen solely by me based on my requirements and preferences. For most packages, excepting
% babel, this should not affect university requirements.

% Encoding
\usepackage[utf8x]{inputenc} % enables the use of UTF-8 as character encoding
\usepackage{microtype} % enables certain features 'towards typographical perfection'

\usepackage[T1]{fontenc} % ensures the use of font encodings that support accented characters

% Fonts
\usepackage{calrsfs}
\DeclareMathAlphabet{\pazocal}{OMS}{zplm}{m}{n}

% Graphics
\usepackage[pdftex]{graphicx}
\usepackage[usenames,dvipsnames]{xcolor}

% Tables
\usepackage{booktabs}
\usepackage{array}
\usepackage{rotating}

% Algorithms
\usepackage{listings}
\usepackage{algorithm}
\usepackage{algpseudocode}

% Others
\usepackage{amsthm}
\usepackage{amsmath}
\usepackage{amssymb}
\usepackage{mathptmx}
\usepackage{etoolbox}
%\usepackage{url}
\usepackage{multirow}
\usepackage{multicol}
% \usepackage{textcomp}
\usepackage{footnote}
% SB: According to the hyperref documentation, the algorithm package should be loaded after hyperref
\usepackage[pdftex,breaklinks,bookmarksnumbered,linktocpage=false,hidelinks=true]{hyperref}
\usepackage{cite}
\usepackage{bm}
\usepackage{xparse}
\usepackage[section]{placeins}
\usepackage{lipsum}

%=========================================
% custom from siinn
%=========================================
\usepackage[super,negative]{nth}
\usepackage{eucal}

% for SI units
\usepackage[T1]{fontenc}
\usepackage{siunitx}
\usepackage{microtype,textcomp}
\setlength{\parskip}{0.8em}
\usepackage[caption=false]{subfig}

\usepackage[skip=2ex]{caption}
\setlength{\abovecaptionskip}{10pt}
\setlength{\belowcaptionskip}{-15pt}

% set LOT, LOF spacing
%\usepackage[titles]{tocloft}
%\renewcommand\cftfigafterpnum{\vskip5pt\par}
%\renewcommand\cfttabafterpnum{\vskip5pt\par}

%=========================================



\graphicspath{ {figs/} }

\newcolumntype{H}{>{\setbox0=\hbox\bgroup}c<{\egroup}@{}}

\setcounter{secnumdepth}{4}

%%% Local Variables:
%%% mode: latex
%%% TeX-master: "dissertation"
%%% End:
