% SB: These packages were chosen solely by me based on my requirements and preferences. For most packages, excepting
% babel, this should not affect university requirements.

% Encoding
\usepackage[utf8x]{inputenc} % enables the use of UTF-8 as character encoding
\usepackage{microtype} % enables certain features 'towards typographical perfection'
\usepackage[T1]{fontenc} % ensures the use of font encodings that support accented characters




%%%%%%%%%%%%%%%%%%%%%%%%% Packages %%%%%%%%%%%%%%%%%%%%%%%%%
% Load your favorite packages here
\usepackage{lipsum} % for fake latin text---you probably don't want this
\usepackage{bm} % for bold math---useful
\usepackage{booktabs} % for more professional tables
\usepackage{hyperref}
\usepackage{bookmark} % helps booksmarks look better in PDF
\hypersetup{colorlinks=false,linkcolor=blue, breaklinks} %internal links in blue, citations in green
\usepackage[all]{hypcap}

%Use of natbib is STRONGLY recommended to sort and compress your references within each citation
%With these options, natbib will convert i.e. [5,3,9,4] to [3-5, 9]
%\usepackage[sort&compress]{natbib}
%\usepackage[nottoc,numbib]{tocbibind}
\usepackage{cite}

%load package for bra-ket notation
\usepackage{braket}

% package for graphics
\usepackage{graphicx}

% package for big tables
\usepackage{adjustbox}

%include package for appendix
\usepackage[title]{appendix}
\usepackage{cleveref}
\usepackage{makecell}

% package for landscape style table
\usepackage{rotating}
\usepackage{longtable}
\usepackage{tabularx}
\usepackage{ltablex}

%load glossaries packages
%\usepackage[toc,acronym, section=chapter]{glossaries}

%load xspace
\usepackage{xspace}

%load verbatim for multiline comments
\usepackage{verbatim}

%%%%%%%%%% packages for whelicty %%%%%%%%%%%%
\usepackage{mathrsfs}
\usepackage{upgreek}
\usepackage{amssymb}
\usepackage{multirow}
%%%%%%%%%%%%%%%%%%%%%%%%%%%%%%%%%%%%%%%%%%%%%

%=========================================
% custom from siinn
%=========================================
%%%% SB: This is the propsal-wide macros file. Define shorthands to your own commands.
%
%
%% SB: I reckon these are handy defaults.
%
%\graphicspath{ {figures/} }
%
%\definecolor{mygreen}{rgb}{0,0.6,0}
%\definecolor{mygray}{rgb}{0.5,0.5,0.5}
%\definecolor{mymauve}{rgb}{0.58,0,0.82}
%
%\newcommand\later[1]{\begin{quote}\textcolor{darkgreen}{\textbackslash \textbf{later\{}} #1 \textcolor{darkgreen}{\}}\end{quote}}
%% \renewcommand{\later}[1]{}
%
%\newcommand\notes[1]{\begin{quote}\textcolor{darkgreen}{\textbackslash \textbf{notes\{}} #1 \textcolor{darkgreen}{\}}\end{quote}}
%% \renewcommand{\notes}[1]{}

\newcommand{\dzero}{$d_{0}$}
\newcommand{\zzero}{$z_{0}$}
\newcommand{\mumu}{$\mu\mu$}
\newcommand{\emu}{$e\mu$}
\newcommand{\ee}{$ee$}


\usepackage[super,negative]{nth}
\usepackage{eucal}

% for SI units
\usepackage[T1]{fontenc}
\usepackage{siunitx}
\usepackage{microtype,textcomp}
\setlength{\parskip}{0.8em}
\usepackage[caption=false]{subfig}
\usepackage{amsmath}
\usepackage{rotating}
\usepackage{pdflscape}
\usepackage{geometry}
\usepackage{xifthen}



% for rotated header
\newcommand*\rot{\rotatebox{60}}


% ATLAS
\newcommand*{\ATLASLATEXPATH}{latex/}
\usepackage[bsm,jetetmiss]{\ATLASLATEXPATH atlasphysics}


%\usepackage[skip=2ex]{caption}
%\setlength{\abovecaptionskip}{10pt}
%\setlength{\belowcaptionskip}{-15pt}

% set LOT, LOF spacing
%\usepackage[titles]{tocloft}
%\renewcommand\cftfigafterpnum{\vskip5pt\par}
%\renewcommand\cfttabafterpnum{\vskip5pt\par}

%=========================================

%\usepackage[xindy,toc,acronym, section=chapter]{glossaries} - recommended if supported by your OS


%\makeglossaries %required to actually make a glossary

%%%%%%%%%%%%%%%%%%%%%%%%% Custom Commands/Environments %%%%%%%%%%%%%%%%%%%%%%%%%
% Put your favorite custom commands here
\newcommand{\fish}{\alpha} % some of my students call it the "fish" symbol
\newcommand{\tab}{\hspace*{2em}}

% Below is an example of customizing the style of headings in your
% dissertation. See osudiss-2.pdf for more information.
%
% For example, if you simply must have uppercase titles:
%\renewcommand\typesetLevelOne[1]{{\Large\textbf{\MakeUppercase{#1}}\par}}
%\renewcommand\typesetLevelTwo[1]{{\Large\textbf{\MakeUppercase{#1}}}}
% Note the \par for \typesetLevelOne
%
% If you want the title to be bold and |\Large| instead of |\Huge|:
%\renewcommand\titleFont{\normalfont\Large\bfseries}

% Add words that TeX may not know how to hyphenate below. This can
% help prevent overfull hboxes. For example,
\hyphenation{eigen-state space-time} 








%=========================================
% from old template
%=========================================

%% Fonts
%\usepackage{calrsfs}
%\DeclareMathAlphabet{\pazocal}{OMS}{zplm}{m}{n}
%
%% Graphics
%\usepackage[pdftex]{graphicx}
%\usepackage[usenames,dvipsnames]{xcolor}
%
%% Tables
%\usepackage{booktabs}
%\usepackage{array}
%\usepackage{rotating}
%
%% Algorithms
%\usepackage{listings}
%\usepackage{algorithm}
%\usepackage{algpseudocode}
%
%% Others
%\usepackage{amsthm}
%\usepackage{amsmath}
%\usepackage{amssymb}
%\usepackage{mathptmx}
%\usepackage{etoolbox}
%%\usepackage{url}
%\usepackage{multirow}
%\usepackage{multicol}
%% \usepackage{textcomp}
%\usepackage{footnote}
%% SB: According to the hyperref documentation, the algorithm package should be loaded after hyperref
%\usepackage[pdftex,breaklinks,bookmarksnumbered,linktocpage=false,hidelinks=true]{hyperref}
%\usepackage{cite}
%\usepackage{bm}
%\usepackage{xparse}
%\usepackage[section]{placeins}
%
%
%\graphicspath{ {figures/} }
%
%\newcolumntype{H}{>{\setbox0=\hbox\bgroup}c<{\egroup}@{}}
%
%\setcounter{secnumdepth}{4}
%
%%%% Local Variables:
%%%% mode: latex
%%%% TeX-master: "dissertation"
%%%% End:
