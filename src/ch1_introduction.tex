\chapter{Introduction and Theory}
\label{chap:introduction}

In this chapter, the theoretical background and motivations for the search for new physics with long-lived particles are presented. In Section~\ref{sec:intro:standard_model}, an overview of the Standard Model of particle physics is presented. In Section~\ref{sec:intro:bsm} the theories beyond the Standard Model that predict new long-lived gauge bosons and the potential discovery mode of the new particles are discussed.


%the motivation for the theories beyond the Standard Model is discussed.
% In Section ~\ref{}, theoretical background for new physics is discussed 

\section{The Standard Model}
\label{sec:intro:standard_model}

The Standard Model (SM)~\cite{Burgess:1003111} of particle physics has been a very successful theory in modern physics that describes the known fundamental particles and their interactions. The SM is a gauge theory based on $SU(3) \otimes SU(2) \otimes U(1)$ symmetry group. The symmetry group describes three fundamental interactions, quantum chromodynamics (QCD), quantum electrodynamics (QED), and weak interactions, which arises from the requirement of local gauge invariance. The much weaker force of gravity is not incorporated in the SM. Common particles, e.g. protons and neutrons, are made up of fermions, and the interactions between the fermions are mediated by spin 1 gauge bosons. Fermions acquire mass by interacting with the Higgs field $H$ via spontaneous symmetry breaking~\cite{Higgs:429539}.

\subsection{Fundamental Particles and Interactions}
\label{sec:intro:fundamental_particles}
The elementary particles in the SM can be divided into two groups, fermions and bosons, and all elementary particles have associated anti-particles with the same property but with opposite charge.

Fermions are spin $\frac{1}{2}$ particles that constitute the building blocks of matter, and they can be divided into two groups, leptons and quarks. Leptons, e.g. electrons and muons, are colorless particles that do not interact through the strong force. Quarks, which make up protons and neutrons, are subject to the strong force due to color charges. There are three generations of leptons and quarks in increasing mass, and each generation consists of two leptons (electric charge 1 or 0) and two quarks (electric charge $\frac{2}{3}$ or $-\frac{1}{3}$). Quarks and charged leptons interact through the electroweak interaction while neutrinos only experience weak interaction. Fermions are described as quantum fields with \textit{left-handed} or \textit{right-handed} chirality, and only \textit{left-handed} fermions and \textit{right-handed} antifermions are subject to the charged-current weak interaction via $W$ boson. Quarks are not observed as free particles due to \textit{color confinement}~\cite{Hata:135318}, but they are only observed in color-neutral bound states, called \textit{hadrons}. There are two types of hadrons: \textit{mesons} and \textit{baryons}. Mesons are composite particles with quark and anti-quark pair, and baryons are composed of three quarks. The elementary fermions are summarized in Table~\ref{table:elementary_fermions}.


\begin{table}[!htb]
  \centering
  \begin{tabular}{ c c c c c c c c}
    \hline
    \hline
    							& \multicolumn{3}{c}{Generation}& \multirow{2}{*}{Q (e)} & \multicolumn{3}{c}{\multirow{2}{*}{Mass (MeV)}} \\
    							& \nth{1} & \nth{2} & \nth{3}	& 	                 & \multicolumn{3}{c}{}	 \\
    \hline
	\multirow{2}{*}{Leptons} 	& $\nu_{e}$ & $\nu_{\mu}$ & $\nu_{\tau}$ & 0    & 0 & 0 & 0 \\
						    	& $e$ 		& $\mu$ 	  & $\tau$ 		 & -1   & 0.511   & 105.7     & 1777     \\
	\hline
	\multirow{2}{*}{Quarks} 	& $u$ 		& $c$ 	      & $t$ 		 & +2/3 & 2.3     & 1275      & 173070   \\
						    	& $d$ 		& $s$ 	      & $b$ 		 & -1/3 & 4.8     & 95        & 4180     \\
    \hline
    \hline
  \end{tabular}
  \caption{The fundamental fermions and their electric charge $Q$ and masses.}
  \label{table:elementary_fermions}
\end{table}

The fundamental interactions are described by gauge bosons, spin 1 particles that are generated by the symmetry groups in the SM. Gluon fields, $g^{a}_{\mu}$, are generated by $SU(3)$ group where $a$ = 1,...,8 and $\mu$ is a Lorentz index (= 0,1,2,3). The quanta of gluon fields produce massless gluons that mediate strong forces. The group $SU(2) \otimes U(1)$ generates gauge fields $W^{a}_{\mu}$ ($a$ = 1,2,3) and $B_{\mu}$ which mediate electroweak force. The physical observable gauge bosons $W^{\pm}$, $Z$, and photon are created by the mixing of these gauge fields, 

\begin{equation}
\label{eq:electroweak_mixing}
\begin{split}
	W^{\pm}_{\mu} = (W^{1}_{\mu} \mp iW^{2}_{\mu}) / \sqrt{2} \\
	Z_{\mu} = \cos \theta_{W} W^{3}_{\mu} - \sin\theta_{W} B_{\mu} \\
	A_{\mu} = \sin \theta_{W} W^{3}_{\mu} + \cos\theta_{W} B_{\mu}
\end{split}
\end{equation}
%
where $\theta_{W}$ is the weak mixing angle.

The photon and gluons are massless, and $W^{\pm}$ and $Z$ bosons gain masses through the Higgs mechanism via spontaneous symmetry breaking. In the Higgs mechanism, an additional complex scalar field, called Higgs field, is introduced with $SU(2)$ symmetry, and because the Higgs potential has non trivial vacuum expectation value, the symmetry of the ground state is spontaneously broken, leading to a massive scalar particle with spin 0, known as Higgs boson. The gauge bosons and their associated fields, and masses are summarized in Table~\ref{table:gauge_bosons}

\begin{table}[!htb]
  \centering
  \begin{tabular}{c c c c c c}
    \hline
    \hline
	Symmetry	& Gauge boson	& Gauge field	& Q (e) & Mass (GeV) \\
	\hline
	\multirow{3}{*}{$SU(2) \otimes U(1)$} & $\gamma$  &	$A_{\mu}$       & 0	   	& 0 	\\
										  & $Z$		  &	$Z_{\mu}$       & 0	   	& 91.2 	\\
										  & $W^{\pm}$ &	$W_{\mu}^{\pm}$ & $\pm$1  & 80.4 	\\
	\hline
	$SU(3)$								  & $g$		  &	$g^{a}_{\mu}$   &  0		& 0		\\

    \hline
    \hline
  \end{tabular}
  \caption{Gauge bosons and their associated fields and masses.}
  \label{table:gauge_bosons}
\end{table}


\section{Beyond the Standard Model}
\label{sec:intro:bsm}

Although the SM has been a very successful theory at explaining fundamental particles and their interactions, there are several experimental observations and phenomena in nature that are not fully explained by the SM. These phenomena include gravity~\cite{PhysRevD.69.105009}, hierarchy problem~\cite{Magg:875284,Magg:133759}, dark matter~\cite{Alpigiani:2281629,bertone2005particle,clowe2006direct}, neutrino oscillations~\cite{ahn2003indications}, and matter-antimatter asymmetry~\cite{toussaint1979matter,dine2003origin}. 

Many Beyond the Standard Model (BSM) theories predict the existence of new particles to explain these unexplained phenomena. In particular, theories such as Hidden Valley~\cite{strassler2007echoes,cassel2010electroweak}, R-parity violation~\cite{senjanovic1975exact,mohapatra1981neutrino} including Minimal Supersymmetric SM~\cite{Barbier:2004ez}, and $Z'$ models with neutrinos~\cite{Basso:2008iv} predict the existence of weakly-coupled, neutral gauge boson at the weak scale. The new gauge boson is called $Z'$ due to the similarity to the standard $Z$ boson. 

\subsection{\texorpdfstring{$Z'$}{Z'} from the extension of the Standard Model}
\label{sec:intro:zprime_extension}

The new weakly-coupled gauge boson can be added to the SM by including an additional $U(1)'$ symmetry to the existing $SU(3) \otimes SU(2) \otimes U(1)$ symmetry. The spontaneous breaking of the $U(1)'$ symmetry, similar to the electroweak symmetry breaking, produces the new gauge boson, $Z'$~\cite{Langacker:2008yv}. The mechanism through which the new symmetry is added to the SM varies by theories. Nonetheless, the $Z'$ boson has two sets of parameters defining its property: the couplings to the SM particles and the energy scale at which the $U(1)'$ symmetry is broken. The former defines the lifetime, $c\tau$, of the particle while the latter defines the mass of the particle.

In one case, $Z'$ can have the same couplings to fermions as the SM $Z$ boson, and the particle is called \textit{sequential} $Z'$~\cite{BARGER1980377}. There have been several searches for the sequential $Z'$~\cite{PhysRevD.86.095010,ABAZOV201188,PhysRevD.90.052005}, and although the sequential $Z'$ provides a useful reference for some theories, it will not be considered in this thesis as the main focus of the analysis is the long-lived particles.

In other case, the $Z'$ can have very small couplings to the SM particles such that the particle have a finite lifetime compatible with the detector volume at the ATLAS experiment. This metastable particle is called the \textit{long-lived} $Z'$. Because of its small coupling to the SM, a direct production of the long-lived $Z'$ is unlikely to be observable at the LHC. Instead, the long-lived $Z'$ should be produced as a decay product of other particles in order to have enough sensitivity to be observed at the LHC. 

In this thesis, this long-lived $Z'$ will be used as a basis in the search for a long-lived resonance. However, $Z'$ is only used as a convenient model to produce a genetic long-lived particle, and no assumption is made on the particle and its production mechanism from existing theories.


\subsection{Long-lived \texorpdfstring{$Z'$}{Z'} Discovery Mode at the LHC}
\label{sec:intro:zprime_discovery}

The primary discovery mode for the long-lived $Z'$, or a similar long-lived particle, is via a dilepton decay channel $Z' \rightarrow  \ell^{+}\ell^{-}$ where $\ell= e$ or $\mu$. The number of dilepton pairs produced, $N_{\ell^{+}\ell^{-}}$, in this process for the integrated luminosity, $\mathcal{L}_{Int.}$, at the LHC is given by,
%
\begin{equation}
\label{eq:cross_section}
N_{\ell^{+}\ell^{-}} = \mathcal{L}_{Int.} \times \sigma_{Z'} \times B_{\ell^{+}\ell^{-}},
\end{equation}
%
where $\sigma_{Z'}$ is the production cross section of $Z'$, and $B_{\ell^{+}\ell^{-}} = \Gamma_{\ell^{+}\ell^{-}} / \Gamma_{Z'}$ is the branching ratio of $Z'$ into $\ell^{+}\ell^{-}$. Therefore, if $Z'$ is light enough to be produced at the LHC, the sensitivity to detect $Z'$ depends on luminosity, the production cross section, and the branching ratio into a particular channel.

Long-lived particles naturally have small width ($\tau_{0} = \hbar / \Gamma$). The detectable mass range and lifetime of $Z'$ is constrained by the center of mass energy ($\sqrt{s} = $ 13 TeV in Run II) and the detector volume ($\sim O(1m)$). There have been other searches for long-lived dilepton resonance at ATLAS~\cite{Aad:2010949} and CMS~\cite{Chatrchyan:1493239} in Run I, and no excess was observed. In this thesis, the dilepton resonance mass, up to 1 TeV, and the lifetime up to $c\tau= 1000~\si{\mm}$ are considered at $\sqrt{s}=$ 13 TeV in Run II.

Other potential discovery channels exist in searches for $Z'$ such as $Z' \rightarrow \tau^{+}\tau^{-}$ and hadronic decay, $Z' \rightarrow jj$ where $j =$ jet although these decay modes are more experimentally difficult to detect due to the irreducible QCD background~\cite{Aaboud:2017yvp,Sirunyan:2016iap}. The search for a long-lived $Z'$ through a dilepton resonance is particularly interesting due to its clean final signature and low backgrounds from the SM.
%More rare decays are also possible such as $Z' \rightarrow V f_{1} \bar{f}_{2}$ where $V = W, Z$ or $Z'$ decaying to two bosons, but these rare decays will be extremely difficult to detect due to the small branching ratio~\cite{Langacker:2009su}.

