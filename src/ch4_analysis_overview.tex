\chapter{Analysis Overview}
\label{chap:analysis_overview}

This thesis presents a search for a heavy long-lived resonance decaying to a dilepton pair, \mumu, \ee, or \emu within the ATLAS ID. The analysis uses 32.8 $\mathrm{fb^{-1}}$ of $pp$ collision data at $\sqrt{s}=13$ TeV collected in 2016 with the ATLAS detector. The long-lived particle (LLP) is referred as $Z'$ but with no assumption on the $Z'$ production mechanism for a model-independent search.

There have been several searches for the LLPs produced in $pp$ collisions in Run I at $\sqrt{s} =$ 8 TeV, including the search for displaced hadronic jet~\cite{Blackburn:1550730}, displaced heavy flavors~\cite{Harris:1512932}, or multi-track displaced vertex~\cite{Aad:2015rba}, and no significant excess was observed. The signature considered in this analysis is distinct from the previous searches, and it is one of the first efforts in the ATLAS experiment to search for a generic displaced vertex signature decaying to a dilepton pair. This analysis focuses on interpreting the LLPs decaying to displaced dilepton vertices in the context of model-independent, exotic resonance search.

In this search, a special setup (Section~\ref{sec:reco:lrt}) of data reprocessing and reconstruction is used in order to gain sensitivity for the non-conventional signature of LLPs. The special setup allows the reconstruction of tracks with large impact parameters and secondary vertices significantly displaced from primary vertices. 
