\chapter{Signal Selection}
\label{chap:signal_selection}

The primary physics objects of this analysis are muons, electrons, and dilepton displaced vertices (DVs). In this section, the signal selection criteria for these physics objects are described.

\section{Event preselection}
\label{sec:selection:pre}

Events are pre-selected in data processing using a combination of offline triggers and custom filters, called \texttt{DRAW\_RPVLL}. The selected events from the main physics stream are reprocessed for the special reconstruction. The filters are designed to select events containing displaced vertex candidates while maintaining reasonably low filter rates (<30 \si{\hertz}). Single photon ($\gamma$), single electron ($e$), di-photon ($\gamma\gamma$), di-electron (\ee), and the combination of photon and electron ($e\gamma$) filters are used to select events with \ee or \emu candidates of interest. Single $\mu$ filter is used to select events with \mumu or \emu candidates of interest.

The filters require events to pass one of the HLTs listed in Table~\ref{table:triggers}. Most of the HLTs developed in the ATLAS are designed for prompt searches, and there are implicit requirements on particles so that the extrapolation of a particle points back to the IP. Therefore, the triggers with no requirement on ID tracks are used to increase the sensitivity to electrons and muons with large transverse and longitudinal impact parameters, $d_{0}$ and $z_{0}$. Consequently, the muon trigger that only uses the muon spectrometer information is used to trigger displaced muons. The photon triggers are used to trigger displaced electrons so that only the calorimeter information is used.

\begin{table}[!htb]
  \centering
  \begin{tabular}{l l}
    \hline
    \hline
    Description     			& Trigger	        	                \\
    \hline
	Single photon 	            & \texttt{HLT\_g140\_loose}             \\
	Di-photon	                & \texttt{HLT\_2g50\_loose}             \\
	Single muon                 & \texttt{HLT\_mu60\_0eta105\_msonly}   \\
    \hline
    \hline
  \end{tabular}
  \caption{HLTs used to select events in \texttt{DRAW\_RPVLL} filter. Single photon trigger requires one photon with $p_{T} > 140$ GeV. Di-photon trigger requires two photons with $p_{T} > 50$ GeV. Single muon trigger requires one muon with $p_{T} > 60$ within $0 < |\eta| < 1.05$ using MS information only.}
  \label{table:triggers}
\end{table}

In addition to the HLT requirements, each filter requires offline selections on particles such as $p_{T}$, $\eta$, and $d_{0}$. Single photon ($\gamma$) or electron ($e$) filter requires a leading photon or electron, respectively, with $p_{T} > 150$ GeV, $\eta < 2.5$, and $d_{0} > 2.0$ mm. These filters also require a second photon or lepton with $p_{T} > 10$ GeV and $\eta < 2.5$ to keep the filter rate reasonably low. Single muon $\mu$ filter requires a muon with $p_{T} > 60$ GeV, $\eta < 2.5$, and $d_{0} > 1.5$ mm. Di-photon ($\gamma\gamma$), di-electron (\ee), and the combination of photon and electron ($e\gamma$) filters require two photons/leptons with $p_{T} > 50$ GeV, $\eta < 2.5$, and $d_{0} > 2.0$ mm. The offline selection is summarized in Table~\ref{table:rpvll_filter_selection}.

\begin{table}[!htb]
  \centering
  \begin{tabular}{l c c c | c c c}
    \hline
    \hline
    Filter          & \multicolumn{3}{c|}{Leading}  &  \multicolumn{3}{c}{Second} \\
                    & $p_{T}$ (GeV) & $|\eta|$    & $d_{0}$ (mm) & $p_{T}$ (GeV) & $|\eta|$    & $d_{0}$ (mm)  \\
    \hline
    $\gamma$, $e$                   & $>$ 150   & $<$2.5  & $>$2.0  & $>$ 10 & $<$2.5 & -       \\
    $\mu$                           & $>$ 60    & $<$2.5  & $>$1.5  & -      & -      & -       \\
    $\gamma\gamma$, \ee, $e\gamma$ & $>$ 50    & $<$2.5  & $>$2.0  & $>$ 50 & $<$2.5 & $>$ 2.0 \\
    \hline
    \hline
  \end{tabular}
  \caption{\texttt{RPVLL} filter offline selection on photon and leptons. Single photon ($\gamma$) or electron ($e$) filter requires a leading photon or electron, respectively, with $p_{T} > 150$ GeV, $\eta < 2.5$, and $d_{0} > 2.0$ mm and a second photon or lepton with $p_{T} > 10$ GeV, $\eta < 2.5$. Single muon $\mu$ filter requires a muon with $p_{T} > 60$ GeV, $\eta < 2.5$, and $d_{0} > 1.5$ mm. Di-photon ($\gamma\gamma$), di-electron (\ee), and the combination of photon and electron ($e\gamma$) filters require a photon or lepton with $p_{T} > 50$ GeV, $\eta < 2.5$, and $d_{0} > 2.0$ mm.}
  \label{table:rpvll_filter_selection}
\end{table}

The event selected by the \texttt{RPVLL} filters are passed downstream for the special reconstruction process, and the physics objects such as electron, muon, and secondary vertices are reconstructed for anlaysis,




\subsection{Event selection}
\label{sec:event_selection}
%Events are selected by minimum requirements on the quality of events, completeness of luminosity blocks, and HLTs used in this analysis. The event selections are as follows.
At analysis-level, minimum requirements are placed on events based on the quality of events, completeness of the corresponding luminosity blocks, primary vertex, and HLTs used in this analysis. In addition, cosmic veto is applied to reject events with back-to-back muons. The event selection is described below.

\begin{itemize}
    \item \texttt{GoodRunsList} removes events from incomplete luminosity blocks.
    \item Event cleaning removes corrupted/bad events due to problems in TileCal, LAr calorimeter noise bursts, and detector downtimes.
    \item Events are required to pass one of the HLTs listed in Table~\ref{table:triggers}.
    \item At least one primary vertex is required along the beam line ($z<200$ \si{\mm}).
    \item Events are rejected if there is a pair of leptons with $R_{\mathrm{CR}} < 0.01$ where $R_{\mathrm{CR}} = \sqrt{(\Delta \phi - \pi)^{2} + (\Sigma \eta)^{2}}$.
\end{itemize}

The event selection is summarized in Table~\ref{table:signal_selection}.


\subsection{Muon and electron requirements}
\label{sec:muon_electron_selection}

The analysis searches for displaced vertices with two leptonic tracks, \mumu, \ee, and \emu. Prior to applying the vertex level selections, tracks from vertices are required pass muon or electron selections based on track quality, kinematics, and lepton identification criteria. 

Electron requirements are based on the recommendations from the Electron-Gamma (EG) group with a few optimization for electrons with large impact parameters. Electrons are rejected if there is a bad cluster associated with an electron. Basic kinematic cuts are applied to electrons, $|\eta| < 2.47$ and $p_{T}$ > 10 GeV. The EG group provides several electron identification criteria, called working points, based on a likelihood discriminant to suppress background electrons originating from photon conversions and heavy flavour decays. In this analysis, the electron \texttt{LooseLH} working point is used, but the requirements on $d_{0}$ and silicon hits are removed to improve electron detection efficiency at large impact parameters.

Muon requirements are based on the recommendations from Muon Combined Performance group with similar optimizations as electrons. Muon \texttt{Loose} working point is used for the identification criteria, and a fiducial cut, $|\eta| < 2.5$, and kinematic cut, $p_{T}$ > 10 GeV, are applied to muons. The requirements on Pixel hits are removed to improve muon detection efficiency at large impact parameters. Muons are required to have an associated ID track for vertex reconstruction. In case of MC samples, muon momentum resolution and scale correction are applied to the simulated muons for better agreement between data and simulation~\cite{Aad:2016jkr}.

Overlap removal is applied to both muons and electrons to ensure that a ID track is associated with only one muon or electron.%Muons or electrons are removed from vertices if they do not pass the muon or electron requirements.
The muon and electron requirements are summarized in Table~\ref{table:lepton_requirement}.

\begin{table}[!htb]
  \centering
  \begin{tabular}{ l  l }
    \hline
    \hline
    \textbf{Muon}     &       Overlap removal                                                                      \\
                                  &       Muon \texttt{Loose} (no requirement on Pixel hits)                       \\
                                  &       $|\eta| < 2.5$                                                           \\
                                  &       $p_{T} > 10.0$ GeV                                                       \\
                                 % &       Associated Inner detector track                                          \\
                                  &       Combined Muon                                                            \\
    \hline
    \textbf{Electron} &       Overlap removal                                                                      \\
                                  &       Bad cluster removal                                                      \\
                                  &       Electron \texttt{LooseLH} (no requirement related to $d_{0}$, Silicon hits)\\
                                  &       $|\eta| < 2.47$                                                          \\
                                  &       $p_{T} > 10.0$ GeV                                                       \\
    \hline
    \hline
  \end{tabular}
  \caption{Muon and electron requirements applied at analysis level.}
  \label{table:lepton_requirement}
\end{table}







\subsection{Vertex selection}
\label{sec:vertex_selection}
The vertex selection is applied to two-track secondary vertices found in Section~\ref{sec:reco:dv}. Secondary vertices with minimum displacement of 2 mm from the primary vertex are selected. The selected displaced vertices are made of two tracks which can be any combination of muon, electron, and non-lepton tracks. Therefore, vertices are separated into three vertex types, control, validation, and signal regions.

In the control region, vertices are required to have two non-leptonic tracks (\xx). In the validation region, vertices are required to have a muon or an electron and another non-leptonic track ($\mu$x, $e$x). In signal region, vertices are required to have a muon pair, an electron pair, or a muon-electron pair (\mumu, \ee, \emu). The control region and the validation region are used for background (Chapter~\ref{chap:bkg}) and systematic uncertainty (Section~\ref{chap:syst}) estimations. The control, validation, and signal regions are summarized in Table~\ref{table:vertex_type}.

\begin{table}[!h]
  \centering
  \begin{tabular}{ l  c}
    \hline
    \hline
	Region				& Vertex Type										\\
    \hline
	Control     		& \xx   										\\
	Validation       	& \mux, \ex										\\
	Signal       		& \mumu, \ee, \emu							\\
    \hline
    \hline
  \end{tabular}
  \caption{The control, validation, and signal regions defined by the vertex type.}
  \label{table:vertex_type}
\end{table}

In all regions, vertices are required to pass a common set of vertex selections described as follows. Vertices are required to have $\chi^2 / \mathrm{ DOF} <$ 5 to reject poorly reconstructed vertices. A minimum transverse displacement of 2 mm from the primary vertex is required to suppress background from prompt decays. Two tracks from a vertex are required to have opposite charges. Vertices are rejected if they are within the volume of disabled Pixel module~\cite{Backhaus:2110260}. Hadronic interaction of charged particles with detector material is a major source of backgrounds. Therefore, the vertices are rejected if they are within dense detector material~\cite{Aaboud:2016poq}. The material veto is not applied to \mumu type vertex due to low probability of muon interaction with detector material. Vertices are also required to be in the detector volume covered by the material mapping ($r < 300$ \si{mm}, $z < 300$ \si{mm}). The vertices are required to have $m >$ 10 GeV to suppress backgrounds from low mass SM particles such as $J/\Psi$. The vertex mass is calculated by the secondary vertex reconstruction algorithm with the assumption that all tracks have pion mass. Cosmic veto is applied to vertices by requiring $R_{\mathrm{CR}} > 0.01$. The cosmic veto is very effective in rejecting cosmic muons reconstructed as back-to-back muon vertices, and the details are discussed in Section~\ref{sec:bkg:cosmic}.

In addition to the common vertex selection, at least one electron or muon from the vertex is required to be matched with one of the triggers listed on Table~\ref{table:triggers} and the filters listed on Table~\ref{table:rpvll_filter_selection} in the signal region. The vertex selection is summarized in Table~\ref{table:signal_selection}.

\begin{table}[!htb]
  \centering
  \begin{tabular}{ l  l }
    \hline
    \hline
    \textbf{Event}           &       GoodRunsList                                                                \\
                             &       Event cleaning                                                              \\
                             &       Trigger filter                                                              \\
                             &       Cosmic veto                                                                 \\
                             &       Primary vertex ($z<$200~\si{\mm})                                           \\
    \hline
    \textbf{Vertex}          &       Trigger matching (signal region only)                                       \\
                             &       $\chi^2 / \mathrm{ DOF} < 5$                                                \\
                             &       $r > $2 mm                                                                  \\
                             &       Opposite charge                                                             \\
                             &       Disabled module veto                                                        \\
                             &       Material veto (excluding \mumu)                                          \\
                             &       $m > 10$ GeV                                                                \\
                             %&       $R_{\mathrm{CR}} > 0.01$                                                    \\
                             &       $r < 300$ \si{mm}, $z < 300$ \si{mm}                                        \\
                             &       Filter matching (signal region only)                                                            \\
    \hline
    \hline
  \end{tabular}
  \caption{Event and vertex selections applied to select displaced vertices.}
  \label{table:signal_selection}
\end{table}




%\begin{itemize}
%    \item Vertices are required to have $\chi^2 / \mathrm{ DOF} <$ 5 to reject poorly reconstructed vertices.
%    \item A minimum transverse displacement of 2 mm from the primary vertex is required to suppress background from prompt decays ($r >$ 2 mm).
%    \item Two tracks from a vertex are required to have opposite charges.
%    \item Vertices are rejected if they are within the volume of disabled Pixel module ~\cite{Backhaus:2110260}.
%    \item Hadronic interaction of charged particles with detector material is a major source of backgrounds. Therefore, the vertices are rejected if they are within dense detector material ~\cite{Aaboud:2016poq}. The material veto is not applied to \mumu type vertex due to low chance of muon interaction with detector material.
%    \item The vertices are required to have $m >$ 10 GeV to suppress backgrounds from long-lived, low mass SM particles such as $J/\Psi$. The vertex mass is calculated by the secondary vertex reconstruction algorithm with the assumption that all tracks have pion mass.
%    \item Cosmic veto is applied to vertices by requiring $R_{\mathrm{CR}} > 0.04$. The cosmic veto is very effective in rejecting cosmic muons reconstructed as back-to-back muon vertices, and the details are discussed in Section~\ref{sec:cosmic_ray}
%\end{itemize}

