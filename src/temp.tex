%%%%%%%%%%%%%%%%%%%%%%%%%%%%%%%%%%%%%%%%%%%%%%%%%%%%%%%%%%%%%%%%%%%%%%%%%%%%%%%
% template.tex
%
% This is a template file for the osudiss-2 class. See
% osudiss-2.pdf for documentation, and the GS material
% for the requirements.
%
% To compile this file:
% latex template
% bibtex template
% latex template
% latex template
%
% (You can also use pdflatex if you prefer.)
%
%%%%%%%%%%%%%%%%%%%%%%%%%%%%%%%%%%%%%%%%%%%%%%%%%%%%%%%%%%%%%%%%%%%%%%%%%%%%%%%
%\documentclass[11pt, draft, onehalf, phd]{osudiss-2}
\documentclass[11pt, onehalf, phd]{osudiss-2}
% The `11pt' option is unnecessary since it is the default
 
% `onehalf' sets the line spacing to one-and-a-half spacing instead of
% double spacing.
 
% The `phd' option is unnecessary since it is the default
 
% Remove `draft' option for final draft
 
%%%%%%%%%%%%%%%%%%%%%%%%% chapter compilation %%%%%%%%%%%%%%%%%%%%%%%%%
%\includeonly{latex/shortcuts, chapters/wHelicity}
 
%%%%%%%%%%%%%%%%%%%%%%%%% Packages %%%%%%%%%%%%%%%%%%%%%%%%%
% Load your favorite packages here
\usepackage{bm} % for bold math---useful
\usepackage{booktabs} % for more professional tables
 
%hyperref packages and options
\usepackage{hyperref}
\usepackage{bookmark} % helps booksmarks look better in PDF
%hypersetup option 'breaklinks' is reguired for line wrapping in the table of contents during latex compilation, and can be removed if you use pdflatex
\hypersetup{colorlinks=true,linkcolor=blue, breaklinks} %internal links in blue, citations in green
%\hypersetup{colorlinks=true,linkcolor=black, citecolor=black, breaklinks} %all links in black
\usepackage[all]{hypcap}
 
%Use of natbib is STRONGLY recommended to sort and compress your references within each citation
%With these options, natbib will convert i.e. [5,3,9,4] to [3-5, 9]
%\usepackage[sort&compress]{natbib}
%\usepackage[nottoc,numbib]{tocbibind}
\usepackage{cite}
 
%
%load package for bra-ket notation
\usepackage{braket}
 
% package for graphics.... why the fuck isn't this default?
\usepackage{graphicx}
 
% package for big tables
\usepackage{adjustbox}
 
%include package for appendix
\usepackage{appendix}
\usepackage{cleveref}
\usepackage{makecell}
 
% package for landscape style tabl
\usepackage{rotating}
\usepackage{longtable}
\usepackage{tabularx}
\usepackage{ltablex}
 
%load glossaries packages
\usepackage[toc,acronym, section=chapter]{glossaries}
 
%load xspace
\usepackage{xspace}
 
%load verbatim for multiline comments
\usepackage{verbatim}
 
%%%%%%%%%% packages for whelicty %%%%%%%%%%%%
\usepackage{mathrsfs}
\usepackage{upgreek}
\usepackage{amssymb}
\usepackage{multirow}
%%%%%%%%%%%%%%%%%%%%%%%%%%%%%%%%%%%%%%%%%%%%%
%=========================================
% custom from siinn
%=========================================
\usepackage[super,negative]{nth}
\usepackage{eucal}

% for SI units
\usepackage[T1]{fontenc}
\usepackage{siunitx}
\usepackage{microtype,textcomp}
\setlength{\parskip}{0.8em}
\usepackage[caption=false]{subfig}

%\usepackage[skip=2ex]{caption}
%\setlength{\abovecaptionskip}{10pt}
%\setlength{\belowcaptionskip}{-15pt}

%%% SB: This is the propsal-wide macros file. Define shorthands to your own commands.
%
%
%% SB: I reckon these are handy defaults.
%
%\graphicspath{ {figs/} }
%
%\definecolor{mygreen}{rgb}{0,0.6,0}
%\definecolor{mygray}{rgb}{0.5,0.5,0.5}
%\definecolor{mymauve}{rgb}{0.58,0,0.82}
%
%\newcommand\later[1]{\begin{quote}\textcolor{darkgreen}{\textbackslash \textbf{later\{}} #1 \textcolor{darkgreen}{\}}\end{quote}}
%% \renewcommand{\later}[1]{}
%
%\newcommand\notes[1]{\begin{quote}\textcolor{darkgreen}{\textbackslash \textbf{notes\{}} #1 \textcolor{darkgreen}{\}}\end{quote}}
%% \renewcommand{\notes}[1]{}

\newcommand{\dzero}{$d_{0}$}
\newcommand{\zzero}{$z_{0}$}
\newcommand{\mumu}{$\mu\mu$}
\newcommand{\emu}{$e\mu$}
\newcommand{\ee}{$ee$}



%=========================================
 
%\usepackage[xindy,toc,acronym, section=chapter]{glossaries} - recommended if supported by your OS
\makeglossaries %required to actually make a glossary
%%A list of common acronyms
%Only those used will be displayed, so you can just add to this list
%\newacronym{LRT}{LRT}{Large radius tracking}
%\newacronym{DV}{DV}{Displaced Vertex}
 %load list of acronyms contained in acronyms.tex
%%A list of shortcuts I intend to use
% Added by Siinn
\newcommand*{\dzero}{\ensuremath{d_{0}}\xspace}
\newcommand*{\zzero}{\ensuremath{z_{0}}\xspace}

% signed
\newcommand*{\ee}{\ensuremath{e^{+}e^{-}}\xspace}
\newcommand*{\mumu}{\ensuremath{\mu^{+}\mu^{-}}\xspace}
\newcommand*{\emu}{\ensuremath{e^{\pm}\mu^{\mp}}\xspace}

\newcommand*{\mux}{\ensuremath{\mu^{\pm}\mathrm{x}^{\mp}}\xspace}
\newcommand*{\ex}{\ensuremath{e^{\pm}\mathrm{x}^{\mp}}\xspace}
\newcommand*{\xx}{\ensuremath{\mathrm{x}^{+}\mathrm{x}^{-}}\xspace}

% non-signed 
\newcommand*{\eens}{\ensuremath{ee}\xspace}
\newcommand*{\mumuns}{\ensuremath{\mu\mu}\xspace}
\newcommand*{\emuns}{\ensuremath{e\mu}\xspace}

\newcommand*{\muxns}{\ensuremath{\mu\mathrm{x}}\xspace}
\newcommand*{\exns}{\ensuremath{e\mathrm{x}}\xspace}
\newcommand*{\xxns}{\ensuremath{\mathrm{xx}}\xspace}

\newcommand*{\Rcr}{\ensuremath{\DeltaR_{\mathrm{cosmic}}}\xspace}
\newcommand*{\Ks}{\ensuremath{K_{S}}\xspace}
\newcommand*{\CLs}{\ensuremath{\mathrm{CL}_{s}}\xspace}


% notation for TF extrapolations
\newcommand*{\Nx}{\ensuremath{N_{\mathrm{x}}}\xspace}
\newcommand*{\Ne}{\ensuremath{N_{e}}\xspace}
\newcommand*{\Nm}{\ensuremath{N_{\mu}}\xspace}

\newcommand*{\Nxx}{\ensuremath{N_{\mathrm{xx}}}\xspace}
\newcommand*{\Nmx}{\ensuremath{N_{\mu\mathrm{x}}}\xspace}
\newcommand*{\Nex}{\ensuremath{N_{e\mathrm{x}}}\xspace}
\newcommand*{\Nmm}{\ensuremath{N_{\mu\mu}}\xspace}
\newcommand*{\Nee}{\ensuremath{N_{ee}}\xspace}
\newcommand*{\Nem}{\ensuremath{N_{e\mu}}\xspace}


\newcommand*{\NVxx}[1]{%
    \ifthenelse{\isempty{#1}}%
    {\ensuremath{    N_{\mathrm{xx}}^{\mathrm{v}}             }\xspace}%
    {\ensuremath{    N_{\mathrm{xx}}^{\mathrm{v(#1)}}          }\xspace}%
    }
\newcommand*{\NVmx}[1]{%
    \ifthenelse{\isempty{#1}}%
    {\ensuremath{    N_{\mu\mathrm{x}}^{\mathrm{v}}           }\xspace}%
    {\ensuremath{    N_{\mu\mathrm{x}}^{\mathrm{v(#1)}}        }\xspace}%
    }
\newcommand*{\NVex}[1]{%
    \ifthenelse{\isempty{#1}}%
    {\ensuremath{    N_{e\mathrm{x}}^{\mathrm{v}}           }\xspace}%
    {\ensuremath{    N_{e\mathrm{x}}^{\mathrm{v(#1)}}        }\xspace}%
    }
\newcommand*{\NVmm}[1]{%
    \ifthenelse{\isempty{#1}}%
    {\ensuremath{    N_{\mu\mu}^{\mathrm{v}}           }\xspace}%
    {\ensuremath{    N_{\mu\mu}^{\mathrm{v(#1)}}        }\xspace}%
    }
\newcommand*{\NVee}[1]{%
    \ifthenelse{\isempty{#1}}%
    {\ensuremath{    N_{ee}^{\mathrm{v}}           }\xspace}%
    {\ensuremath{    N_{ee}^{\mathrm{v(#1)}}        }\xspace}%
    }
\newcommand*{\NVem}[1]{%
    \ifthenelse{\isempty{#1}}%
    {\ensuremath{    N_{e\mu}^{\mathrm{v}}           }\xspace}%
    {\ensuremath{    N_{e\mu}^{\mathrm{v(#1)}}        }\xspace}%
    }

% vertex probability
\newcommand*{\Pxx}{\ensuremath{P_{\mathrm{xx}}^{\mathrm{v}}                        }\xspace}
\newcommand*{\Pmx}{\ensuremath{P_{\mu \mathrm{x}}^{\mathrm{v}}                     }\xspace}
\newcommand*{\Pex}{\ensuremath{P_{e \mathrm{x}}^{\mathrm{v}}                       }\xspace}
\newcommand*{\Pee}{\ensuremath{P_{ee}^{\mathrm{v}}                                 }\xspace}
\newcommand*{\Pmm}{\ensuremath{P_{\mu\mu}^{\mathrm{v}}                             }\xspace}
\newcommand*{\Pem}{\ensuremath{P_{e\mu}^{\mathrm{v}}                               }\xspace}

% lepton probability
\newcommand*{\Px}{\ensuremath{P_{\mathrm{x}}           }\xspace}
\newcommand*{\Pm}{\ensuremath{P_{\mu}                  }\xspace}
\newcommand*{\Pe}{\ensuremath{P_{e}                    }\xspace}

% scale factors
\newcommand*{\SxxTOmx}{\ensuremath{S_{\mathrm{xx}\rightarrow \mu \mathrm{x}}    }\xspace}
\newcommand*{\SxxTOex}{\ensuremath{S_{\mathrm{xx}\rightarrow e \mathrm{x}}      }\xspace}
\newcommand*{\SxxTOmm}{\ensuremath{S_{\mathrm{xx}\rightarrow \mu\mu}            }\xspace}
\newcommand*{\SxxTOee}{\ensuremath{S_{\mathrm{xx}\rightarrow ee}                }\xspace}
\newcommand*{\SxxTOem}{\ensuremath{S_{\mathrm{xx}\rightarrow e\mu}              }\xspace}

\newcommand*{\SmxTOmm}{\ensuremath{S_{\mu \mathrm{x}\rightarrow \mu\mu}         }\xspace}
\newcommand*{\SexTOee}{\ensuremath{S_{e \mathrm{x}\rightarrow ee}               }\xspace}
\newcommand*{\SemxTOem}{\ensuremath{S_{e \mathrm{x},~\mu \mathrm{x} \rightarrow e\mu} }\xspace}



% generators
\newcommand{\pythia}{\textsc{Pythia}\xspace}
\newcommand{\sherpa}{\textsc{Sherpa}\xspace}






 %load list of shortcuts shortcuts.tex
 
%%%%%%%%%%%%%%%%%%%%%%%%% Custom Commands/Environments %%%%%%%%%%%%%%%%%%%%%%%%%
% Put your favorite custom commands here
\newcommand{\fish}{\alpha} % some of my students call it the "fish" symbol
\newcommand{\tab}{\hspace*{2em}}
 
% Below is an example of customizing the style of headings in your
% dissertation. See osudiss-2.pdf for more information.
%
% For example, if you simply must have uppercase titles:
%\renewcommand\typesetLevelOne[1]{{\Large\textbf{\MakeUppercase{#1}}\par}}
%\renewcommand\typesetLevelTwo[1]{{\Large\textbf{\MakeUppercase{#1}}}}
% Note the \par for \typesetLevelOne
%
% If you want the title to be bold and |\Large| instead of |\Huge|:
%\renewcommand\titleFont{\normalfont\Large\bfseries}
 
% Add words that TeX may not know how to hyphenate below. This can
% help prevent overfull hboxes. For example,
\hyphenation{eigen-state space-time}
 
%%%%%%%%%%%%%%%%%%%%%%%%% Document Metadata %%%%%%%%%%%%%%%%%%%%%%%%%
\title{Search for long-lived resonance decaying to a dilepton pair in $pp$ collisions at $\sqrt{s}=13$ TeV with the ATLAS detector}
\author{Siinn Che}
\advisorname{Professor K.K. Gan}
\degree{Doctor of Philosophy} % Default value
\member{Professor Richard Kass}
\member{Professor Junko Shigemitsu}
\member{Professor Fengyuan Yang}
%\member{Kenneth Williams}
\authordegrees{B.S., M.S.}
\graduationyear{2018}
\unit{Graduate Program in Physics} % defaults to ``Graduate Program in Physics''
 
%%%%%%%%%%%%%%%%%%%%%%%%% Begin Document %%%%%%%%%%%%%%%%%%%%%%%%%
\begin{document}
 
%\begin{comment}
\frontmatter
 
\begin{abstract}
A search for a long-lived neutral massive particle decaying to a $\mu\mu$, $ee$, or $e\mu$ pair is presented with 32.8 $\mathrm{fb^{-1}}$ of $pp$ collisions at $\sqrt{s}=13$ TeV taken by the ATLAS detector at the LHC.
Upper limits are presented on the production cross section times branching ratio for resonances decaying to a lepton pair. Also presented is the detection efficiency as a function of $p_{T}$ and $\eta$ for a resonance with mass of 0.1$-$2.0 TeV and lifetime ($c\tau$) of 100$-$500 mm to allow for an estimate of the upper limit on the cross section for any model of interest.
\end{abstract}

\dedication{Dedication\\\vspace{5in}
some more text}
\begin{acknowledgements}

My PhD years had been blessed years with so many good colleagues and friends. I traveled all around the world, and I worked on the most advanced collider in the world with countless nights of despair and triumph. If I look back, I will remember these years as the most adventurous, full of hopes and dreams, and unforgettable golden years.

And none of these would be possible without my advisor, Prof. KK. From the first day I met him in his office, he guided me through this journey with his continuous supports, patience, wisdom, and fine sense of humor. He trusted in me even when I was losing faith in myself. He truly showed me what a great advisor and scientist should be.

Also, throughout the program, Prof. Kass provided me invaluable advice and supports, and his insightful comments allowed me to see a new perspective on many problems and helped me to refine this dissertation. I would like to extend my special gratitude to the other members of the PhD Committee, Prof. Shigemitsu and Prof. Yang, for their continuous supports and encouragement. They had been the most supportive committee I could ever imagine, and this dissertation wouldn't be complete without their encouragement.

My great colleague and mentor, Dr. Chris Martin had been the most valuable reinforcement in this uphill battle, and I express my sincere gratitude to him. This analysis would be incomplete without his strong supports, dedication, and extraordinary work.

Lastly, I would like to express my deepest gratitude to my parents and sister for their endless love and supports. I still remember the first day I arrived in Columbus with my family, and it was their supports and love that kept me going during the most difficult times. Most importantly, nothing would be possible without God, and I thank him forever.


\end{acknowledgements}


%\begin{vita}
\vspace{-20pt}
\dateitem{May 2011}{B.S., B.A. University of Kansas, Lawrence, Kansas, USA}
\dateitem{September 2011 -- May 2013}{Graduate Teaching Assistant, Ohio State University, Columbus, Ohio, USA}
\dateitem{August 2013}{M.Sc. Ohio State University, Columbus, Ohio, USA}
\dateitem{September 2013 -- August 2017}{Graduate Research Assistant, Ohio State University, Columbus, Ohio, USA}

%\dateitem{February 31, 1969}{Born---Crazytown, OH}
%\dateitem{Mayuary, 2000}{B.S., Party College, Party Town, Party State}
% Insert other relevant items here (GTA, etc.)

\begin{publist}
\pubitem{``Search for Higgs boson pair production in the \bbWW final state at $\sqrt{s}=13$ TeV the ATLAS detector'', with M. Aaboud et. al. [ATLAS Collaboration], In preparation}
\pubitem{``Measurement of the $W$ Boson Helicity Fractions in \ttbar Events at $\sqrt{s}=8$ TeV in the Lepton + Jets Channel with ATLAS'', with M. Aaboud et. al. [ATLAS Collaboration], Eur. Phys. J. C 77 (2017) 264}
\pubitem{``Measurements of $Z\gamma$ electroweak production in association with a high-mass dijet system at $\sqrt{s}=$ 8 TeV with the ATLAS detector'', with M. Aaboud et. al. [ATLAS Collaboration], Submitted to Journal of High Energy Physics}
\pubitem{``Search for Higgs boson decays to a photon and a $Z$ boson in $pp$ collisions at $\sqrt{s}=$ 7 and 8 TeV with the ATLAS detector'', with M. Aaboud et. al. [ATLAS Collaboration], Phys. Lett. B, 732, 8 (2014)}
%\pubitem{Paper 1}
%\pubitem{Paper 2}
\end{publist}

\begin{fieldsstudy}
\vspace{-10pt}
\majorfield{Physics}
\onestudy{LHC physics, higgs decays, top quark physics, di-higgs searches, $t\bar{t}$ measurements, statistical interpretations of data, hardware and software development} % optional
% Alternatively you can do:
% \begin{studieslist}
% \studyitem{Topic 1}{Professor 1}
% \studyitem{Topic 2}{Professor 2}
% \studyitem{Topic 3}{Professor 3}
% \end{studieslist}
\end{fieldsstudy}

\end{vita}

\tableofcontents
 
% list of figures (comment out if you don't have any figures)
\clearpage
\listoffigures
 
% list of tables (comment out if you don't have any tables)
\clearpage
\listoftables
 
%print glossary (comment out if you don't want a glossary)
%\printglossary[type=\acronymtype,title={List of Abbreviations},toctitle={List of Abbreviations}]
 
%\end{comment}
 
\mainmatter
\chapter{Introduction and Theory}
\label{chap:introduction}

In this chapter, the theoretical background and motivations for the search for new physics with long-lived particles are presented. In Section~\ref{sec:intro:standard_model}, an overview of the Standard Model of particle physics is presented. In Section~\ref{sec:intro:bsm} the theories beyond the Standard Model that predict new long-lived gauge bosons and the potential discovery mode of the new particles are discussed.


%the motivation for the theories beyond the Standard Model is discussed.
% In Section ~\ref{}, theoretical background for new physics is discussed 

\section{The Standard Model}
\label{sec:intro:standard_model}

The Standard Model (SM)~\cite{Burgess:1003111} of particle physics has been a very successful theory in modern physics that describes the known fundamental particles and their interactions. The SM is a gauge theory based on $SU(3) \otimes SU(2) \otimes U(1)$ symmetry group. The symmetry group describes three fundamental interactions, quantum chromodynamics (QCD), quantum electrodynamics (QED), and weak interactions, which arises from the requirement of local gauge invariance. The much weaker force of gravity is not incorporated in the SM. Common particles, e.g. protons and neutrons, are made up of fermions, and the interactions between the fermions are mediated by spin 1 gauge bosons. Fermions acquire mass by interacting with the Higgs field $H$ via spontaneous symmetry breaking~\cite{Higgs:429539}.

\subsection{Fundamental Particles and Interactions}
\label{sec:intro:fundamental_particles}
The elementary particles in the SM can be divided into two groups, fermions and bosons, and all elementary particles have associated anti-particles with the same property but with opposite charge.

Fermions are spin $\frac{1}{2}$ particles that constitute the building blocks of matter, and they can be divided into two groups, leptons and quarks. Leptons, e.g. electrons and muons, are colorless particles that do not interact through the strong force. Quarks, which make up protons and neutrons, are subject to the strong force due to color charges. There are three generations of leptons and quarks in increasing mass, and each generation consists of two leptons (electric charge 1 or 0) and two quarks (electric charge $\frac{2}{3}$ or $-\frac{1}{3}$). Quarks and charged leptons interact through the electroweak interaction while neutrinos only experience weak interaction. Fermions are described as quantum fields with \textit{left-handed} or \textit{right-handed} chirality, and only \textit{left-handed} fermions and \textit{right-handed} antifermions are subject to the charged-current weak interaction via $W$ boson. Quarks are not observed as free particles due to \textit{color confinement}~\cite{Hata:135318}, but they are only observed in color-neutral bound states, called \textit{hadrons}. There are two types of hadrons: \textit{mesons} and \textit{baryons}. Mesons are composite particles with quark and anti-quark pair, and baryons are composed of three quarks. The elementary fermions are summarized in Table~\ref{table:elementary_fermions}.


\begin{table}[!htb]
  \centering
  \begin{tabular}{ c c c c c c c c}
    \hline
    \hline
    							& \multicolumn{3}{c}{Generation}& \multirow{2}{*}{Q (e)} & \multicolumn{3}{c}{\multirow{2}{*}{Mass (MeV)}} \\
    							& \nth{1} & \nth{2} & \nth{3}	& 	                 & \multicolumn{3}{c}{}	 \\
    \hline
	\multirow{2}{*}{Leptons} 	& $\nu_{e}$ & $\nu_{\mu}$ & $\nu_{\tau}$ & 0    & 0 & 0 & 0 \\
						    	& $e$ 		& $\mu$ 	  & $\tau$ 		 & -1   & 0.511   & 105.7     & 1777     \\
	\hline
	\multirow{2}{*}{Quarks} 	& $u$ 		& $c$ 	      & $t$ 		 & +2/3 & 2.3     & 1275      & 173070   \\
						    	& $d$ 		& $s$ 	      & $b$ 		 & -1/3 & 4.8     & 95        & 4180     \\
    \hline
    \hline
  \end{tabular}
  \caption{The fundamental fermions and their electric charge $Q$ and masses.}
  \label{table:elementary_fermions}
\end{table}

The fundamental interactions are described by gauge bosons, spin 1 particles that are generated by the symmetry groups in the SM. Gluon fields, $g^{a}_{\mu}$, are generated by $SU(3)$ group where $a$ = 1,...,8 and $\mu$ is a Lorentz index (= 0,1,2,3). The quanta of gluon fields produce massless gluons that mediate strong forces. The group $SU(2) \otimes U(1)$ generates gauge fields $W^{a}_{\mu}$ ($a$ = 1,2,3) and $B_{\mu}$ which mediate electroweak force. The physical observable gauge bosons $W^{\pm}$, $Z$, and photon are created by the mixing of these gauge fields, 

\begin{equation}
\label{eq:electroweak_mixing}
\begin{split}
	W^{\pm}_{\mu} = (W^{1}_{\mu} \mp iW^{2}_{\mu}) / \sqrt{2} \\
	Z_{\mu} = \cos \theta_{W} W^{3}_{\mu} - \sin\theta_{W} B_{\mu} \\
	A_{\mu} = \sin \theta_{W} W^{3}_{\mu} + \cos\theta_{W} B_{\mu}
\end{split}
\end{equation}
%
where $\theta_{W}$ is the weak mixing angle.

The photon and gluons are massless, and $W^{\pm}$ and $Z$ bosons gain masses through the Higgs mechanism via spontaneous symmetry breaking. In the Higgs mechanism, an additional complex scalar field, called Higgs field, is introduced with $SU(2)$ symmetry, and because the Higgs potential has non trivial vacuum expectation value, the symmetry of the ground state is spontaneously broken, leading to a massive scalar particle with spin 0, known as Higgs boson. The gauge bosons and their associated fields, and masses are summarized in Table~\ref{table:gauge_bosons}

\begin{table}[!htb]
  \centering
  \begin{tabular}{c c c c c c}
    \hline
    \hline
	Symmetry	& Gauge boson	& Gauge field	& Q (e) & Mass (GeV) \\
	\hline
	\multirow{3}{*}{$SU(2) \otimes U(1)$} & $\gamma$  &	$A_{\mu}$       & 0	   	& 0 	\\
										  & $Z$		  &	$Z_{\mu}$       & 0	   	& 91.2 	\\
										  & $W^{\pm}$ &	$W_{\mu}^{\pm}$ & $\pm$1  & 80.4 	\\
	\hline
	$SU(3)$								  & $g$		  &	$g^{a}_{\mu}$   &  0		& 0		\\

    \hline
    \hline
  \end{tabular}
  \caption{Gauge bosons and their associated fields and masses.}
  \label{table:gauge_bosons}
\end{table}


\section{Beyond the Standard Model}
\label{sec:intro:bsm}

Although the SM has been a very successful theory at explaining fundamental particles and their interactions, there are several experimental observations and phenomena in nature that are not fully explained by the SM. These phenomena include gravity~\cite{PhysRevD.69.105009}, hierarchy problem~\cite{Magg:875284,Magg:133759}, dark matter~\cite{Alpigiani:2281629,bertone2005particle,clowe2006direct}, neutrino oscillations~\cite{ahn2003indications}, and matter-antimatter asymmetry~\cite{toussaint1979matter,dine2003origin}. 

Many Beyond the Standard Model (BSM) theories predict the existence of new particles to explain these unexplained phenomena. In particular, theories such as Hidden Valley~\cite{strassler2007echoes,cassel2010electroweak}, R-parity violation~\cite{senjanovic1975exact,mohapatra1981neutrino} including Minimal Supersymmetric SM~\cite{Barbier:2004ez}, and $Z'$ models with neutrinos~\cite{Basso:2008iv} predict the existence of weakly-coupled, neutral gauge boson at the weak scale. The new gauge boson is called $Z'$ due to the similarity to the standard $Z$ boson. 

\subsection{\texorpdfstring{$Z'$}{Z'} from the extension of the Standard Model}
\label{sec:intro:zprime_extension}

The new weakly-coupled gauge boson can be added to the SM by including an additional $U(1)'$ symmetry to the existing $SU(3) \otimes SU(2) \otimes U(1)$ symmetry. The spontaneous breaking of the $U(1)'$ symmetry, similar to the electroweak symmetry breaking, produces the new gauge boson, $Z'$~\cite{Langacker:2008yv}. The mechanism through which the new symmetry is added to the SM varies by theories. Nonetheless, the $Z'$ boson has two sets of parameters defining its property: the couplings to the SM particles and the energy scale at which the $U(1)'$ symmetry is broken. The former defines the lifetime, $c\tau$, of the particle while the latter defines the mass of the particle.

In one case, $Z'$ can have the same couplings to fermions as the SM $Z$ boson, and the particle is called \textit{sequential} $Z'$~\cite{BARGER1980377}. There have been several searches for the sequential $Z'$~\cite{PhysRevD.86.095010,ABAZOV201188,PhysRevD.90.052005}, and although the sequential $Z'$ provides a useful reference for some theories, it will not be considered in this thesis as the main focus of the analysis is the long-lived particles.

In other case, the $Z'$ can have very small couplings to the SM particles such that the particle have a finite lifetime compatible with the detector volume at the ATLAS experiment. This metastable particle is called the \textit{long-lived} $Z'$. Because of its small coupling to the SM, a direct production of the long-lived $Z'$ is unlikely to be observable at the LHC. Instead, the long-lived $Z'$ should be produced as a decay product of other particles in order to have enough sensitivity to be observed at the LHC. 

In this thesis, this long-lived $Z'$ will be used as a basis in the search for a long-lived resonance. However, $Z'$ is only used as a convenient model to produce a genetic long-lived particle, and no assumption is made on the particle and its production mechanism from existing theories.


\subsection{Long-lived \texorpdfstring{$Z'$}{Z'} Discovery Mode at the LHC}
\label{sec:intro:zprime_discovery}

The primary discovery mode for the long-lived $Z'$, or a similar long-lived particle, is via a dilepton decay channel $Z' \rightarrow  \ell^{+}\ell^{-}$ where $\ell= e$ or $\mu$. The number of dilepton pairs produced, $N_{\ell^{+}\ell^{-}}$, in this process for the integrated luminosity, $\mathcal{L}_{Int.}$, at the LHC is given by,
%
\begin{equation}
\label{eq:cross_section}
N_{\ell^{+}\ell^{-}} = \mathcal{L}_{Int.} \times \sigma_{Z'} \times B_{\ell^{+}\ell^{-}},
\end{equation}
%
where $\sigma_{Z'}$ is the production cross section of $Z'$, and $B_{\ell^{+}\ell^{-}} = \Gamma_{\ell^{+}\ell^{-}} / \Gamma_{Z'}$ is the branching ratio of $Z'$ into $\ell^{+}\ell^{-}$. Therefore, if $Z'$ is light enough to be produced at the LHC, the sensitivity to detect $Z'$ depends on luminosity, the production cross section, and the branching ratio into a particular channel.

Long-lived particles naturally have small width ($\tau_{0} = \hbar / \Gamma$). The detectable mass range and lifetime of $Z'$ is constrained by the center of mass energy ($\sqrt{s} = $ 13 TeV in Run II) and the detector volume ($\sim O(1m)$). There have been other searches for long-lived dilepton resonance at ATLAS~\cite{Aad:2010949} and CMS~\cite{Chatrchyan:1493239} in Run I, and no excess was observed. In this thesis, the dilepton resonance mass, up to 1 TeV, and the lifetime up to $c\tau= 1000~\si{\mm}$ are considered at $\sqrt{s}=$ 13 TeV in Run II.

Other potential discovery channels exist in searches for $Z'$ such as $Z' \rightarrow \tau^{+}\tau^{-}$ and hadronic decay, $Z' \rightarrow jj$ where $j =$ jet although these decay modes are more experimentally difficult to detect due to the irreducible QCD background~\cite{Aaboud:2017yvp,Sirunyan:2016iap}. The search for a long-lived $Z'$ through a dilepton resonance is particularly interesting due to its clean final signature and low backgrounds from the SM.
%More rare decays are also possible such as $Z' \rightarrow V f_{1} \bar{f}_{2}$ where $V = W, Z$ or $Z'$ decaying to two bosons, but these rare decays will be extremely difficult to detect due to the small branching ratio~\cite{Langacker:2009su}.


\chapter{The ATLAS Experiment at the LHC}
\label{chap:atlas_experiment}
The ATLAS experiment is one of the four major experiments at the LHC at The European Organization for Nuclear Research (CERN). The ATLAS experiment is designed as a general-purpose particle experiment, together with the CMS experiment. In Section~\ref{sec:atlas:lhc}, a brief description of the LHC is given, and in Section~\ref{sec:atlas:detector}, the ATLAS detector and its sub-detector systems are described.

\section{The Large Hadron Collider}
\label{sec:atlas:lhc}

The LHC is the world's largest synchrotron accelerator and $pp$ collider located underneath the CERN campus near Geneva, Switzerland. The LHC's circular beam pipes are 27 km in circumference, and two beams of protons are accelerated in opposite direction producing $pp$ collisions at $\sqrt{s} = $ 13 TeV in Run II. Separate magnet systems are used to direct proton beams in each direction.

\begin{figure}[!htb]
    \includegraphics[width=0.8\textwidth]{figs/lhc.png}
    \centering
    \caption{An illustration of the LHC accelerators and the four main experiments.}
    \label{fig:lhc}
\end{figure}

There are 8 interaction regions (IRs) at which two proton beams cross, and protons beams are injected into the LHC from two IRs. Before protons are injected to the LHC, they undergo a multi-stage acceleration by several accelerators~\cite{Bruning:782076}: a linear accelerator (LINAC2), the Proton Synchrotron Booster, the Proton Synchrotron (PS), and the Super Proton Synchrotron (SPS). The proton beams are accelerated up to 450 GeV when they are injected to the LHC with a 25 ns bunch spacing in Run II. There are more than $10^{11}$ protons in each bunch, and the large number of protons in each bunch results in multiple collisions per bunch crossing, knowns as \textit{pile-up}. In 2016, the mean number of interactions per bunch crossing was $\langle\mu\rangle = 24.9$.

The four main experiments at CERN are distributed around the LHC at collision points. Two experiments, ATLAS and CMS, are designed as general purpose experiments, and A large Ion Collider Experiment (ALICE) and the Large Hadron Collider beauty (LHCb) are designed to study strong interaction using heavy ion collisions and the matter-antimatter asymmetry using b quarks, respectively. Figure~\ref{fig:lhc} shows the four main experiments and the accelerators at the LHC.


\section{The ATLAS detector}
\label{sec:atlas:detector}

\begin{figure}[!htb]
    \includegraphics[width=0.8\textwidth]{figs/atlas.png}
    \centering
    \caption{The ATLAS detector and the sub-detector systems.}
    \label{fig:atlas}
\end{figure}

The ATLAS detector is a multi-purpose detector designed to investigate a wide range of physics, including the search for the Higgs boson in Run I and many searches beyond the SM. The detector measures 46 m long, 25 m in diameter, and it has three main layers of sub-detectors to detect particles created from $pp$ collisions at the Interaction Point (IP). Figure~\ref{fig:atlas} shows the ATLAS detector and the sub-detector systems: the Inner detector, the electromagnetic and hadronic calorimeters, and the muon spectrometer. In this section, the sub-detectors and the magnet system of the ATLAS detector are discussed.

\subsection{The Inner Detector}
\label{sec:atlas:id}
The Inner Detector (ID) is a particle tracker designed to provides information about transverse momentum, charge, and trajectory of charged particles. It consists of cylindrical barrels and two end-cap disks from three sub-detectors centered around the IP. The ID covers the pseudo-rapidity region up to $|\eta| < 2.5$. Pseudo-rapidity describes the angle of a particle with respect to the $z$ axis as defined in Eq.~\ref{eq:eta}, and it is particularly useful because the quantity is Lorentz invariant under a boost along the $z$ axis.

\begin{equation}
    \label{eq:eta}
    \eta = -\ln (\tan (\frac{\theta}{2}))
\end{equation}

The detector is placed under a 2 T magnetic field generated by the superconducting solenoid magnets, and the magnetic field is used to alter trajectories of charged particles. The curvature of the trajectory provides the transverse momentum and the charge of a particle, and the spatial information is used to reconstruct primary and secondary vertices.

The sub-detectors of the ID, the Pixel tracker, the semiconductor tracker (SCT), and the transition radiation tracker (TRT), are discussed in the following sections.

\subsubsection{Pixel Detector}
\label{sec:atlas:pixel}

The Pixel detector is a semiconductor detector in the innermost part of the ATLAS tracking system. The detector consists of 4 concentric layers of barrel detectors and 3 disk detectors at each end-cap region. The barrels and disks are made of a rectangular Pixel modules containing 50$\times$400 \si{\micro\meter^{2}} pixels (50$\times$250 \si{\micro\meter^{2}} for the IBL)~\cite{1748-0221-10-06-C06012}. As a charged particle traverse through pixels, the currents generated by ionizing electrons are measured and registered as hits. The pixels provide spatial information with resolution of $\sim$ 8 \si{\micro\meter} in radial direction and $\sim$ 75 \si{\micro\meter} in the beam axis, and the information is used for momentum measurements as well as reconstruction of primary and secondary vertices. Several upgrades had been performed for Run 2 including the installation of the Insertable B-Layer (IBL)~\cite{2011NIMPA.650...45H} to maintain and improve the performance of the ATLAS detector under increasing pile-up.

\subsubsection{Semi-conductor Tracker}
\label{sec:atlas:sct}

The Semi-Conductor Tracker (SCT) is the next tracking system following the Pixel detector. Similar to the Pixel detector, the SCT consists of 4 layers of barrel detectors and 9 disk detectors at each end-cap region. Each barrel/disk is made of SCT modules containing double-sided silicon strips, measuring 80 \si{\micro\meter} wide and 12 \si{\centi\meter} long\footnote{There are two versions of SCT strips in end-cap modules with lengths of 12 \si{cm} and 7 \si{cm}}. Strips are positioned parallel (perpendicular) to the beam axis in the barrel (end-cap) region. Because a single strip can only provide spatial information in ($r$-$\phi$) direction in barrel and ($z$-$\phi$) direction in end-cap region, double-sided strips are displaced by a relative angle of 40 \si{\milli\radian} to provide three-dimensional spatial measurements of charged particles. The SCT has a spatial resolution of 17 \si{\micro\meter} in radial direction and 580 \si{\micro\meter} in $z$ direction. The information collected by the SCT is used for transverse momentum measurement and reconstruction of vertices.


\subsubsection{Transition Radiation Tracker}
\label{sec:atlas:trt}

The Transition Radiation Tracker (TRT) is the outermost tracking system in the ID, covering the region up to $|\eta| < $ 2.0. The TRT barrel region is covered by 52,544 straw tubes aligned parallel to the beam axis. The TRT end-cap region is covered by 122,800 straw tubes aligned in radial direction. Each straw tube is filled with a Xe-based gas mixture, and a wire is placed at the center of the tube, acting as an anode. When a charged particle traverses the detector, it ionizes the gas mixture inside straws, producing a cascade of electrons. These electrons from the ionization drift toward the center wire, creating signal for the readout. The intrinsic resolution of a single straw tube is larger ($\sim$120\si{\micro\meter}) than the silicon detectors~\cite{Vogel:1537991}, and a single straw tube do not provide information along the tube direction. However, time information\footnote{The signals from straw tubes are sampled in 24 time bins of 3.12\si{\nano\second}.} from large numbers of nearby TRT hits can be used to reconstruct the trajectory of a charged particle.

The TRT also provides important information on particle identification. The spaces between straws are filled with polymer fiber (barrels) and foils (end-caps) for the production of transition radiation. When a highly relativistic charged particle passes through them, photons may be emitted by transition radiation, and these photons can be absorbed by the gas mixture, resulting in higher readout signals than usual, called high-threshold hits. The effect is stronger for electrons due to larger relativistic factor ($\gamma = E/m$) than particles with a lower boost such as hadrons. Therefore, the high-threshold hits in the TRT can be used for electron/pion identification~\cite{ATLAS-CONF-2011-128}.

\subsection{The Calorimeters}
\label{sec:atlas:calorimeter}

\begin{figure}[!htb]
    \includegraphics[width=0.8\textwidth]{figs/calorimeter.jpg}
    \centering
    \caption{The ATLAS calorimeter system.}
    \label{fig:calorimeter}
\end{figure}


The ATLAS calorimetry system, shown in Figure~\ref{fig:calorimeter}, consists of two sub-systems, the Liquid Argon (LAr) calorimeters~\cite{1742-6596-293-1-012044} and the Tile calorimeters (TileCals)~\cite{HenriquesCorreia:2004868}. The calorimeters are designed to measure the energy deposited by a particle as it traverse through the detector, and the signals from calorimeters are used for the Level 1 (L1) trigger system. Based on its usage, the calorimeters can be grouped into two sets of calorimeters: the Electromagnetic (EM) calorimeters and hadronic calorimeters.


\subsubsection{Electromagnetic Calorimeter}
\label{sec:atlas:EMcal}
The EM calorimeters (EMC) are located outside the ID and the solenoid magnet, and they are designed to measure the energy deposition from electrons and photons. They are composed of two LAr calorimeters, the EM End-Cap calorimeter (EMEC) covering the region of $|\eta|<$ 1.475 and the EM Barrel (EMB) calorimeter covering the region of 1.375 $<|\eta|<$ 3.2. The LAr calorimeters are composed of layers of high density material (Pb) and LAr sampling layer interspaced for absorption of electron/photons and energy measurement, respectively. The first layer of the LAr calorimeters is called \textit{presampler} which is a thin layer of liquid argon without absorber in front, and it is used to correct for the energy loss before a particle reach the calorimeter. The LAr calorimeters are called Sampling Calorimeters because only a small fraction of the deposited energy is measured by sampling layers.

When an electron or a photon enters the calorimeters, the electron/photon interacts with the absorber layers, creating the initial EM shower via bremsstrahlung and pair-production. The EM shower is amplified and collected by the sampling layers. The EM calorimeters have the minimum number of radiation length of 24 $X_{0}$~\cite{ATLAS-LAR-CALORIMETER}.




\subsubsection{Hadronic Calorimeter}
\label{sec:atlas:Hcal}
Hadrons are less likely to produce bremsstrahlung radiation due to heavier mass, and they can traverse through the EMC without losing significant energy. Therefore, the hadronic calorimeters are located outside the EMC to measure the energy of hadrons penetrating the EMC. The hadron calorimeters are composed of both LAr calorimeters and TileCals.

The hadronic LAr calorimeters consists of two parts, the hadronic end-cap calorimeter (HEC) covering the region of 1.5 $<|\eta|<$ 3.2 and the forward calorimeter (FCAL) covering the region of 3.1 $<|\eta|<$ 4.9. The principal of hadronic LAr calorimeters is the same as EM LAr calorimeters. The HEC is divided into four longitudinal layers with copper absorber. The FCAL consists of one EM layer which uses copper as absorber and two hadronic layers with tungsten as passive material. When a hadron enters the hadronic LAr calorimeters, the particle interacts with nuclei of the absorber material via strong force, creating a hadronic shower. The hadronic shower is sampled by sampling layers.

 TileCals are designed to cover the central barrel region ($|\eta| <$ 1.0) and the extended barrel region (0.8 $<|\eta|$ 1.7). The TileCals are made of alternating layers of iron and scintillating tiles. Hadrons entering the absorber layers produce hadronic showers, and the secondary particles from the showers interact with the scintillating tiles to produce lights. The photons are delivered to photomultipliers via wave-length shifting fibers and registered as calorimeter cluster hit.

The TileCals, combined with the hadronic LAr calorimeters, provide precise measurement of hadrons, jets, taus, and missing transverse energy ($E_{T}^{miss}$).


\subsection{The Muon Spectrometer}
\label{sec:atlas:ms}

\begin{figure}[!htb]
    \includegraphics[width=0.7\textwidth]{figs/ms.png}
    \centering
    \caption{The ATLAS Muon Spectrometer}
    \label{fig:ms}
\end{figure}
The Muon Spectrometer (MS) is the outermost detector in the ATLAS detector, responsible for muon identification, momentum measurements, and triggering of L1 Trigger. The MS is designed to provide high $p_{T}$ momentum measurement with resolution of $\sigma_{p_{T}} / p_{T} = 10\%$, independent from the ID. The MS consists of four sub-detector systems. The Resistive Plate Chambers (RPCs) and the Thin Gap Chambers (TGCs) are used to trigger L1 Trigger. The Monitored Drift Tubes (MDTs) and the Cathode Strip Chambers (CSCs) allow precise tracking and momentum measurements. The MS barrel region covers the region of $|\eta|<$ 1, and the coverage is extended to $|\eta|<$ 2.7 in the end-cap. The magnetic field is applied with an average 0.5 T~\cite{ARNAUD2008265} by a system of a large toroidal magnet (barrel) and two smaller toroidal magnets (end-cap). A layout of the MS is shown in Figure~\ref{fig:ms}.

\subsubsection{Monitored Drift Tube Chambers}
\label{sec:atlas:mdt}

The MDT chambers provide precision momentum measurements using chambers of drift tubes filled with a gas mixture of Ar (97\%) and $\mathrm{CO}_{2}$(3\%). The barrel MDT chambers are arranged in three layers, covering the region of $|\eta|<$ 1.1. The end-cap MDT chambers are arranged in three wheels, covering the region of 1.1 $<|\eta|<$ 2.7. Each drift tubes measures 3 \si{\centi\meter} in a diameter, and a tungsten-rhenium wire is placed at the center as an anode. The layers and wheels have 6-8 layers of drift tubes with a space in between. The principal of particle detection is similar to the TRT, but the drift tubes are arranged perpendicular to the beam axis in the barrel layers and tangential in the end-cap layers. This configuration allows the MDT to measure the curvature of a muon in $\eta$-$z$ plane under the toroidal magnetic field.

\subsubsection{Cathode Strip Chambers}
\label{sec:atlas:csc}

The CSCs are located on the innermost wheel of the end-caps, covering the region of 2.0 $<|\eta|<$ 2.7. In this region, the MDTs would not operate properly due to the high rates of muons coming from the beam pipe. Instead, sixteen CSCs on each end-cap provide precise tracking for high density muons with the resolution of 60 \si{\micro\meter} in $\eta$-$z$ plane and 5 \si{\milli\meter} in the radial direction. The CSCs are multi-wire proportional chambers with segmented cathode strips in alternating alignment. The multi-wires are aligned in the radial direction, and the strips are aligned perpendicular or parallel to the wires to provide 2 dimensional measurements in both $\eta$ and $\phi$ directions.


\subsubsection{Resistive Plate Chambers}
\label{sec:atlas:rpc}

The RPCs provide trigger capability for the L1 muon trigger and measurements in $\eta$, $\phi$ coordinates in the barrel region. There are three RPCs arranged in concentric cylindrical shells around the beam axis. The two inner RPCs are placed at the radii of 5\si{\meter} and 7.5\si{\meter} to provide the low-$p_{T}$ trigger information. The outer RPCs are installed at the radius of 10\si{\meter} to provide high-$p_{T}$ trigger information. Each chamber consists of two parallel resistive plots separated by a 2 \si{\milli\meter} gap filled with a gas mixture based on $\mathrm{C}_{2}\mathrm{H}_{2}\mathrm{F}_{4}$. A muon track passing through the RPCs ionizes the gas, and the signal is multiplied and read out by each chamber, providing 6 measurements for each track.

\subsubsection{Thin Gap Chambers}
\label{sec:atlas:tgc}

The TGCs are designed to provide trigger information and $\phi$ measurements (non-bending direction) in the end-cap region. Similar to the CSCs, the TGCs are made of multi-wire proportional chambers filled with a gas mixture of 55\% $\mathrm{CO}_{2}$ and 45\% $\mathrm{C}_{5}\mathrm{H}_{12}$. The strips are oriented in radial direction, allowing measurements in $\phi$. The TCGs consists of 4 layers of chambers on each end-cap, covering the region of 1 $<|\eta|<$ 2.7.


\subsection{The ATLAS Magnet}
\label{sec:atlas:magnet}

The ATLAS detector has three major superconducting magnet systems surrounding the ID and the MS: the Central Solenoid magnet and the Barrel/End-cap Toroid magnets.

\subsubsection{Central Solenoid Magnet}
\label{sec:atlas:solenoid}
The Central Solenoid magnet surrounds the ID, producing a 2 T magnetic field. The solenoid is designed to bend the trajectory of a charged particle in $r$-$\phi$ plane. The curvatures of charged particles are used for the measurements of charge-to-momentum ratio. The solenoid magnet measures 2.4 \si{\meter} in diameter and 5.3 \si{\meter} in length, and 7.73 \si{\kilo\ampere} of current is applied to generate the solenoidal magnetic field.


\subsubsection{Toroid Magnets}
\label{sec:atlas:toroid}

In the outer region of the ATLAS detector, two superconducting toroid magnet systems are used to generate a magnetic field within the MS. The Barrel Toroid are composed of 8 separate coils made of Al/NbTi/Cu conductor with 120 turns. The toroid measures 20.1 \si{\meter} in diameter and 25.3 \si{\meter} in length. The End-cap Toroid has 8 coils with 116 turns, and the coils are made of the same material as the Barrel Toroid. The nominal current of 20.5 \si{\kilo\ampere} is applied to generate a magnetic field in the range of 1 to 2 T with the field integral between 2 to 8 Tm.

\section{The ATLAS Trigger System}
\label{sec:atlas:daq}

\begin{figure}[!htb]
    \includegraphics[width=0.8\textwidth]{figs/tdaq.png}
    \centering
    \caption{Layout of the ATLAS Trigger and data acquisition system in Run 2.}
    \label{fig:tdaq}
\end{figure}

The ATLAS trigger system is desigend to select interesting events from raw event data at high rate, generated by $pp$ collisions. In Run 2, the LHC collide proton bunches every 25 \si{\nano\second}, resulting in about a billion $pp$ colliisions per second at $\langle\mu\rangle = 24.9$ (in 2016). Due to the limited bandwidth and computing resources, it is impossible to record all events from the collisions. Therefore, the ATLAS trigger system is implemented to reduce the data taking rate from 40 \si{\mega\hertz} to 100 \si{\kilo\hertz}. In Run 2,several upgrades have been performed in both hardware and software to maintain data quality in the environment with increasing pile-up, higher instananeous luminosity at the center-of-mass energy of $\sqrt{s}=$13 TeV.

The trigger system in Run 2 consists of a hardware-based Level 1 (L1) trigger and a software-based High Level Trigger (HLT). Figure~\ref{fig:tdaq} shows the ATLAS trigger and data acquisition system in Run 2. The events accepted by the trigger system are further processed by the Data Acquisition (DAQ) system where the information from front-end electronics of each detector component are used to build an individual event. The reconstructed events from DAQ are sent to Data Storage (SFO) for permanenet storage.



\subsection{Level 1 Trigger}
\label{sec:atlas:l1trigger}

The L1 trigger is a hardware-based trigger system that operates at the maximum rate of 100 \si{\kilo\hertz}~\cite{1742-6596-762-1-012003}. The main components of the L1 trigger consist of the L1 calorimeter trigger system and the L1 muon trigger system . The L1 trigger uses custom electronics to make fast decisions and find regions of interest (RoI) in the detector, an important concept of the ATLAS trigger system, where potentially interesting activities are seen by the calorimeters or the MS.

A list of trigger selection from the L1 the HLT is developed based on the physics goal of the collaboration and the needs of individual analyses. The list is called the Trigger Menu~\cite{VazquezSchroeder:2287548}. The L1 trigger accepts the events with high $p_{T}$ tracks, jets, or large $E_{T}^{miss}$ that satisfies one of the trigger menu. The events accepted by the L1 trigger is passed to the HLT.


\subsection{High Level Trigger}
\label{sec:atlas:hlt}

In Run 2, two software-based trigger systems, the Level 2 trigger and the Event Filter, are combined into a single software trigger system, called the High Level Trigger. The HLT makes the decision on events based on full information from the detector read-out in the RoI passed by the L1 trigger. This includes a fast reconstruction of the inner detector tracks. The HLT has the average output rate of 1 \si{\kilo\hertz}~\cite{1742-6596-762-1-012003}, constrained by data storage limitation. 



\chapter{Z' Reconstruction}
\label{chap:reco}

The experimental signature in the search for long-lived $Z'$ is characterized by a high-$p_{T}$ dilepton vertex, displaced (>2 \si{\milli\meter}) from the primary vertex in transverse plane, referred a \textit{displaced vertex}. In this chapter, the reconstruction of displaced vertices in the ATLAS ID is described.

\section{Track Reconstruction in the Inner Detector}
\label{sec:reco:track}

Track reconstruction in the ATLAS uses pattern recognition algorithms to reconstruct the trajectories of charged particles, referred as \textit{track}. When a charged particle traverse through the ID, the particle interact with the sub-detectors of the ID (Pixel, SCT, and TRT), leaving raw detector signals. The raw signals are digitized and registered as detector \textit{hits}, and these detector hits are used for track reconstruction. 

\subsection{Standard Tracking}
\label{sec:reco:st}


\begin{figure}[!htb]
    \includegraphics[width=0.6\textwidth]{figures/tracking.png}
    \centering
    \caption{Illustration of detector hits and reconstructed tracks in the ID. The bright colors represent the detector hits associated with reconstructed tracks.}
    \label{fig:tracking}
\end{figure}

The standard ATLAS track reconstruction is the main track reconstruction algorithm used in the ATLAS experiment. In the first stage of the track reconstruction, detector hits from the Pixel or the SCT detector are used to create \textit{track seeds}, collections of silicon hits used for the initial track finding. If a track seed passes certain quality criteria, including a $p_{T}$ and impact parameter selection, the track seed is extended to the outer part of the ID using a window search and pattern recognition algorithms. The extended tracks are evaluated based on $p_{T}$, number of hits, and impact parameters, and only the tracks satisfying the standard track selections are stored in the track collection. Figure~\ref{fig:tracking} illustrates detector hits and reconstructed tracks in the ID. The important standard track selections is summarized in Table~\ref{table:tracking}~\cite{ATL-PHYS-PUB-2017-014}.


Tracks are described by five parameters (helix) and a reference point, using a perigee representation. The parameters are the transverse (longitudinal) impact parameter $d_{0}$ ($z_{0}$), the azimuthal angle $\phi$, the polar angle $\theta$, and the charge-to-momentum ratio, $q/p$. The average position of the $pp$ interaction, referred as beamspot position, is used as the reference point for the track representation~\cite{Aaboud:2016rmg}.


\begin{table}[!htb]
  \centering
  \begin{tabular}{ c  c  c }
    \hline
    \hline
    & Standard & Large radius \\ [0.5ex]
    \hline
    Maximum $d_{\rm{0}}$ (mm) & 10 & 300 \\
    Maximum $z_{\rm{0}}$ (mm) & 250 & 1500 \\
    Maximum $|\eta|$ & 2.7 & 5 \\
    Maximum shared silicon modules & 1 & 2 \\
    Minimum unshared silicon hits& 6 & 5 \\
    Minimum silicon hits & 7 & 7\\
    \hline
    \hline
  \end{tabular}
  \caption{Main track selection in the standard and the LRT algorithms.}
  \label{table:tracking}
\end{table}


\subsection{Large Radius Tracking}
\label{sec:reco:lrt}

The standard track reconstruction is proven to be very efficient in Run 2 with the tracking efficiency > 90\%~\cite{Aaboud:2017all}. However, the tracking algorithm is optimized for primary charged particles promptly produced from $pp$ collisions, and the strict requirements on the impact parameters, \dzero and \zzero potentially limit the tracking efficiency for charged particles with large impact parameters ($|d_{0}|$ > 2 mm). Therefore, a dedicated track reconstruction algorithm, referred to as the large radius tracking (LRT), is developed to improve the track reconstruction efficiency for the tracks highly displaced from the primary vertex.

The LRT is performed in a sequence, following the standard tracking. It follows the same reconstruction strategy as the standard tracking, but there are a few importance difference between the standard tracking and the LRT:

\begin{itemize}
	\item The pattern recognition algorithms in the LRT only uses \textit{un-used} hits, the silicon hits that have not been used in the standard tracking, in creating and extending track seeds.
	\item The requirements on tracks such as $d_{0}$, $z_{0}$, and number of hits are relaxed.
\end{itemize}

Tracks reconstructed by the LRT using un-used hits are required to pass certain criteria such as minimum $p_{T}$ and number of detector hits, and the selected tracks are merged into the standard track collection for the next step of reconstruction. The track selections in the LRT are summarized and compared with the standard track reconstruction in Table~\ref{table:tracking}. The combined track collection is used as an input for the lepton reconstruction and identification and secondary vertex reconstruction. More details on the LRT can be found in Ref.~\cite{ATL-PHYS-PUB-2017-014}.

Combined with the standard track reconstruction, the LRT provides overall efficiency of $90\%$ or above for the particles that satisfy the fiducial selections in Table~\ref{table:TechEffTable} with a displacement in the transverse plane up to 300 \si{\milli\meter}.

\begin{table}[!htb]
\centering
\begin{tabular}{ r c l }
 \hline
 \hline
  \multicolumn{3}{c}{Fiducial Selections}  \\
 \hline
 $r_{\rm prod}$ & $<$ & 300~mm \\
$|\eta|$ & $<$ & 5 \\
 $\pt$ & $>$ & 1~GeV  \\
 Number of silicon hits & $\ge$ & 7  \\
\hline
\hline
\end{tabular}
\caption{Fiducial selections on particles for tracking efficiency measurements.}
\label{table:TechEffTable}
\end{table}


\section{Electron Reconstruction}
\label{sec:reco:lep}

Electrons are characterized by energy deposits in the EM calorimeter and the associated reconstructed tracks in the ID. The electron reconstruction algorithm uses the energy deposits with total transverse energy $>2.5$ \si{\GeV} in a window size of 0.075$\times$0.125 in ($\eta$,$\phi$) to reconstruct EM clusters. The EM clusters are associated with tracks within the same RoI, $|\Delta\eta|<0.05$ and $|\Delta\phi|<0.1$ where the effect of bremsstrahlung is taken into account~\cite{Aad:2014fxa}. In the absence of a matching track, the EM cluster is classified as a photon candidate. The associated pairs of tracks and energy clusters are refitted to improve the energy resolution of the electron candidate. Additional electron requirements, including likelihood-based identification criteria, are imposed on the electron candidates to reduce misidentification.

%By default, electrons are required to have a minimum number of pixel hits and small transverse impact parameters. This is not optimal for the searches that aim to detect displaced vertices as the decay products of displaced vertices tend to have large impact parameters (\dzero, \zzero) and missing hits in the inner layers of the detector. Therefore, these requirements are removed in this analysis.

\section{Muon Reconstruction}
\label{sec:reco:muon}

Muons deposit very small energy in the calorimeter as they traverse through the detector due to the relatively larger mass, given that the probability of bremsstrahlung is $\propto 1/m^{2}$. However, they leave tracks in both the ID and the MS, referred as inner detector (ID) tracks and muon standalone (MS) tracks. The muon reconstruction algorithm uses MS tracks as seeds, and the MS tracks are extrapolated to the ID for the association with ID tracks. A \textit{combined muon} track is created if a MS track is successfully associated with an ID track after the momentum correction for the energy loss from the interaction with the detector material. There are other types of reconstructed muons such as standalone muons, segment-tagged muons, and calorimeter-tagged muons~\cite{Aad:2016jkr}. However, these types of muons are not considered in this analysis as they do not have associated ID tracks which are required for the reconstruction of displaced vertices.

By default, muons are required to have a minimum number of Pixel, SCT hits and small transverse impact parameters. This is not optimal for the searches that aim to detect displaced vertices as the decay products of displaced vertices tend to have large impact parameters (\dzero, \zzero) and missing hits in the inner layers of the detector. Therefore, the requirements on \dzero and Pixel hits are removed. Also minimum SCT hits on muon tracks are lowered to 2.

\section{Secondary Vertex Reconstruction}
\label{sec:reco:dv}

\begin{figure}[!htb]
    \includegraphics[width=0.6\textwidth]{figures/vertex.png}
    \centering
    \caption{Main vertex topologies in $pp$ collisions: primary, pile-up, and secondary vertices.}
    \label{fig:vertex_topology}
\end{figure}

In the LHC, when two proton bunches collide, several different vertex topologies arise. The primary vertex and several pile-up vertices are formed along the beam line, and the vertices from photon conversion or long-lived particles are formed displaced from the primary vertex as shown in Figure~\ref{fig:vertex_topology}. These displaced vertices are referred as secondary vertices. In the search for long-lived $Z'$ in the dilepton decay channel, the decay products of long-lived particles can be reconstructed as a secondary vertex.

The secondary vertex reconstruction is based on the primary vertex reconstruction algorithm~\cite{1742-6596-119-3-032033} and was originally developed for the study of material mapping of the ID in Run I. The algorithm was updated for several long-lived particle searches in Run 2.

In the first stage, track are selected using the requirements on track parameters and hit patterns shown in Table~\ref{table:vertex_track_selection_simple}. The tracks reconstructed by both the standard track reconstruction and the LRT are used as input. The tracks passing the track requirements are used to create two-track vertices, based on the closeness of two tracks in the space. This process results in a large number of fake vertices. The fake vertices are rejected by considering the location of a vertex and hit patterns of the tracks associated with the vertex. A vertex is rejected if the associated tracks have any hits at a radius smaller than the vertex position. Two-track vertices passing the fake rejection are refitted using a Kalman Filter~\cite{Kalman:434680} for precise vertex position measurements, and track parameters are calculated with respect to the secondary vertex.

The two-track vertices reconstructed by the secondary vertex reconstruction serves as the primary analysis object in this thesis. %Section~\ref{sec:vertex_selection} discusses the vertex selection applied to these two-track vertices to select displaced vertices. Lepton identification requirements are applied to the tracks from secondary vertices after applying analysis level vertex selection, so secondary vertices reconstructed can have any combination of muon, electron, or non-leptonic tracks.







\begin{table}[!htb]
  \centering
  \begin{tabular}{ l c }
    \hline
    \hline
	Variable      		& Cut                                         	\\
    \hline
	$p_{T}$ (GeV)		& $>$ 1.0										\\
	$\chi^{2} / DOF$	& $<$ 50  										\\
	$d_{0}$	(mm)		& 2.0 - 300.0									\\
	$z_{0}$ (mm)		& $<$ 1500.0									\\
	SCT hits			& $\geq$ 2										\\
	Si shared hits	    & $\leq$ 2										\\
	Pixel and TRT hits  & TRT hits > 0 or Pixel hits $\geq$ 2			\\
    \hline
    \hline
  \end{tabular}
  \caption{Track requirements for secondary vertex reconstruction.}
  \label{table:vertex_track_selection_simple}
\end{table}



\chapter{Analysis Overview}
\label{chap:analysis_overview}

This thesis presents a search for a heavy long-lived resonance decaying to a dilepton pair, \mumu, \ee, or \emu within the ATLAS ID. The analysis uses 32.8 $\mathrm{fb^{-1}}$ of $pp$ collision data at $\sqrt{s}=13$ TeV collected in 2016 with the ATLAS detector. The long-lived particle (LLP) is referred as $Z'$ but with no assumption on the $Z'$ production mechanism for a model-independent search.

There have been several searches for the LLPs produced in $pp$ collisions in Run I at $\sqrt{s} =$ 8 TeV, including the search for displaced hadronic jet~\cite{Blackburn:1550730}, displaced heavy flavors~\cite{Harris:1512932}, or multi-track displaced vertex~\cite{Aad:2015rba}, and no significant excess was observed. The signature considered in this analysis is distinct from the previous searches, and it is one of the first efforts in the ATLAS experiment to search for a generic displaced vertex signature decaying to a dilepton pair. This analysis focuses on interpreting the LLPs decaying to displaced dilepton vertices in the context of model-independent, exotic resonance search.

In this search, a special setup (Section~\ref{sec:reco:lrt}) of data reprocessing and reconstruction is used in order to gain sensitivity for the non-conventional signature of LLPs. The special setup allows the reconstruction of tracks with large impact parameters and secondary vertices significantly displaced from primary vertices. 

\chapter{Data and MC samples}
\label{chap:data_mc}

\section{Data samples}
\label{sec:data_mc:data}

The analysis uses the 2016 $pp$ collisions data (periods A-L) with the integrated luminosity of 32.8 $\mathrm{fb^{-1}}$. In this search, because the standard ATLAS track reconstruction does not provide good sensitivity for long-lived particles, a dedicated physics stream, \texttt{DRAW\_RPVLL}, is used to reconstruct events using the non-standard reconstruction as described in Chapter~\ref{chap:reco}. The stream is used in several Exotics and SUSY analyses, searching for long-lived particles. %non-standard reconstruction objects.

In this stream, a subset of events from the main physics stream is selected by \texttt{RPVLL} filters. The filters select events using the HLT and offline selections configured for each analysis. The triggers and offline selection used in this search is discussed in Chapter~\ref{chap:signal_selection}. The selected events are passed downstream for reconstruction. The data is in \texttt{RAW} format so that low-level information such as detector hits can be used for the special reconstruction algorithms to reconstruct displaced tracks and vertices.

The events passing HLTs are processed by a sequence of reconstruction algorithms with varying configurations, and the ATLAS metadate interface (AMI) tags are used to specify the configurations to be used for a data processing chain. In this analysis, the events are centrally processed with AMI tag \texttt{r8669} in which the dedicated track reconstruction algorithm, the LRT, and the secondary vertex reconstruction algorithm are enabled to reconstruct tracks and vertices, respectively. The output of \texttt{DRAW\_RPVLL} stream is in \texttt{DAOD\_RPVLL} format which is a standard \texttt{xAOD} data format with additional displaced tracks and secondary vertices reconstructed. 

The \texttt{DAOD\_RPVLL} is further processed to produce the \texttt{DAOD\_SUSY15} derivation for data reduction and software fixes on analysis objects such as energy calibration\footnote{Software fixes are released by the ATLAS collaboration in the form of \texttt{AODfix}.}, as recommended by the Analysis Model Study Group (AMSG)~\cite{Catmore:1543445}. Table~\ref{table:data_samples} summarizes datasets used in this search.

\begin{table}[!htb]
  \centering
  \resizebox{\textwidth}{!}{
    \begin{tabular}{ l l }
      \hline
      \hline
      Format     				& Dataset       													\\
      \hline
      \texttt{DRAW\_RPVLL}	& data16\_13TeV.*.physics\_Main.merge.DRAW\_RPVLL.f*\_m*			\\
      \texttt{DAOD\_RPVLL}	& data16\_13TeV.*.physics\_Main.recon.DAOD\_RPVLL.f*\_r8669			\\
      \texttt{DAOD\_SUSY15}	& data16\_13TeV.*.physics\_Main.recon.DAOD\_RPVLL.f*\_r8669\_p3185	\\
      \hline
      \hline
    \end{tabular}
  }
  \caption{Dataset used in \texttt{DRAW\_RPVLL}, \texttt{DAOD\_RPVLL}, and \texttt{DAOD\_SUSY15} format.}
  \label{table:data_samples}
\end{table}

This search uses a modified version of the standard \texttt{GoodRunsList} because a small number of events selected by \texttt{DRAW\_RPVLL} was not reconstructed successfully. The corresponding luminosity blocks were removed from the \texttt{GoodRunsList}\footnote{\url{data16\_13TeV.periodAllYear\_DetStatus-v83-pro20-15\_DQDefects-00-02-04\_PHYS\_StandardGRL\_All\_Good\\\_25ns\_DAOD\_RPVLL\_r8669.xml}}.


The tag and probe studies of Section~\ref{sec:trigger_efficiency} are performed on the standard xAOD dataset using derivations of the performance groups, given in Table~\ref{table:data_samples:tag_probe}.  

\begin{table}[!htb]
  \centering
  \resizebox{\textwidth}{!}{
    \begin{tabular}{ l l }
      \hline
      \hline
      Format     				& Dataset       													            \\
      \hline
      \texttt{DAOD\_EGAM1}	& data16\_13TeV.*.physics\_Main.PhysCont.DAOD\_EGAM1.grp16\_v01\_p3013    \\
      \texttt{DAOD\_MUON1}	& data16\_13TeV.*.physics\_Main.PhysCont.DAOD\_MUON1.grp16\_v01\_p3043	\\
      \hline
      \hline
    \end{tabular}
  }
  \caption{Datasets used for tag and probe studies.}
  \label{table:data_samples:tag_probe}
\end{table}



\section{MC samples}
\label{sec:mc_sample}

\subsection{Signal samples}
\label{sec:signal_sample}
The long-lived $Z'$ is generated using \pythia 8.1/8.2~\cite{1126-6708-2006-05-026} with the \texttt{NNPDF23LO} PDF set~\cite{Ball:2014uwa} and the min-bias tune \texttt{A14}. In this signal samples, $Z'$ is singly produced from $q\bar{q}$ scattering and decays to a $\mu\mu$, $ee$, or $e\mu$ pair. The proper lifetime, $c\tau$, is set to 100, 250, or 500 mm. The mass of $Z'$ is set between 100 and 1000 GeV. A width based on relativistic Breit-Wigner is assumed for the new resonance. A sample of 20k events are generated for each mass and lifetime. Table~\ref{table:MC_signal_samples} summarizes dataset identifiers (DIDs), mass, width, and lifetime of the signal MC samples used in this search.

\begin{table}[!htb]
  \centering
  \begin{tabular}{ c c c c c c c }
    \hline
    \hline
           &   &    & \multicolumn{3}{c}{DID}                 \\
    $m_{Z'}$ (GeV) & $\Gamma$ (GeV) & $c\tau$ (mm) &$\mu\mu$ & $ee$ & $e\mu$ \\
    \hline
    100			   &	2.8         &   100	& 308264	& 309539		&	309554		\\
    100			   &	2.8         &   250	& 308265	& 309540		&	309555		\\
    100			   &	2.8         &   500	& 308266	& 309541		&	309556		\\
    250			   &   6.9	        &   100	& 301911	& 309542		&	309557		\\
    250			   &	6.9         &   250	& 301912	& 309543		&	309558		\\
    250			   &	6.9         &   500	& 301913	& 309544		&	309559		\\
    500			   &   14.7         &   100	& 301914	& 309545		&	309560		\\
    500			   &	14.7        &   250	& 301915	& 309546		&	309561		\\
    500			   &	14.7        &   500	& 301916	& 309547		&	309562		\\
    750			   &	23.0        &   100	& 308285	& 309548		&	309563		\\
    750			   &	23.0        &   250	& 308286	& 309549		&	309564		\\
    750			   &	23.0        &   500	& 308287	& 309550		&	309565		\\
    1000	       &	31.0        &   100	& 301917	& 309551		&	309566		\\
    1000	       &	31.0        &   250	& 301918	& 309552		&	309567		\\
    1000	       &	31.0        &   500	& 301919	& 309553		&	309568		\\
    \hline
    \hline
  \end{tabular}
  \caption{Mass, lifetime, and DID of the signal MC samples.}
  \label{table:MC_signal_samples}
\end{table}

The signal MC samples generated using \pythia are processed to include detector simulation using the AMI tags \texttt{s2698} and \texttt{s2726}. The samples are overlaid with simulated minimum-bias events to model multiple interactions (pile-up) in data samples. In the signal MC samples, the average number of pile-ups, $\langle\mu\rangle$, ranges from 10 to 40 with small number of events having $\langle\mu\rangle$ < 10. The difference in the $\langle\mu\rangle$ distributions between MC and data samples are corrected for by pile-up reweighting. The resulting MC samples are reconstructed using AMI tag \texttt{r8788}.

In the reconstruction process, the LRT and the secondary vertex reconstruction algorithms are used with the same configuration as data samples to reconstruct displaced tracks and vertices. The reconstructed events are stored in \texttt{DAOD\_RPVLL}, and the samples are processed to produce the \texttt{DAOD\_SUSY15} derivation for data reduction and software fixes on analysis objects.

The representative plots of generator-level $p_{T}$ and $\eta$ distributions of $Z'$ and the muons from the decay of $Z'$, referred as \textit{signal} muons, are shown in Figure~\ref{fig:truth_zp_muon} using the signal MC samples with $m=$ 500, 1000 GeV and $c\tau=$ 100 mm. %The signal MC samples with $ee$ and $e\mu$ final states produce similar distributions as shown in Appendix~\ref{app:signal_truth}.

%The $\eta$ distribution of $Z'$ shows a forward distribution, and it is reweighted to a flat $\eta$ distribution to provide model-independent result. The details are discussed in Section~\ref{sec:efficiency_reweighting}.
The $\eta$ distribution of signal muons shows that most of the signal muons are produced within the detector acceptance ($\eta <$ 2.7). The characteristic upper edge in the $p_{T}$ spectrum is related to the $Z'$ mass.
%The $p_{T}$ distribution of muons shows that muon $p_{T}$ is constrained by the $Z'$ mass.


\begin{figure}[!htb]
    \centering
    \subfloat[]{\includegraphics[width=0.45\textwidth]{figures/m_truth_zp_pt.pdf}}
    \subfloat[]{\includegraphics[width=0.45\textwidth]{figures/m_truth_zp_eta.pdf}} \\
    \subfloat[]{\includegraphics[width=0.45\textwidth]{figures/m_truth_muon_pt.pdf}}
    \subfloat[]{\includegraphics[width=0.45\textwidth]{figures/m_truth_muon_eta.pdf}}
    \caption{Representative plots of generator-level (a) $p_{T}$ and (b) $\eta$ distributions of $Z'$, and (c), (d) are the corresponding distributions for the signal muons. The signal MC samples are generated with $m=$ 500, 1000 GeV, and $c\tau=$ 100 mm.}
    \label{fig:truth_zp_muon}
\end{figure}


\subsection{Background MC samples}
\label{sec:background_mc_sample}
In this analysis, backgrounds are estimated from data because most of the backgrounds are expected to be originated from non-collision processes such as cosmic rays or random-crossing of tracks. %sources other than the SM collision process 

However, SM background samples are used to study the performance of random-crossing background estimation (Section~\ref{sec:bkg:random}) and to estimate the systematic uncertainties in vertexing and tracking (Section~\ref{sec:syst:vertexing}). The $t\bar{t}$ samples are generated for QCD background study using \sherpa 2.2~\cite{Gleisberg:2008ta} with the \texttt{NNPDF30NNLO} PDF set. The samples with leptonic decay of di-boson ($ZZ$, $WW$, $W^{\pm}Z$) are generated using \sherpa 2.1 with the \texttt{CT10} PDF set. The di-jet samples (JZ3W-JZ7W) are generated in slices of leading jet $p_{T}$ (160-400, 400-800, 800-1300, 1300-1800, 1800-2500 GeV) using \pythia 8.1 with \texttt{NNPDF23LO} PDF set. These samples are sufficient for testing purposes, as they contain high-$p_{T}$ isolated leptons (from $W$ boson decays) and leptons and displaced tracks in $b$-jets, in addition to tracks from pile-up vertices. Details on the PDF sets can be found in Ref.~\cite{Ball:2014uwa}.

The SM background samples are reprocessed using the same configuration as the signal MC sample for consistency. The background MC samples used for background and systematic uncertainty estimations are summarized in Table~\ref{table:background_MC}.


\begin{table}[!htb]
  \centering
  \begin{tabular}{ l l l l l}
    \hline
    \hline
    Process         &   DID     &   $\sigma$ (pb)       & Events ($10^{6}$) &   $\mathcal{L}_{Int} (\mathrm{fb^{-1}})$ \\
    \hline
    $t\bar{t}$                              &   410252  &   87.8   &   0.70        &   7.97                 \\
    $ZZ\rightarrow \ell \ell \ell \ell$     &   361063  &   11.7   &   0.12        &   10.3                 \\
    $W^{-}Z\rightarrow \ell \ell \ell v$    &   361064  &   1.68   &   0.020       &   10.2                 \\
    $W^{+}Z\rightarrow \ell \ell \ell v$    &   361066  &   2.33   &   0.025       &   30.0                 \\
    $WW \rightarrow \ell \ell vv$           &   361068  &   12.8   &   0.13        &   1.95                 \\
    JZ3W                                    &   361023  &   8.45$\cdot10^{3}$      &   0.20  &   0.0247     \\
    JZ4W                                    &   361024  &   135    &   0.20        &   1.48                 \\
    JZ5W                                    &   361025  &   4.20   &   0.20        &   47.6                 \\
    JZ6W                                    &   361026  &   2.42$\cdot10^{-2}$     &   0.20  &   826        \\
    %$t\bar{t}$                              &   410252  &   76.3   &   0.70        &   9.17                 \\
    %$ZZ\rightarrow \ell \ell \ell \ell$     &   361063  &   12.8   &   0.12        &   9.30                 \\
    %$W^{-}Z\rightarrow \ell \ell \ell v$    &   361064  &   1.84   &   0.020       &   10.9                 \\
    %$W^{+}Z\rightarrow \ell \ell \ell v$    &   361066  &   2.56   &   0.025       &   9.77                 \\
    %$WW \rightarrow \ell \ell vv$           &   361068  &   14.0   &   0.13        &   9.29                 \\
    %JZ3W                                    &   361023  &   2.65$\cdot10^{7}$      &   0.20  &   < $10^{-5}$\\
    %JZ4W                                    &   361024  &   0.255  &   0.20        &   784                  \\
    %JZ5W                                    &   361025  &   0.455  &   0.20        &   440                  \\
    %JZ6W                                    &   361026  &   258    &   0.20  &   0.775      \\
    %JZ7W                                    &   361027  &   16.2                   &   0.10  &   0.617      \\
    \hline
    \hline
  \end{tabular}
  \caption{Background MC samples used in the study of random-crossing background and in the estimation of tracking and vertexing systematic uncertainty.}
  \label{table:background_MC}
\end{table}


Additional samples are used without the special reconstruction to study the efficiency of the triggers using tag-and-probe method (Section~\ref{sec:trigger_efficiency}). These samples are given in Table~\ref{table:background_MC:tag_probe}.

%\begin{table}[!htb]
\begin{sidewaystable}[!htb]
  \centering
  \resizebox{\textwidth}{!}{
    \begin{tabular}{ l l }
      \hline
      \hline
      Format     				& Dataset       													            \\
      \hline
      \texttt{DAOD\_EGAM1}	& mc15\_13TeV.361106.PowhegPythia8EvtGen\_AZNLOCTEQ6L1\_Zee.merge.DAOD\_EGAM1.e3601\_s2576\_s2132\_r7725\_r7676\_p3012  \\
      \texttt{DAOD\_MUON1}	& mc15\_13TeV.361107.PowhegPythia8EvtGen\_AZNLOCTEQ6L1\_Zmumu.merge.DAOD\_MUON1.e3601\_s2576\_s2132\_r7725\_r7676\_p3045	\\
      \hline
      \hline
    \end{tabular}
  }
  \caption{Background MC samples without the LRT used for tag and probe studies.}
  \label{table:background_MC:tag_probe}
%\end{table}
\end{sidewaystable}





\chapter{Data and MC samples}
\label{chap:data_mc}

\section{Data samples}
\label{sec:data_mc:data}

The analysis uses the 2016 $pp$ collisions data (periods A-L) with the integrated luminosity of 32.8 $\mathrm{fb^{-1}}$. In this search, because the standard ATLAS track reconstruction does not provide good sensitivity for long-lived particles, a dedicated physics stream, \texttt{DRAW\_RPVLL}, is used to reconstruct events using the non-standard reconstruction as described in Chapter~\ref{chap:reco}. The stream is used in several Exotics and SUSY analyses, searching for long-lived particles. %non-standard reconstruction objects.

In this stream, a subset of events from the main physics stream is selected by \texttt{RPVLL} filters. The filters select events using the HLT and offline selections configured for each analysis. The triggers and offline selection used in this search is discussed in Chapter~\ref{chap:signal_selection}. The selected events are passed downstream for reconstruction. The data is in \texttt{RAW} format so that low-level information such as detector hits can be used for the special reconstruction algorithms to reconstruct displaced tracks and vertices.

The events passing HLTs are processed by a sequence of reconstruction algorithms with varying configurations, and the ATLAS metadate interface (AMI) tags are used to specify the configurations to be used for a data processing chain. In this analysis, the events are centrally processed with AMI tag \texttt{r8669} in which the dedicated track reconstruction algorithm, the LRT, and the secondary vertex reconstruction algorithm are enabled to reconstruct tracks and vertices, respectively. The output of \texttt{DRAW\_RPVLL} stream is in \texttt{DAOD\_RPVLL} format which is a standard \texttt{xAOD} data format with additional displaced tracks and secondary vertices reconstructed. 

The \texttt{DAOD\_RPVLL} is further processed to produce the \texttt{DAOD\_SUSY15} derivation for data reduction and software fixes on analysis objects such as energy calibration\footnote{Software fixes are released by the ATLAS collaboration in the form of \texttt{AODfix}.}, as recommended by the Analysis Model Study Group (AMSG)~\cite{Catmore:1543445}. Table~\ref{table:data_samples} summarizes datasets used in this search.

\begin{table}[!htb]
  \centering
  \resizebox{\textwidth}{!}{
    \begin{tabular}{ l l }
      \hline
      \hline
      Format     				& Dataset       													\\
      \hline
      \texttt{DRAW\_RPVLL}	& data16\_13TeV.*.physics\_Main.merge.DRAW\_RPVLL.f*\_m*			\\
      \texttt{DAOD\_RPVLL}	& data16\_13TeV.*.physics\_Main.recon.DAOD\_RPVLL.f*\_r8669			\\
      \texttt{DAOD\_SUSY15}	& data16\_13TeV.*.physics\_Main.recon.DAOD\_RPVLL.f*\_r8669\_p3185	\\
      \hline
      \hline
    \end{tabular}
  }
  \caption{Dataset used in \texttt{DRAW\_RPVLL}, \texttt{DAOD\_RPVLL}, and \texttt{DAOD\_SUSY15} format.}
  \label{table:data_samples}
\end{table}

This search uses a modified version of the standard \texttt{GoodRunsList} because a small number of events selected by \texttt{DRAW\_RPVLL} was not reconstructed successfully. The corresponding luminosity blocks were removed from the \texttt{GoodRunsList}\footnote{\url{data16\_13TeV.periodAllYear\_DetStatus-v83-pro20-15\_DQDefects-00-02-04\_PHYS\_StandardGRL\_All\_Good\\\_25ns\_DAOD\_RPVLL\_r8669.xml}}.


The tag and probe studies of Section~\ref{sec:trigger_efficiency} are performed on the standard xAOD dataset using derivations of the performance groups, given in Table~\ref{table:data_samples:tag_probe}.  

\begin{table}[!htb]
  \centering
  \resizebox{\textwidth}{!}{
    \begin{tabular}{ l l }
      \hline
      \hline
      Format     				& Dataset       													            \\
      \hline
      \texttt{DAOD\_EGAM1}	& data16\_13TeV.*.physics\_Main.PhysCont.DAOD\_EGAM1.grp16\_v01\_p3013    \\
      \texttt{DAOD\_MUON1}	& data16\_13TeV.*.physics\_Main.PhysCont.DAOD\_MUON1.grp16\_v01\_p3043	\\
      \hline
      \hline
    \end{tabular}
  }
  \caption{Datasets used for tag and probe studies.}
  \label{table:data_samples:tag_probe}
\end{table}



\section{MC samples}
\label{sec:mc_sample}

\subsection{Signal samples}
\label{sec:signal_sample}
The long-lived $Z'$ is generated using \pythia 8.1/8.2~\cite{1126-6708-2006-05-026} with the \texttt{NNPDF23LO} PDF set~\cite{Ball:2014uwa} and the min-bias tune \texttt{A14}. In this signal samples, $Z'$ is singly produced from $q\bar{q}$ scattering and decays to a $\mu\mu$, $ee$, or $e\mu$ pair. The proper lifetime, $c\tau$, is set to 100, 250, or 500 mm. The mass of $Z'$ is set between 100 and 1000 GeV. A width based on relativistic Breit-Wigner is assumed for the new resonance. A sample of 20k events are generated for each mass and lifetime. Table~\ref{table:MC_signal_samples} summarizes dataset identifiers (DIDs), mass, width, and lifetime of the signal MC samples used in this search.

\begin{table}[!htb]
  \centering
  \begin{tabular}{ c c c c c c c }
    \hline
    \hline
           &   &    & \multicolumn{3}{c}{DID}                 \\
    $m_{Z'}$ (GeV) & $\Gamma$ (GeV) & $c\tau$ (mm) &$\mu\mu$ & $ee$ & $e\mu$ \\
    \hline
    100			   &	2.8         &   100	& 308264	& 309539		&	309554		\\
    100			   &	2.8         &   250	& 308265	& 309540		&	309555		\\
    100			   &	2.8         &   500	& 308266	& 309541		&	309556		\\
    250			   &   6.9	        &   100	& 301911	& 309542		&	309557		\\
    250			   &	6.9         &   250	& 301912	& 309543		&	309558		\\
    250			   &	6.9         &   500	& 301913	& 309544		&	309559		\\
    500			   &   14.7         &   100	& 301914	& 309545		&	309560		\\
    500			   &	14.7        &   250	& 301915	& 309546		&	309561		\\
    500			   &	14.7        &   500	& 301916	& 309547		&	309562		\\
    750			   &	23.0        &   100	& 308285	& 309548		&	309563		\\
    750			   &	23.0        &   250	& 308286	& 309549		&	309564		\\
    750			   &	23.0        &   500	& 308287	& 309550		&	309565		\\
    1000	       &	31.0        &   100	& 301917	& 309551		&	309566		\\
    1000	       &	31.0        &   250	& 301918	& 309552		&	309567		\\
    1000	       &	31.0        &   500	& 301919	& 309553		&	309568		\\
    \hline
    \hline
  \end{tabular}
  \caption{Mass, lifetime, and DID of the signal MC samples.}
  \label{table:MC_signal_samples}
\end{table}

The signal MC samples generated using \pythia are processed to include detector simulation using the AMI tags \texttt{s2698} and \texttt{s2726}. The samples are overlaid with simulated minimum-bias events to model multiple interactions (pile-up) in data samples. In the signal MC samples, the average number of pile-ups, $\langle\mu\rangle$, ranges from 10 to 40 with small number of events having $\langle\mu\rangle$ < 10. The difference in the $\langle\mu\rangle$ distributions between MC and data samples are corrected for by pile-up reweighting. The resulting MC samples are reconstructed using AMI tag \texttt{r8788}.

In the reconstruction process, the LRT and the secondary vertex reconstruction algorithms are used with the same configuration as data samples to reconstruct displaced tracks and vertices. The reconstructed events are stored in \texttt{DAOD\_RPVLL}, and the samples are processed to produce the \texttt{DAOD\_SUSY15} derivation for data reduction and software fixes on analysis objects.

The representative plots of generator-level $p_{T}$ and $\eta$ distributions of $Z'$ and the muons from the decay of $Z'$, referred as \textit{signal} muons, are shown in Figure~\ref{fig:truth_zp_muon} using the signal MC samples with $m=$ 500, 1000 GeV and $c\tau=$ 100 mm. %The signal MC samples with $ee$ and $e\mu$ final states produce similar distributions as shown in Appendix~\ref{app:signal_truth}.

%The $\eta$ distribution of $Z'$ shows a forward distribution, and it is reweighted to a flat $\eta$ distribution to provide model-independent result. The details are discussed in Section~\ref{sec:efficiency_reweighting}.
The $\eta$ distribution of signal muons shows that most of the signal muons are produced within the detector acceptance ($\eta <$ 2.7). The characteristic upper edge in the $p_{T}$ spectrum is related to the $Z'$ mass.
%The $p_{T}$ distribution of muons shows that muon $p_{T}$ is constrained by the $Z'$ mass.


\begin{figure}[!htb]
    \centering
    \subfloat[]{\includegraphics[width=0.45\textwidth]{figures/m_truth_zp_pt.pdf}}
    \subfloat[]{\includegraphics[width=0.45\textwidth]{figures/m_truth_zp_eta.pdf}} \\
    \subfloat[]{\includegraphics[width=0.45\textwidth]{figures/m_truth_muon_pt.pdf}}
    \subfloat[]{\includegraphics[width=0.45\textwidth]{figures/m_truth_muon_eta.pdf}}
    \caption{Representative plots of generator-level (a) $p_{T}$ and (b) $\eta$ distributions of $Z'$, and (c), (d) are the corresponding distributions for the signal muons. The signal MC samples are generated with $m=$ 500, 1000 GeV, and $c\tau=$ 100 mm.}
    \label{fig:truth_zp_muon}
\end{figure}


\subsection{Background MC samples}
\label{sec:background_mc_sample}
In this analysis, backgrounds are estimated from data because most of the backgrounds are expected to be originated from non-collision processes such as cosmic rays or random-crossing of tracks. %sources other than the SM collision process 

However, SM background samples are used to study the performance of random-crossing background estimation (Section~\ref{sec:bkg:random}) and to estimate the systematic uncertainties in vertexing and tracking (Section~\ref{sec:syst:vertexing}). The $t\bar{t}$ samples are generated for QCD background study using \sherpa 2.2~\cite{Gleisberg:2008ta} with the \texttt{NNPDF30NNLO} PDF set. The samples with leptonic decay of di-boson ($ZZ$, $WW$, $W^{\pm}Z$) are generated using \sherpa 2.1 with the \texttt{CT10} PDF set. The di-jet samples (JZ3W-JZ7W) are generated in slices of leading jet $p_{T}$ (160-400, 400-800, 800-1300, 1300-1800, 1800-2500 GeV) using \pythia 8.1 with \texttt{NNPDF23LO} PDF set. These samples are sufficient for testing purposes, as they contain high-$p_{T}$ isolated leptons (from $W$ boson decays) and leptons and displaced tracks in $b$-jets, in addition to tracks from pile-up vertices. Details on the PDF sets can be found in Ref.~\cite{Ball:2014uwa}.

The SM background samples are reprocessed using the same configuration as the signal MC sample for consistency. The background MC samples used for background and systematic uncertainty estimations are summarized in Table~\ref{table:background_MC}.


\begin{table}[!htb]
  \centering
  \begin{tabular}{ l l l l l}
    \hline
    \hline
    Process         &   DID     &   $\sigma$ (pb)       & Events ($10^{6}$) &   $\mathcal{L}_{Int} (\mathrm{fb^{-1}})$ \\
    \hline
    $t\bar{t}$                              &   410252  &   87.8   &   0.70        &   7.97                 \\
    $ZZ\rightarrow \ell \ell \ell \ell$     &   361063  &   11.7   &   0.12        &   10.3                 \\
    $W^{-}Z\rightarrow \ell \ell \ell v$    &   361064  &   1.68   &   0.020       &   10.2                 \\
    $W^{+}Z\rightarrow \ell \ell \ell v$    &   361066  &   2.33   &   0.025       &   30.0                 \\
    $WW \rightarrow \ell \ell vv$           &   361068  &   12.8   &   0.13        &   1.95                 \\
    JZ3W                                    &   361023  &   8.45$\cdot10^{3}$      &   0.20  &   0.0247     \\
    JZ4W                                    &   361024  &   135    &   0.20        &   1.48                 \\
    JZ5W                                    &   361025  &   4.20   &   0.20        &   47.6                 \\
    JZ6W                                    &   361026  &   2.42$\cdot10^{-2}$     &   0.20  &   826        \\
    %$t\bar{t}$                              &   410252  &   76.3   &   0.70        &   9.17                 \\
    %$ZZ\rightarrow \ell \ell \ell \ell$     &   361063  &   12.8   &   0.12        &   9.30                 \\
    %$W^{-}Z\rightarrow \ell \ell \ell v$    &   361064  &   1.84   &   0.020       &   10.9                 \\
    %$W^{+}Z\rightarrow \ell \ell \ell v$    &   361066  &   2.56   &   0.025       &   9.77                 \\
    %$WW \rightarrow \ell \ell vv$           &   361068  &   14.0   &   0.13        &   9.29                 \\
    %JZ3W                                    &   361023  &   2.65$\cdot10^{7}$      &   0.20  &   < $10^{-5}$\\
    %JZ4W                                    &   361024  &   0.255  &   0.20        &   784                  \\
    %JZ5W                                    &   361025  &   0.455  &   0.20        &   440                  \\
    %JZ6W                                    &   361026  &   258    &   0.20  &   0.775      \\
    %JZ7W                                    &   361027  &   16.2                   &   0.10  &   0.617      \\
    \hline
    \hline
  \end{tabular}
  \caption{Background MC samples used in the study of random-crossing background and in the estimation of tracking and vertexing systematic uncertainty.}
  \label{table:background_MC}
\end{table}


Additional samples are used without the special reconstruction to study the efficiency of the triggers using tag-and-probe method (Section~\ref{sec:trigger_efficiency}). These samples are given in Table~\ref{table:background_MC:tag_probe}.

%\begin{table}[!htb]
\begin{sidewaystable}[!htb]
  \centering
  \resizebox{\textwidth}{!}{
    \begin{tabular}{ l l }
      \hline
      \hline
      Format     				& Dataset       													            \\
      \hline
      \texttt{DAOD\_EGAM1}	& mc15\_13TeV.361106.PowhegPythia8EvtGen\_AZNLOCTEQ6L1\_Zee.merge.DAOD\_EGAM1.e3601\_s2576\_s2132\_r7725\_r7676\_p3012  \\
      \texttt{DAOD\_MUON1}	& mc15\_13TeV.361107.PowhegPythia8EvtGen\_AZNLOCTEQ6L1\_Zmumu.merge.DAOD\_MUON1.e3601\_s2576\_s2132\_r7725\_r7676\_p3045	\\
      \hline
      \hline
    \end{tabular}
  }
  \caption{Background MC samples without the LRT used for tag and probe studies.}
  \label{table:background_MC:tag_probe}
%\end{table}
\end{sidewaystable}





\chapter{Background Estimation}
\label{chap:bkg}

Due to the lifetime ($c\tau > 2$ mm) and mass ($m > 10$ GeV) requirements applied at vertex selection, no SM background is expected in the signal region in search for displaced dilepton resonance. Therefore, two non-collision backgrounds are considered in this search: cosmic background and \textit{random-crossing} background. The cosmic background is the dominant background in this analysis and is estimated In Section~\ref{sec:bkg:cosmic}. In Section~\ref{sec:bkg:random}, background from random-crossing of two uncorrelated tracks are estimated.


\section{Cosmic Background}
\label{sec:bkg:cosmic}

A cosmic muon passing through the ID during a collision event can be reconstructed as a back-to-back muon pair with opposite electric charges, and the pair can be reconstructed as a displaced \mumu vertex passing all signal selection criteria in this analysis. \mumu vertex.

This cosmic background is suppressed by implementing the cosmic veto, $R_{\mathrm{CR}} < 0.01$ (Section~\ref{sec:event_selection}). The event cut flow in Figure~\ref{subfig:event_cutflow_MC} shows that the cosmic veto is very effective in signal selection, i.e., the signal efficiency loss due to the cosmic veto is negligible. To illustrate the $R_{\mathrm{CR}}$ distribution of cosmic muons, a cosmic control region is defined as follows:

\begin{itemize}
  \item Events are required to fail the cosmic veto.
  \item Events are required to have at least two muons satisfying vertex track selection (Table~\ref{table:vertex_track_selection_simple}).
  \item All other event selections are kept the same as the signal region.
\end{itemize}

In this control region, pairs of two muons with leading $p_{T}$ are studied using the data sample. Figure~\ref{fig:cosmicData} shows the distribution of the pairs in $|\Delta\phi|$ and $\Sigma \eta$. The distribution shows that a significant faction of muons pairs is from cosmic rays and is constraint in small $R_{\mathrm{CR}}$ region.

\begin{figure}[!htb]
    \begin{center}
    \includegraphics[width=0.6\textwidth]{figures/Cosmics/DataValidation/mup_seta_dphi.pdf}
    \label{fig:cosmicData}
    \caption{Distribution of pairs of two muons with leading $p_{T}$ in $|\Delta \phi|$ and $\Sigma \eta$ found in the cosmic control region using the data sample. The sharp peak at $|\Delta\phi| = \pi$ and $\Sigma \eta = 0$ shows that a significant fraction of muon pairs in the region is from cosmic muons.}
\end{center}
\end{figure}

To estimate the cosmic background in the signal region, the distributions of \mumu vertices and pairs from the data sample in small $R_{\mathrm{CR}}$ region are shown in Figure~\ref{fig:cosmicCR}. There are 246 \mumu vertices found in the data sample that pass all of the signal selection except the cosmic veto, and the distribution vanishes around $R_{\mathrm{CR}}=0.004$\mumu, suggesting that the cosmic background is effectively suppressed by the cosmic veto cut of $R_{\mathrm{CR}}=0.01$.

For more accurate estimation of the background, the \mumu pair distribution is normalized to the \mumu vertex distribution as shown in Figure~\ref{fig:cosmicCRb}, and the normalized distribution is extrapolated to the signal region. With this method, the cosmic background is estimated to be 0.27$\pm$0.14 (stat.). This background is about two
orders of magnitude larger than the random crossing background. Therefore, the cosmic muon background is the dominant source of background for this analysis.


\begin{figure}[!htb]
    \centering
    \subfloat[Unscaled\label{fig:cosmicCRa}]{\includegraphics[width = 0.49 \textwidth]{figures/Cosmics/DataValidation/cosmics_unscaled.pdf}}
    \subfloat[Scaled\label{fig:cosmicCRb}]{\includegraphics[width = 0.49 \textwidth]{figures/Cosmics/DataValidation/cosmics_scaled.pdf}}
    \label{fig:cosmicCR} 
	\caption{(a) Distribution of \mumu vertices and pairs from the data sample in small $R_{\mathrm{CR}}$ region. The red distribution represents all \mumu pairs. The black dots represent the subset of pairs forming a \mumu vertex. The same distribution is shown in (b) with \mumu pairs normalized to \mumu vertices.}
\end{figure}



\section{Random-Crossing Background}
\label{sec:bkg:random}

The random-crossing of two uncorrelated tracks can be a major source of the backgrounds in the search for displaced dilepton vertices. This background is expected to increase with more pile up in Run 2.
 
This random-crossing background is estimated by a data-driven method called the \textit{track flipping} (TF). In this method, secondary vertex reconstruction is performed on each pair of tracks from all possible combinations of tracks after one random track from each pair is flipped with respect to the beam spot. Because one track is flipped in each pair of tracks, the resulting vertices provide good estimation for random-crossing background. In addition, another random-crossing background method called the \textit{Event mixing} is used to estimate systematic uncertainty in the background estimation.

The TF and event mixing methods are described in Section~\ref{sec:bkg:random_crossing_tf} and ~\ref{sec:bkg:random_crossing_em}, respectively. In Section~\ref{sec:bkg:random_crossing_MC}, the TF method is tested on the background MC samples, and the result is compared with the corresponding result from the event mixing. In Section~\ref{sec:bkg:random_crossing_data}, the random-crossing background in data is estimated by the TF method, and its systematic uncertainty is estimated by comparing the estimation from the TF and event mixing in Section~\ref{sec:bkg:random_crossing_syst}.

\subsection{Track flipping method}
\label{sec:bkg:random_crossing_tf}

In the TF method, events are selected by the same requirement described in Section~\ref{sec:selection:pre}. From the selected events, ID tracks associated with a muon, electron, or neither, referred as muon, electron, or non-leptonic track, respectively, are selected with the track criteria (Table~\ref{table:vertex_track_selection_simple}) used for the secondary vertexing algorithm. Lepton tracks are required to pass the same selection criteria described in Table~\ref{table:lepton_requirement}. Non-leptonic tracks are required to pass the minimal kinematic selection ($p_{T} > 10$ GeV, $\eta < $2.5 to match with the kinematic selection for leptons.

\begin{figure}[!htb]
    \subfloat[]{\label{subfig:TF_diagram_a}\includegraphics[width=0.50\textwidth]{figures/TF_diagram_a.pdf}}
    \subfloat[]{\label{subfig:TF_diagram_a}\includegraphics[width=0.50\textwidth]{figures/TF_diagram_b.pdf}}
	\centering
	\caption{Description of track flipping method in (a) $r-z$ and (b) $x-y$ plane. Given two tracks $A$ and $B$ originating from a common secondary vertex, one random track is flipped with respect to the beam spot and is denoted as $B'$. The resulting flipped track pair, $AB'$, cannot form a vertex due to the separation in space.}
	\label{fig:TF_diagram}
\end{figure}

From the selected tracks, track pairs are created from all possible combination of muon, electron, or non-leptonic tracks which fall into one of the six categories, \mumu, \ee, \emu, \ex, \mux, or \xx track pair, where x represents a non-leptonic track. For each pair of tracks, one random track is flipped with respect to the beam spot ($\dzero \rightarrow -\dzero, \zzero \rightarrow -\zzero, \phi\rightarrow\phi+\pi, \theta\rightarrow\pi-\theta$), creating a \textit{flipped track pair}. The secondary vertex algorithm used in the reconstruction of data or MC sample is used to reconstruct displaced vertices using these flipped track pairs. Vertex selection cuts similar to the cuts listed in Table~\ref{table:vertex_track_selection_simple} are applied to the vertices found from flipped track pairs. The only differences in the vertex cuts are:

\begin{itemize}
\item Trigger matching is only required for \mumu, \ee, and \emu vertices because non-leptonic tracks cannot be matched to lepton triggers, 
\item Filter matching is only required for \mumu, \ee, and \emu vertices for the same reason.
\end{itemize}

The resulting vertices, referred as \textit{track-flipping vertices}, fall into one of the same six categories as track pairs. Because one random track is flipped, track-flipping vertices are purely from random-crossing of tracks as depicted in Figure~\ref{fig:TF_diagram}. Therefore, track-flipping vertex yields provide a good estimation for random-crossing background of each type. Also, because trigger and filter matchings are not required in the control and validation region, the TF method provides conservative background estimation. 



\subsection{Event mixing method}
\label{sec:bkg:random_crossing_em}

The event mixing method is similar to the TF, but instead of flipping a track from pair of tracks, it combines tracks from different events to create uncorrelated track pairs, i.e., two tracks are not originating from a real vertex.

The event mixing method proceed as follows. First, muon, electron, and non-leptonic tracks that satisfy the track criteria (Table~\ref{table:vertex_track_selection_simple}) are collected from all events, resulting in a collection of all potential seed objects in the sample. Lepton tracks are required to pass the same selection criteria\footnote{Only lepton pairs with an invariant mass greater than $6~\si{\GeV}$ are used in the normalization procedure to remove any contamination from low mass processes.} described in Table~\ref{table:lepton_requirement}. Non-leptonic tracks are required to pass the minimal kinematic selection ($p_{T} > 10$ GeV, $\eta < $2.5 to match with the kinematic selection for leptons.

Pairs of tracks are randomly sampled from the collection. For each track pair, a primary vertex is randomly chosen from all events with a lepton candidate. Primary vertices are required to evaluate the displacement cut and quality requirements of the vertexing algorithm. The secondary vertex reconstruction is performed on each track pair and the associated primary vertex to reconstruct displaced vertices, referred as \textit{event-mixing vertices}.

The ratio of event-mixing vertex yields to the number of track pairs sampled represents the probability, $p_{\mathrm{xing}}$, of two tracks forming a displaced vertex by a random chance. Using $p_{\mathrm{xing}}$ and the total number of track pairs of each type (\mumu, \ee, \emu, \mux, \ex, \xx) in data, the random-crossing background is estimated for each type by, e.g.,

\begin{equation}
    N_{\mumu} = T_{\mumu} \times p_{\mathrm{xing}},
\label{eq:N_Comb_Vertices}
\end{equation}

where $N_{\mumu}$ represent the estimated random-crossing background of \mumu type, and $T_{\mumu}$ represents the total number of \mumu pairs present in data. The random-crossing probability, $p_{\mathrm{xing}}$, is estimated individually for each type of vertices. The details on this method can be found in [LINK TO INTERNAL NOTE].







\subsection{MC study}
\label{sec:bkg:random_crossing_MC}
The TF method is tested using the background MC sample described in Section~\ref{sec:background_mc_sample}, and the resulting vertex yields and distributions are compared with the corresponding result from event mixing method. A representative plot of vertex cut flows in the TF method is shown in Figure~\ref{fig:m_FBE_cutflow_MC} using track-flipping \xx vertices from the background MC samples.

\begin{figure}[!htb]
	\includegraphics[width=0.60\textwidth]{figures/m_FBE_cutflow_MC.pdf}
	\centering
	\caption{Vertex cut flow applied on \xx vertices from the TF method.}
	\label{fig:m_FBE_cutflow_MC}
\end{figure}

The vertex yields from the TF and event mixing method, which represent the estimation for random-crossing background, are compared with the vertex yields from reconstruction in Table~\ref{table:random_vertex_count}. No random-crossing background of \mumu, \ee, or \emu type is expected from the MC samples using both methods. 
 
\begin{table}[!htb]
  \centering
  \begin{tabular}{ c  c c c }
    \hline
    \hline
	Vertex Type					& Track Flipping	        & Event Mixing	        & Background MC Samples \\
    \hline
	\mux						&	1						&	1.3 				&	0					\\
	\ex						    &	0						&	0.3 				&	0					\\
	\xx						    &	741 					&	714.0				&	676 				\\
    \hline
    \hline
  \end{tabular}
  \caption{Comparison of the number of \mux, \ex, and \xx vertices found in the track flipping, estimated from the event mixing, and reconstructed in the background MC samples.}
  \label{table:random_vertex_count}
\end{table}

The \xx vertex yields from the track flipping and the event mixing methods agree within the statistical uncertainty. The \xx vertex distributions of the vertices from the track flipping, event mixing, and from reconstruction are shown in Figure~\ref{fig:random-crossing_vertex_dist}. The distributions of the track flipping and the event mixing agree reasonably well with the distributions of vertices found from the reconstruction.

\begin{figure}[!htb]
    \centering
    \subfloat[]{\label{subfig:random-crossing_MC_M}\includegraphics[width=0.45\textwidth]{figures/m_FBE_M.pdf}}
    \subfloat[]{\label{subfig:random-crossing_MC_chi2ndof}\includegraphics[width=0.45\textwidth]{figures/m_FBE_chi2_ndof.pdf}} \\
    \subfloat[]{\label{subfig:random-crossing_MC_r}\includegraphics[width=0.45\textwidth]{figures/m_FBE_R.pdf}}
    \subfloat[]{\label{subfig:random-crossing_MC_z}\includegraphics[width=0.45\textwidth]{figures/m_FBE_z.pdf}}
    %\subfloat[l]{\label{subfig:random-crossing_l}\includegraphics[width=0.45\textwidth]{figures/m_FBE_l.eps}}
    \caption{Comparison of of (a) vertex mass, (b) $\chi^{2} / \mathrm{DOF}$, (c) transverse, and (d) longitudinal position of \xx vertices reconstructed, found from the track flipping, and estimated using the event mixing. In (c), the red dashed lines indicate the four Pixel layers and the first layer of SCT. The green dotted lines indicate the Inner Support Tube (45.5 mm) and Pixel Support Tube (229 mm).}
    \label{fig:random-crossing_vertex_dist}
\end{figure}












\subsection{Estimating random-crossing background with data sample}
\label{sec:bkg:random_crossing_data}

Random-crossing background is estimated by performing the TF method on the data sample. Following the procedure described in Section~\ref{sec:bkg:random_crossing_tf}, track-flipping vertices are created, and the vertex yields in the control and validation regions are used to make an estimation in the signal region. 

\textbf{Vertex distribution in control region} In the control region, the \xx vertices found from the track flipping method are compared to the vertices reconstructed in the data sample in Figure~\ref{fig:random-crossing_vertex_dist_data}. The distributions shows that the TF method reproduces the data reasonably well including some of the structures of the ID, suggesting that the TF method provides a reasonable estimation of the random-crossing background.

\begin{figure}[!htb]
    \centering
    \subfloat[]{\label{subfig:random-crossing_M}\includegraphics[width=0.45\textwidth]{figures/m_FBE_data_M.pdf}}
    \subfloat[]{\label{subfig:random-crossing_chi2ndof}\includegraphics[width=0.45\textwidth]{figures/m_FBE_data_chi2_ndof.pdf}} \\
    \subfloat[]{\label{subfig:random-crossing_r}\includegraphics[width=0.45\textwidth]{figures/m_FBE_data_R.pdf}}
    \subfloat[]{\label{subfig:random-crossing_z}\includegraphics[width=0.45\textwidth]{figures/m_FBE_data_z.pdf}}
    %\subfloat[l]{\label{subfig:random-crossing_l}\includegraphics[width=0.45\textwidth]{figures/m_FBE_data_l.eps}}
    \caption{Comparison of (a) vertex mass, (b) $\chi^{2} / \mathrm{DOF}$, (c) transverse, and (d) longitudinal position of vertices found from reconstruction with the track-flipping vertices in the control region of the data sample. In (c), the red dashed lines indicate the four Pixel layers and the first layer of SCT. The green dotted lines indicate the Inner Support Tube (45.5 mm) and Pixel Support Tube (229 mm).}
    \label{fig:random-crossing_vertex_dist_data}
\end{figure}

However, because of the limited statistics in lepton pairs in the data sample, the track-flipping vertex yields in the signal region cannot be directly used as random-crossing background estimation. Instead, the track-flipping vertex yields in the control region and validation region are extrapolated to estimate random-crossing background in the signal region using lepton probability, defined as follows:
\begin{itemize}
\item $P(e)$ is defined as the ratio of electrons to inner detector tracks in the entire sample,
\item $P(\mu)$ is defined as the ratio of muons to inner detector tracks in the entire sample,
\end{itemize}
where track requirements described in Table~\ref{table:vertex_track_selection_simple} is required to both leptons and inner detector tracks.

\textbf{Extrapolation from control region} The track-flipping vertex yields in the control region is extrapolated using Eq.~\ref{eq:TF_extrapolation_from_control} to the validation regions to estimate the vertex yields, representing random-crossing background estimation in the validation region. The estimated \mux and \ex vertex yields are compared with the observed track-flipping vertex yield in the region. The ratio of the observed track-flipping vertex to the extrapolated vertex in the validation region is used as scale factors for the estimation in the signal region,

\begin{align}
    %S_{xx\rightarrow \mu x} &= N_{\mu x}^{obs} / N_{\mu x}^{est}, \nonumber \\
    %S_{xx\rightarrow \ex}   &= N_{e x}^{obs} / N_{e x}^{est}
    S_{xx\rightarrow \mu x} &= \frac{N_{\mu x}^{obs}}{N_{\mu x}^{est}}, \nonumber \\
    S_{xx\rightarrow \ex}   &= \frac{N_{e x}^{obs}}{ N_{e x}^{est}}
\label{eq:random_crossing_scale_factor}
\end{align}
where $N_{\mu x}^{obs}$ and $N_{\mu x}^{est}$ ($N_{ex}^{obs}$ and $N_{ex}^{est}$) represent the number of observed track-flipping vertex and the estimated vertex of each type.

The track-flipping vertex yield is then extrapolated to the signal region using Eq.~\ref{eq:TF_extrapolation_from_control} to estimate random-crossing background, and the scale factors defined by Eq.~\ref{eq:TF_scale_factors} are applied to the extrapolation for more accurate estimation of the background.

\textbf{Extrapolation from validation region} Similarly, the track-flipping vertex yields in the validation region is extrapolated to the signal to estimate vertex yields using Eq.~\ref{eq:TF_extrapolation_from_validation}, and the scale factors defined by Eq.~\ref{eq:TF_scale_factors} are applied to the extrapolation to obtain the background estimation.

The lepton probability, track-flipping and reconstructed vertex yields in the control and validation region, the scale factors, and the estimated random-crossing background are summarized in Table~\ref{table:track_flipping}.

\begin{table}[!htb]%
  \centering
  \begin{minipage}[b]{\textwidth}
  \resizebox{\textwidth}{!}{
  \subfloat[Lepton probability]{
    \begin{tabular}[t]{ccc}
        \hline\hline
                & Tracks             & p($\ell$)           \\
         \hline
         x      & $2.47\times10^{7}$ & -                   \\
         $\mu$  & $5.23\times10^{4}$ & $2.10\times10^{-3}$ \\
         $e$    & $3.63\times10^{4}$ & $1.46\times10^{-3}$ \\
         Sum    & $2.48\times10^{7}$ & - \\
        \hline\hline
    \end{tabular}
  }
  \qquad
  \subfloat[Vertex yields in the control and validation region]{
    \begin{tabular}[t]{ccc}
        \hline\hline
                & Tracks-flipping    & Data                \\
         \hline
         $xx$   & 1255               & 1346                \\
         $\mu x$& 3                  & 4                   \\
         $ex$   & 1                  & 0                   \\
        \hline\hline
    \end{tabular}
  }%
  \qquad
  \subfloat[Scale factors]{
    \begin{tabular}[t]{cc}
        \hline\hline
         Type                      & SF      \\
         \hline
         $S_{xx\rightarrow\mu x}$  & 0.82              \\
         $S_{xx\rightarrow ex}$    & 0.19              \\
        \hline\hline
    \end{tabular}
  }
  }
  \end{minipage}


  \subfloat[Extrapolation from control region]{
    \begin{tabular}[t]{ccc}
        \hline\hline
                         & Estimation          & Applying SF         \\
         \hline
         $N_{\mu x}$     & 4                   & -                   \\
         $N_{e x}$       & 5                   & -                   \\
         $N_{\mu\mu}$    & $2.69\times10^{-3}$ & $1.79\times10^{-3}$ \\
         $N_{ee}$        & $5.56\times10^{-3}$ & $1.99\times10^{-4}$ \\
         $N_{e\mu}$      & $7.73\times10^{-3}$ & $1.96\times10^{-3}$ \\
        \hline\hline
    \end{tabular}
  }%
  \qquad
  \subfloat[Extrapolation from validation region]{
    \begin{tabular}[t]{ccc}
        \hline\hline
                         & Estimation          & Applying SF         \\
         \hline
         $N_{\mu\mu}$    & $2.19\times10^{-3}$ & $1.79\times10^{-3}$ \\
         $N_{ee}$        & $1.05\times10^{-3}$ & $1.99\times10^{-4}$ \\
         $N_{e\mu}$      & $3.89\times10^{-3}$ & $1.96\times10^{-3}$ \\
        \hline\hline
    \end{tabular}
  }%
  %}
  
  \caption{Random-crossing background estimation in data by the TF method}%
  \label{table:track_flipping}
\end{table}


%The systematic uncertainty in the track flipping method is estimated by studying the variance in background estimation from different variations of the track flipping method. In addition to the transformation described in the previous section ($d_{0}\rightarrow -d_{0}, z_{0}\rightarrow -z_{0}, \phi\rightarrow\phi-\pi, \theta\rightarrow\pi-\theta$), the following track transformations are considered.
%\begin{itemize}
%\item \textbf{Same sign d0}: $d_{0}\rightarrow d_{0}, z_{0}\rightarrow -z_{0}, \phi\rightarrow\phi-\pi, \theta\rightarrow\pi-\theta$
%\item \textbf{Same sign $\theta$}: $d_{0}\rightarrow -d_{0}, z_{0}\rightarrow -z_{0}, \phi\rightarrow\phi-\pi, \theta\rightarrow\theta$
%\item \textbf{90 degrees rotation in $\phi$}: $d_{0}\rightarrow -d_{0}, z_{0}\rightarrow -z_{0}, \phi\rightarrow\phi-\frac{\pi}{2}, \theta\rightarrow\theta$
%\end{itemize}

\subsection{Systematic uncertainty in random-crossing background}
\label{sec:bkg:random_crossing_syst}

\chapter{Systematic Uncertainties}
\label{chap:syst}

The systematic uncertainties in the signal efficiency is computed by estimating the contribution of each efficiency factor in Eq.~\ref{eq:OverallEff} when possible. 

The systematic uncertainty in track and vertex reconstruction is studied using \Ks in Section~\ref{sec:syst:vertexing}. The systematic uncertainty in trigger is studied in Section~\ref{sec:syst:trigger} using tag-and-probe method with $Z\rightarrow \ee,\mumu$ events. %The systematic uncertainty in lepton reconstruction and identification cannot be estimated using data since no known particle exists in nature that has a long lifetime and mass larger than 10 GeV.



\section{Systematic Uncertainty in Track and Vertex Reconstruction}
\label{sec:syst:vertexing}

In a typical analysis, the systematic uncertainty in track and vertex reconstruction is estimated by the Inner Tracking Combined Performance group using the standard tracking setup. This result cannot be directly used for this analysis due to the special reconstruction setup described in Chapter~\ref{chap:reco}. Instead, the systematic uncertainty in track and vertex reconstruction is estimated by comparing vertex yields between the data and the MC samples using the process, $K_{S}\rightarrow\pi^{+}\pi^{-}$. This process is ideal for comparing the efficiencies in the data and the MC samples due to the long lifetime ($c\tau \sim26.8~\si{\mm}$) of \Ks and the decay mode that leaves two charged tracks. However, there are intrinsic differences between \Ks and signal long-lived particles such as mass and $p_{T}$. In order to understand the validity and the limitation of this method, the kinematic distributions of \Ks and $Z'$ candidates are compared in Appendix~\ref{app:syst_Ks_Zp}.

Tracks originating from a \Ks decay can be reconstructed by either the standard tracking (ST) or the LRT algorithm, resulting in three categories of \Ks vertices: vertices with two \textit{standard tracks}, one standard track and one \textit{large-radius track}, and two large-radius tracks. \Ks vertex yields in each category can be expressed by total \Ks produced in a sample and tracking and vertexing efficiency,

\begin{align}
    N_{\mathrm{ST}}  &= K_{S}~\mathrm{produced} \times (\epsilon_{\mathrm{ST}})^{2} \cdot \epsilon_{\mathrm{vertexing~on~ST}}, \nonumber \\
    N_{\mathrm{ST+LRT}}  &= K_{S}~\mathrm{produced} \times (\epsilon_{\mathrm{ST}} \cdot \epsilon_{\mathrm{LRT}}) \cdot \epsilon_{\mathrm{vertexing~on~LRT}}, \nonumber \\
    N_{\mathrm{LRT}}&= K_{S}~\mathrm{produced} \times (\epsilon_{\mathrm{LRT}})^{2} \cdot \epsilon_{\mathrm{vertexing~on~LRT}},
\label{eq:Ks_eq1}
\end{align}
%
where $N$ represents vertex yields in each category. $\epsilon_{\mathrm{ST}}$ and $\epsilon_{\mathrm{LRT}}$ represent track reconstruction efficiency in the ST and the LRT, respectively. $\epsilon_{\mathrm{vertexing~on~ST}}$ and $\epsilon_{\mathrm{vertexing~on~LRT}}$ represents two-track vertex reconstruction efficiency on two standard or large-radius tracks, respectively.

In order to compare the efficiency in data and MC samples, the double ratio of data to MC is taken,
\begin{equation}
\dfrac{N_{\mathrm{LRT}} / N_{\mathrm{ST}}}{N_{\mathrm{LRT}}^{\mathrm{MC}} / N_{\mathrm{ST}}^{\mathrm{MC}}},
\label{eq:Ks_eq2}
\end{equation}
%
where quantities estimated using MC samples are denoted by MC, and quantities estimated using data are denoted otherwise. Using Eq.~\ref{eq:Ks_eq1} and the double ratio, the systematic uncertainty in track and vertex reconstruction is expressed as,

%\begin{align}
%    \dfrac{\dfrac{N_{LRT}}{N_{ST}}}{\dfrac{N'_{LRT}}{N'_{ST}}}  &= 
%        \dfrac
%        {\dfrac{(\epsilon_{\mathrm{LRT}})^{2} \cdot \epsilon_{\mathrm{vertexing~on~LRT}}}{(\epsilon_{\mathrm{ST}})^{2} \cdot \epsilon_{\mathrm{vertexing~on~ST}}}}
%        {\dfrac{(\epsilon'_{\mathrm{LRT}})^{2} \cdot \epsilon'_{\mathrm{vertexing~on~LRT}}}{(\epsilon'_{\mathrm{ST}})^{2} \cdot \epsilon'_{\mathrm{vertexing~on~ST}}}}
%\label{eq:Ks_eq2}
%\end{align}

\begin{align}
    \dfrac{N_{\mathrm{LRT}} / N_{\mathrm{ST}}}{N_{\mathrm{LRT}}^{\mathrm{MC}} / N_{\mathrm{ST}}^{\mathrm{MC}}} \cdot &=
    \dfrac{\dfrac{(\epsilon_{\mathrm{LRT}})^{2} \cdot \epsilon_{\mathrm{vertexing~on~LRT}}}{(\epsilon_{\mathrm{ST}})^{2} \cdot \epsilon_{\mathrm{vertexing~on~ST}}}}{\dfrac{(\epsilon_{\mathrm{LRT}}^{\mathrm{MC}})^{2} \cdot \epsilon_{\mathrm{vertexing~on~LRT}}^{\mathrm{MC}}}{(\epsilon_{\mathrm{ST}}^{\mathrm{MC}})^{2} \cdot \epsilon_{\mathrm{vertexing~on~ST}}^{\mathrm{MC}}}} 
    \\[10pt]
    \Big( \dfrac{\epsilon_{\mathrm{LRT}}}{\epsilon_{\mathrm{LRT}}^{\mathrm{MC}}}\Big)^{2} \cdot
    \Big( \dfrac{\epsilon_{\mathrm{vertexing~on~LRT}}}{\epsilon_{\mathrm{vertexing~on~LRT}}^{\mathrm{MC}}}\Big) &=
    \Big( \dfrac{N_{\mathrm{LRT}} \cdot N_{\mathrm{ST}}^{\mathrm{MC}}}{N_{\mathrm{LRT}}^{\mathrm{MC}} \cdot N_{\mathrm{ST}}} \Big) \cdot
    \Big( \dfrac{\epsilon_{\mathrm{ST}}}{\epsilon_{\mathrm{ST}}^{\mathrm{MC}}}\Big)^{2} \cdot
    \Big( \dfrac{\epsilon_{\mathrm{vertexing~on~ST}}}{\epsilon_{\mathrm{vertexing~on~ST}}^{\mathrm{MC}}}\Big).
\label{eq:Ks_eq3}
\end{align}

Events are selected using the same event selection described in Section~\ref{sec:selection:pre} except trigger filters as the high-$p_{T}$ photon or muon triggers are not suitable for \Ks study. From the selected events, \Ks candidates, referred as \Ks vertices, are selected by applying \Ks vertex selection to secondary vertices in the events. The \Ks vertex selection is similar to the $Z'$ signal vertex selection, but for consistency with \Ks study in Run I and background reduction, additional vertex cuts described in Ref.~\cite{Aad:2011hd} are applied. The mass window of 0.35 to 0.65 GeV is used in the \Ks vertex selection. The difference between \Ks and $Z'$ vertex selections are summarized in Table~\ref{table:ks_vertex_cut}. %Figure~\ref{fig:Ks_vertex_cutflow} shows \Ks vertex cut flow in the data and the background MC samples.

\begin{table}[!htb]
%\begin{table}[tb]
  \centering
  \begin{tabular}{ c c c }
    \hline
    \hline
                    & $Z'$                                                  & \Ks                               \\
    \hline
    Trigger         & Photon or muon trigger (Table~\ref{table:triggers})    & -                                     \\
    Vertex type     & \mumu, \emu, \ee                                      & \xx                                   \\
    Mass (GeV)      & $> 10.0$                                              & $[0.35,0.65]$                         \\
    Additional cut  & -                                                     & \Ks selection~\cite{Aad:2011hd}   \\
    \hline
    \hline
  \end{tabular}
  \caption{Comparison of $Z'$ and \Ks vertex selections.}
  \label{table:ks_vertex_cut}
\end{table}

%\begin{figure}[!htb]
%	\includegraphics[width=0.60\textwidth]{figures/m_syst_Ks_cf.pdf}
%	\centering
%	\caption{Vertex cut flow applied on \Ks vertices in the data and the MC samples}
%	\label{fig:Ks_vertex_cutflow}
%\end{figure}

After applying the event and \Ks vertex selection, the \Ks vertex distributions are compared between the data and the MC samples in Figure~\ref{fig:Ks_data_MC}. The data sample is normalized to the MC samples which have limited statistics. Only \Ks vertices with two large-radius tracks are shown. The distributions show good agreement in the invariant mass, $p_{T}$, transverse, longitudinal position, and decay length of the vertices, except the pile-up distribution as expected.


\begin{figure}[!htb]
    \centering
    \subfloat[]{\label{subfig:Ks_m}\includegraphics[width=0.45\textwidth]{figures/m_syst_Ks_M.pdf}}
    \subfloat[]{\label{subfig:Ks_pt}\includegraphics[width=0.45\textwidth]{figures/m_syst_Ks_pt.pdf}} \\
    \subfloat[]{\label{subfig:Ks_mu}\includegraphics[width=0.45\textwidth]{figures/m_syst_Ks_mu.pdf}}
    \subfloat[]{\label{subfig:Ks_r}\includegraphics[width=0.45\textwidth]{figures/m_syst_Ks_r.pdf}}  \\
    \subfloat[]{\label{subfig:Ks_z}\includegraphics[width=0.45\textwidth]{figures/m_syst_Ks_z.pdf}} 
    \subfloat[]{\label{subfig:Ks_l}\includegraphics[width=0.45\textwidth]{figures/m_syst_Ks_l.eps}} \\
    \caption{Comparison of the (a) invariant mass, (b) $p_{T}$, (c) $\mu$, (d) transverse, (e) longitudinal position, and (f) decay length of \Ks vertex with two large-radius tracks in the data with the MC samples. Data is normalized to MC. In (d), the red dashed lines indicate the four pixel layers and the first layer of SCT. The green dotted lines indicate the Inner Support Tube (45.5 mm) and Pixel Support Tube (229 mm). MC sample is reweighted to the pile-up distribution in data.}
    \label{fig:Ks_data_MC}
\end{figure}

\Ks vertices found in the data and the MC samples are binned in decay radius, $r$, and the \Ks yields in each bin are estimated after subtracting the background contributions using side-bands (350 - 450 GeV, 550 - 650 GeV) in invariant mass distribution. Figure~\ref{fig:Ks_mass} shows the mass distribution of \Ks vertex with two large-radius tracks from the data and the MC samples. The figure shows that backgrounds are small and uniform in the mass window, and the mass distributions are in good agreement between the data and the MC samples.

%\begin{figure}[tb]
\begin{figure}[!htb]
    \centering
    \subfloat[]{\label{subfig:Ks_mass1}\includegraphics[width=0.50\textwidth]{figures/m_syst_Ks_normalized_LRT_R1.pdf}}
    \subfloat[]{\label{subfig:Ks_mass2}\includegraphics[width=0.50\textwidth]{figures/m_syst_Ks_normalized_LRT_R2.pdf}} \\
    \subfloat[]{\label{subfig:Ks_mass3}\includegraphics[width=0.50\textwidth]{figures/m_syst_Ks_normalized_LRT_R3.pdf}} 
    \subfloat[]{\label{subfig:Ks_mass4}\includegraphics[width=0.50\textwidth]{figures/m_syst_Ks_normalized_LRT_R4.pdf}} \\
    \subfloat[]{\label{subfig:Ks_mass5}\includegraphics[width=0.50\textwidth]{figures/m_syst_Ks_normalized_LRT_R5.pdf}} 
    \caption{Representative distributions of the invariant mass of \Ks candidates with two large-radius tracks for (a) 2 < $r$ < 100 mm, (b) 100 $<r<$ 150 mm, (c) 150 $<r<$ 200 mm, (d) 200 $<r<$ 250 mm, and (e) 250 $<r<$ 300 mm in the data and the MC samples. Data is normalized to MC samples.}
    \label{fig:Ks_mass}
\end{figure}

The vertex yields of \Ks with two large-radius tracks are compared between data and MC samples in Figure~\ref{fig:Ks_double_ratio}, where the data is normalized by a factor of $N_{\mathrm{ST}} / N_{\mathrm{ST}}^{\mathrm{MC}}$ where $N_{\mathrm{ST}}$ and $N_{\mathrm{ST}}^{\mathrm{MC}}$ are integrated over $r$. The ratio between two, representing the double ratio (Eq.~\ref{eq:Ks_eq2}), is shown in the lower pane.

\begin{figure}[!htb]
	\includegraphics[width=0.50\textwidth]{figures/m_syst_Ks_ratio_R.pdf}
	\centering
	\caption{The radial distribution of vertex yields of \Ks with two large-radius tracks. Data is normalized to the MC.} %The lower pane shows the double ratio defined in Eq.~\ref{eq:Ks_eq2}.}
	\label{fig:Ks_double_ratio}
\end{figure}

In estimating the systematic uncertainty, largest discrepancies ($\sim0.95$, and $\sim1.05$), shown in the first and fourthbin, are taken as a conservative estimate of $(N_{\mathrm{LRT}} \cdot N_{\mathrm{ST}}^{\mathrm{MC}}) / (N_{\mathrm{LRT}}^{\mathrm{MC}} \cdot N_{\mathrm{ST}})$. The results from previous studies show that the systematic uncertainty in the standard tracking is $2\%$~\cite{ATL-PHYS-PUB-2015-051}, and the systematic uncertainty in secondary vertex reconstruction using standard tracks is $1\%$~\cite{Aaboud:2215485}. The study on secondary vertex reconstruction is performed using min-bias data, but relatively uniform distribution of efficiency as a function of pile-up (Figure~\ref{subfig:vertex_dist_mu}) suggests that this result is still applicable for data with $\langle \mu \rangle\approx$ 24. Using Eq.~\ref{eq:Ks_eq3} together with these results, the systematic uncertainty in track and vertex reconstruction in the LRT is estimated to be $10\%$.


%\subsection{Systematic uncertainty in Trigger Efficiency}
%\label{sec:syst_trigger}
%
%The efficiencies of all analysis triggers have been measured with tag-and-probe studies on data and MC as reported in Section~\ref{sec:trigger_efficiency}. Systematic uncertainties of tag-and-probe methods on the plateau efficiencies are usually of the order one percent and, therefore, negligible compared to other sources like tracking and vertexing. For this reason, no attempt was made to estimate systematic uncertainty in trigger efficiency.
%
%

\subsection{Systematic uncertainties on background estimations}
\label{sec:syst_bkg}

There are two main sources of background in this analysis; cosmic muon background and random crossing background. The cosmic muon background estimation is outlined in Section~\ref{sec:bkg:cosmic}. From this technique the estimate is $0.27 \pm 0.14 \textrm{ (stat.)}$ events. This $\pm 0.14$ is taken to be the total uncertainty on this background. In the case of the random crossing, there are two estimates event mixing and track flipping giving $2.40\times10^{-3}$ and $3.95 \times 10^{-3}$ respectively. For these the larger estimate of $3.95 \times 10^{-3}$ is used as the central value and the difference of $1.5\times10^{-3}$ is used as the uncertainty on this value.











 
%\backmatter
% We use BIBTeX for the bibliography---you don't have to
%\printbibliography
 
% \nocite{*} % To display all refs, even uncited refs (useful when editting)
%\bibliography
\cleardoublepage
\phantomsection
\addcontentsline{toc}{chapter}{Bibliography}
\bibliography{src/ref}
\bibliographystyle{ieeetr}
%\bibliographystyle{unsrt} % use your favorite BIBTeX style
 
% If for some reason you are anti-BIBTeX, then you would use the
% following instead of the above:
%\begin{thebibliography}{99}
% ...
%\end{thebibliography}
 
 
% Note: GS 2010 requires bibliography/references _before_ the appendix
% if you believe their guidelines; however, conversations with GS
% staff suggests _they don't care_. Go figure. So do what you like.
 
% COMMENTED OUT JUNE 2ND 2017
%\appendix
%\renewcommand{\thesection}{\Alph{section}}
%\clearpage
%\include{appendices/app_test}
 
\end{document}
 
